\section*{27. Historical Beer}
\addcontentsline{toc}{section}{27. Historical Beer}
\textit{A categoria Historical Beer contém estilos que quase desapareceram nos tempos modernos ou que eram muito mais populares no passado e agora são conhecidos apenas por meio de recriações. Esta categoria também pode ser usada para cervejas tradicionais ou nativas de importância cultural em certos países. Colocar uma cerveja na categoria histórica não significa que ela não esteja sendo produzida atualmente, apenas que é um estilo pouco difundido ou talvez esteja em processo de redescoberta por cervejeiros artesanais.}

\textit{A Historical Beer pode ser um estilo pouco difundido, atualmente produzido comercialmente ou não, que não está presente nas diretrizes de estilo como um estilo clássico. Pode ser que não se tenha ouvido falar dele, que nunca tenha sido visto em uma competição ou que não se tenha dados suficientes para preparar um conjunto razoável de diretrizes de julgamento. Se é um estilo com um nome que é ou foi realmente usado, provavelmente entrará nessa categoria. Este estilo não é para cervejas experimentais que nunca foram produzidas ou para outros estilos clássicos com ingredientes especiais adicionados.}

\textit{Qualquer Historical Beer listada nesta categoria ou contida na lista de Estilos Provisórios é considerada um Estilo Clássico para fins de inscrição nas categorias de cerveja do Tipo Especial com ingredientes adicionados (frutas, especiarias, madeira, defumado, etc). Isso significa que uma Historical Beer pode ser usada como um estilo base para cervejas do Tipo Especial sem tornar automaticamente a cerveja Experimental. O BJCP aceita submissões de Estilos Históricos bem pesquisados que possam ser apropriados para a lista de Estilos Provisórios em nosso site ou para futura inclusão nestas Diretrizes.}

\textit{Instruções de inscrição: O participante deve especificar um estilo com uma descrição fornecida pelo BJCP da lista abaixo ou especificar um estilo de cerveja histórico diferente que não esteja descrito em outras partes destas diretrizes. No caso de um estilo que mudou substancialmente ao longo dos anos (como Porter ou Stout), o participante pode especificar um estilo BJCP existente, bem como a época (por exemplo, 1820 English Porter).}

\textit{Quando o participante especifica qualquer estilo que não esteja na lista fornecida pelo BJCP nesta categoria ou na lista de Estilos Provisórios, o participante deve fornecer uma descrição do estilo para os juízes com detalhes suficientes para permitir que a cerveja seja julgada. Se uma cerveja for inscrita com apenas o nome do estilo e nenhuma descrição, é muito improvável que os juízes entendam como julgá-la. Exemplos atualmente definidos: Kellerbier, Kentucky Common, Lichtenhainer, London Brown Ale, Piwo Grodziskie, Pre-Prohibition Lager, Pre-Prohibition Porter, Roggenbier, Sahti.}
