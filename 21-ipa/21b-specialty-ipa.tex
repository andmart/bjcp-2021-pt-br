\phantomsection
\subsection*{21B. Specialty IPA}
\addcontentsline{toc}{subsection}{21B. Specialty IPA}
\textbf{Impressão Geral}: Reconhecível como IPA pelo equilíbrio - uma cerveja orientada para o lúpulo, amarga e seca - com a presença de algo a mais para distingui-la das categorias padrão. Deve ser fácil de ser bebida, independente da forma. Aspereza e peso excessivos são tipicamente falhas, assim como sabores fortes que geram conflitos entre o lúpulo e os outros ingredientes especiais.

\textbf{Aparência}: A cor depende do tipo específico de Specialty IPA. A maioria deve ser límpida, mas uma leve turbidez é aceitável na maioria dos estilos. Cervejas mais escuras podem ser opacas, tornando a limpidez irrelevante. Colarinho bom e persistente, com cor dependente do estilo específico da Specialty IPA.

\textbf{Aroma}: Um detectável aroma de lúpulo é necessário; a caracterização do lúpulo depende do tipo específico da Specialty IPA. Outros compostos aromáticos podem estar presentes; mas o aroma de lúpulo é tipicamente o elemento mais forte.

\textbf{Sabor}: Variável em cada estilo, com a qualidade de cada componente dependente da Specialty IPA específica. Sabor de lúpulo tipicamente de médio baixo a alto. Amargor de lúpulo tipicamente de médio alto a muito alto. Sabor de malte geralmente de baixo a médio. Comumente tem um final de seco a médio seco. Algum sabor de álcool limpo pode ser notado em versões mais fortes. Vários tipos de Specialty IPAs podem apresentar características adicionais de malte e levedura, dependendo do estilo.

\textbf{Sensação na Boca}: Macia. Corpo de médio leve a médio. Carbonatação, geralmente, média. Um aquecimento alcoólico, em segundo plano, pode ser percebido em versões mais fortes.

\textbf{Comentários}: Os participantes podem usar esta categoria para uma versão com intensidade diferente de uma IPA definida por sua própria subcategoria BJCP (por exemplo, American or English IPA com intensidade session - exceto quando já existir uma subcategoria BJCP para esse estilo (por exemplo, Double [American] IPA). Uma Session IPA é uma 21B Specialty IPA com um estilo base de 21A American IPA com teor alcoólico session. Uma Double IPA é da Categoria 22A e não da 21B.

\textbf{Comparação de Estilos}: Tipos atualmente definidos: Belgian IPA, Black IPA, Brown IPA, Red IPA, Rye IPA, White IPA e Brut IPA

\textbf{Instruções para Inscrição}: O participante deve especificar o teor alcoólico (session, standard, double); se nenhum teor alcoólico for especificada, o \textit{standard} será assumido. O participante deve precisar o tipo específico de Specialty IPA a partir da lista de estilos atualmente definidos e identificados nas diretrizes de estilos ou conforme definido como estilos provisórios no site do BJCP; OU o participante deve descrever o tipo de Specialty IPA e suas principais características em forma de comentário para que os juízes saibam o que esperar. Os participantes podem especificar as variedades de lúpulo usadas, se os participantes sentirem que os juízes podem não reconhecer as características de varietais de lúpulos mais recentes. Os participantes podem especificar uma combinação de tipos de IPA definidos (por exemplo, Black Rye IPA) sem fornecer descrições adicionais. Classificações de teor alcoólico: Session - ABV: 3,0 - 5,0\% Standard - ABV: 5,0 - 7,5\% Double - ABV: 7,5 - 10,0\%

\textbf{Estatísticas}: Variável por tipo, veja estilos individuais

\textbf{Última Revisão}: Specialty IPA (2015)