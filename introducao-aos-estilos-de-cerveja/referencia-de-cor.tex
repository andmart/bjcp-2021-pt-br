\subsection*{Referência de Cor}
\addcontentsline{toc}{subsection}{Referência de Cor}

\textit{Observe que SRM é mais uma mensuração da densidade da cor da cerveja do que sua matriz, tom ou nuance. Tenha isso em mente ao tentar descrever cervejas utilizando somente o número SRM. Dentro deste guia de estilos, os descritores de cor da cerveja normalmente se aproximam desta tabela de mapeamento dos valores SRM.}\\

\begin{tabular}{ l l }
Palha & 2-3 \\
Amarelo & 3-4 \\
Dourado & 5-6 \\
Âmbar & 6-9 \\
Âmbar profundo/cobre leve & 10-14 \\
Cobre & 14-17 \\
Cobre profundo/Marrom claro & 17-18 \\
Marrom & 19-22 \\
Marrom profundo & 22-30 \\
Marrom muito profundo & 30-35 \\
Preto & 30+ \\
Preto, opaco & 40+
\end{tabular}\\

Não seja demasiadamente pedante ao atribuir nomes de cores às referências SRM percebidas, as condições para a observação diversas vezes influência fortemente a percepção, percepções individuais variam e tons fora do espectro amarelo para marrom podem distorcer resultados. Em caso de contradições aparentes, opte pelas descrições com nomes ao invés de números.
