\phantomsection
\subsection*{11C. Strong Bitter}
\addcontentsline{toc}{subsection}{11C. Strong Bitter}
\textbf{Impressão geral}: Uma cerveja britânica amarga de força de média a moderadamente forte. O equilíbrio pode variar e possuir igualdade na intensidade do malte e lúpulo ou ser um pouco amargo. A facilidade em beber é um componente crítico do estilo. Um estilo bastante amplo que permite uma interpretação considerável pelo cervejeiro.

\textbf{Aroma}: Aroma de lúpulo de moderadamente baixo a moderadamente alto, normalmente com caráter floral, terroso, resinoso ou frutado. Aroma de malte de médio a médio-alto, opcionalmente com um componente de caramelo de baixo a moderado. Ésteres frutados de médio-baixo a médio-alto. Geralmente sem diacetil, embora níveis muito baixos sejam permitidos.

\textbf{Aparência}: Cor de âmbar claro a cobre profundo. Limpidez de boa a brilhante. Colarinho de branco a quase branco de baixo a moderado. Um colarinho baixo é aceitável quando a carbonatação também é baixa.

\textbf{Sabor}: Amargor de médio a médio-alto com sabor do malte evidente e em caráter de suporte. O perfil de malte é tipicamente de pão, biscoito, nozes ou leve tosta, opcionalmente, tem um sabor de caramelo ou toffee de moderadamente baixo a moderado. Sabor de lúpulo de moderado a moderadamente alto, tipicamente com caráter floral, terroso, resinoso ou frutado. O amargor e o sabor do lúpulo devem ser perceptíveis, mas não devem dominar totalmente os sabores do malte. Ésteres frutados de moderadamente baixos a altos. Opcionalmente pode ter baixas quantidades de álcool. Final de meio seco a seco. Geralmente sem diacetil, embora níveis muito baixos sejam permitidos.

\textbf{Sensação na Boca}: Corpo de médio-leve a médio-cheio. Carbonatação de baixa a moderada, embora as versões engarrafadas sejam mais altas. Versões mais fortes podem ter um leve calor alcoólico, mas esse caráter não deve ser muito alto.

\textbf{Comentários}: Na Inglaterra hoje, “ESB” é uma marca registrada da Fullers, e ninguém pensa nela como uma classe genérica de cerveja. É uma cerveja única (mas muito conhecida) que tem um perfil de malte muito forte e complexo não encontrado em outros exemplos, muitas vezes levando os juízes a penalizar excessivamente as tradicionais strong bitters inglesas. Na América, ESB foi associado a uma ale britânica maltada, amarga, avermelhada, de força padrão (para os EUA), e é um estilo popular de cerveja artesanal. Isso pode fazer com que alguns juízes pensem nas ESBs de cervejarias americanas como representantes desse estilo.

\textbf{História}: Veja os comentários na introdução da categoria. Strong bitters podem ser vistas como uma versão de maior densidade das best bitters (embora não necessariamente “mais premium”, já que as best bitters são tradicionalmente o melhor produto das cervejeiras). As British pale ales são geralmente consideradas uma cerveja premium, pálida e amarga com teor alcoólico tipo exportação que se aproxima grosseiramente das strong bitters, embora reformulada para o engarrafamento (incluindo maiores níveis de carbonatação). Enquanto a pale ale britânica moderna é considerada uma bitter engarrafada, historicamente os estilos eram diferentes.

\textbf{Ingredientes}: Maltes pale ale, amber ou crystal, pode ser usado um toque de malte black para ajuste de cor. Pode usar adjuntos como açúcar, milho ou trigo. Lúpulos de finalização ingleses são os mais tradicionais, mas qualquer lúpulo é aceitável; se forem usados lúpulos americanos, é necessário um leve toque. Levedura britânica com caráter presente. As versões de Burton usam água com níveis de sulfato de médio a alto, o que pode aumentar a percepção de secura e adicionar um aroma e sabor mineral ou sulfuroso.

\textbf{Comparação de estilos}: Sabores de malte e lúpulo mais evidentes do que special ou best bitter, assim como mais álcool. Versões mais fortes podem se sobrepor um pouco com as British Strong Ales, embora as Strong Bitters costumem ser mais claras e mais amargas. Mais sabor de malte (principalmente caramelo) e ésteres do que uma American Pale Ale, com caráter de lúpulo de finalização diferente.

\begin{tabular}{@{}p{35mm}p{35mm}@{}}
  \textbf{Estatísticas}: & OG: 1,048 - 1,060 \\
  IBU: 30 - 50  & FG: 1,010 - 1,016  \\
  SRM: 8 - 18  & ABV: 4,6\% - 6,2\%
\end{tabular}

\textbf{Exemplos Comerciais}: Bass Ale, Bateman’s Triple XB, Robinsons Trooper, Samuel Smith’s Organic Pale Ale, Shepherd Neame Bishop's Finger, Summit Extra Pale Ale.

\textbf{Última Revisão}: Strong Bitter (2015)

\textbf{Atributos de Estilo}: amber-ale-family, amber-color, bitter, british-isles, session-strength, top-fermented, traditional-style