\phantomsection
\subsection*{24B. Belgian Pale Ale}
\addcontentsline{toc}{subsection}{24B. Belgian Pale Ale}

\textbf{Impressão Geral}: Uma cerveja belga de alta fermentação, maltada, de força média, moderadamente amarga, sem dry-hopping e sem sabores fortes. Uma cerveja de cor de acobreada que não traz junto o caráter agressivo de levedura ou acidez de muitas cervejas belgas; tem bom equilíbrio entre o maltado e o frutado, tendo muitas vezes um perfil de pão e de tostado.

\textbf{Aparência}: Cor âmbar a cobre. Limpidez é muito boa. Colarinho branco cremoso e sólido. Bem carbonatada.

\textbf{Aroma}: Aroma moderado de malte de pão, que pode incluir notas tostadas, de biscoito ou de nozes, possivelmente com um leve toque de caramelo ou mel. Frutado de moderado a moderadamente alto complementa o malte e é sugestivo de pêra, laranja, maçã ou limão e, às vezes, de frutas de caroço mais escuras, como ameixas. Caráter de lúpulo picante, herbal ou floral de baixo a moderado. Baixo teor de fenólicos apimentados e condimentados são opcionais. O caráter do lúpulo é mais baixo em equilíbrio do que o malte e o frutado.

\textbf{Sabor}: Tem um sabor inicial macio, suave e moderadamente maltado com um perfil variável de tostado, de biscoito, de nozes, caramelo leve ou notas de mel. Frutado de moderado a moderadamente alto, com caráter de pêra, laranja, maçã ou limão. Caráter de lúpulo picante, herbal ou floral de médio baixo a baixo. Amargor médio alto a médio baixo, reforçado por fenólicos apimentados opcionais de baixo a muito baixo. Final de seco a equilibrado, com o lúpulo se tornando mais pronunciado no retrogosto daquelas com final mais seco. No geral bastante equilibrado, sem nenhum componente único de alta intensidade; inicialmente, o malte e o frutado são mais destacados, com um suporte do amargor e um caráter seco no final.

\textbf{Sensação na Boca}: Corpo médio a médio leve. Paladar suave. O nível de álcool é contido e qualquer caráter de aquecimento deve ser baixo, quando presente. Carbonatação média a média alta.

\textbf{Comentários}: Mais comumente encontrado nas províncias flamengas de Antuérpia, Brabante, Hainaut e Flandres Oriental. A Spéciale Belge Ale (Belgian Special Ale) na Bélgica.

\textbf{História}: Surgida após uma competição em 1904 para criar uma cerveja especial regional para competir com cervejas britânicas importadas e lagers continentais. De Koninck da Antuérpia é o exemplo moderno mais conhecido, fazendo a cerveja desde 1913.

\textbf{Ingredientes}: Conjunto de grãos variável com caráter de maltes pale, caramelo. Sem adjuntos. Lúpulos ingleses ou continentais. Levedura frutada com baixo teor de fenóis.

\textbf{Comparação de Estilos}: Bastante semelhante às pale ales da Inglaterra (11C Strong Bitter), tipicamente com um caráter de levedura ligeiramente diferente e um perfil de malte mais variado. Menos caráter de levedura do que muitas outras cervejas belgas, no entanto.

\begin{tabular}{@{}p{35mm}p{35mm}@{}}
  \textbf{Estatísticas} & OG: 1,048 - 1,054 \\
  IBU: 20 - 30  & FG: 1,010 - 1,014  \\
  SRM: 8 - 14  & ABV: 4,8\% - 5,5\%
\end{tabular}

\textbf{Exemplos Comerciais}: De Koninck Bolleke, De Ryck Special, Palm, Palm Dobble.

\textbf{Última Revisão}: Pale Ale Belga (2015)

\textbf{Atributos de Estilo}: amber-color, balanced, pale-ale-family, standard-strength, top-fermented, traditional-style, western-europe