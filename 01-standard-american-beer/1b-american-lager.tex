\phantomsection
\subsection*{1B. American Lager}
\addcontentsline{toc}{subsection}{1B. American Lager}

\textbf{Impressões Gerais}: Uma cerveja lager muito clara, altamente carbonatada, de corpo baixo, bem atenuada, com um sabor neutro e baixo amargor. Servida bem gelada, muito refrescante e para matar a sede.

\textbf{Aparência}: Cor de palha a amarelo médio. Colarinho branco, não muito persistente. Límpida.

\textbf{Aroma}: Aroma de malte de baixo a ausente, pode ser percebido como remetendo à cereais, como milho, adocicado. Aroma de lúpulo de caráter picante ou floral é opcional. Apesar de ser desejável um caráter limpo de fermentação, um leve perfil de levedura não é uma falha.

\textbf{Sabor}: Relativamente neutra no palato, com um final fresco e seco. Sabor de cereais ou milho de baixo a moderadamente baixo, que pode ser percebido como dulçor, devido ao baixo amargor. Sabor de lúpulo de baixo a ausente, podendo ter perfil floral, picante ou herbal, embora seja raramente forte o suficiente para ser detectado. Amargor de lúpulo de baixo a médio-baixo. Equilíbrio pode variar de ligeiramente maltado a ligeiramente amargo, mas é comumente equilibrada. A alta carbonatação pode realçar a sensação de frescor e o final seco. Caráter limpo de fermentação lager.

\textbf{Sensação na Boca}: Corpo de baixo a médio-baixo. Carbonatação muito alta, gera sensação de carbonatação picante na língua.

\textbf{Comentários}: Cerveja que os consumidores de cerveja não artesanal esperam receber no momento em que pedem apenas por uma cerveja nos Estados Unidos. Pode ser comercializada como Pilsner fora da Europa, mas não deve ser confundida com os exemplos tradicionais. Sabores fortes significam falha na cerveja. Com pouco sabor de malte ou lúpulo, o caráter de levedura muitas vezes é o que diferencia entre as diversas marcas.

\textbf{História}: Evolução da Pre-Prohibition Lager (ver Categoria 27) nos Estados Unidos, após a Lei Seca e a Segunda Guerra Mundial. As cervejarias sobreviventes se consolidaram, expandiram a distribuição e promoveram intensamente um estilo de cerveja que satisfazia grande parte da população. Se tornou o estilo dominante por muitas décadas e, por isto, surgiram muitos rivais internacionais, que desenvolveram produtos semelhantes para o mercado de massa, utilizando campanhas publicitárias agressivas.

\textbf{Ingredientes}: Malte de cevada de duas ou seis fileiras, com até 40\% de adjuntos (arroz ou milho). Levedura Lager. Pouco uso de lúpulos.

\textbf{Comparação de Estilos}: Uma versão mais forte, com mais sabor e corpo que uma American Light Lager. Menos amargor e sabor que uma International Pale Lager. Com muito menos sabor, lúpulo e amargor que as tradicionais Pilsners européias.

\begin{tabular}{@{}p{35mm}p{35mm}@{}}
  \textbf{Estatísticas}: & OG: 1,040 - 1,050 \\
  IBU: 8 - 18  & FG: 1,004 - 1,010 \\
  SRM: 2 - 3,5  & ABV: 4,2\% - 5,3\%
\end{tabular}

\textbf{Exemplos Comerciais}: Budweiser, Coors Original, Grain Belt Premium Lager, Miller High Life, Old Style, Pabst Blue Ribbon, Special Export.

\textbf{Última Revisão}: American Lager (2015)

\textbf{Atributos de Estilo}: balanced, bottom-fermented, lagered, north-america, pale-color, pale-lager-family, standard-strength, traditional-style
