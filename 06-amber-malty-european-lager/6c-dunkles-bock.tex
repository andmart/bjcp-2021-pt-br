\phantomsection
\subsection*{6C. Dunkles Bock}
\addcontentsline{toc}{subsection}{6C. Dunkles Bock}

\textbf{Impressão Geral}: Uma cerveja lager alemã álcoólica, escura e maltada, que enfatiza as qualidades ricas do malte e um tanto tostado dos maltes continentais, sem ser doce no final.

\textbf{Aparência}: Cor de cobre claro a marrom, muitas vezes com atraentes reflexos granada. Boa limpidez, apesar da cor escura. Espuma de alta formação, cremosa, persistente e quase branca.

\textbf{Aroma}: Intensidade de malte de médio a médio-alto com rico aroma que remete a panificação, frequentemente com quantidades moderadas de ricos produtos de Maillard ou notas tostadas. Praticamente sem aroma de lúpulo. Um pouco de álcool pode ser perceptível. Caráter limpo de lager, embora seja permitido um leve caráter de frutas escuras.

\textbf{Sabor}: Maltado de médio a médio-alto com caráter rico e complexo, com quantidades moderadas de produtos de Maillard ricos e tostados. Algumas notas de caramelo escuro podem estar presentes. O amargor do lúpulo é geralmente apenas alto o suficiente para sustentar os sabores do malte, permitindo que o um pouco do adociado do malte permaneça até o final. Bem atenuada, não enjoativa. Perfil de fermentação limpo, entretanto o malte pode fornecer um leve caráter de frutas escuras. Sem sabor de lúpulo. Sem caráter torrado, queimado ou de biscoito seco.

\textbf{Sensação na Boca}: Corpo de médio a médio-alto. Carbonatação de moderadamente baixa a moderada. Algum aquecimento alcoólico pode ser encontrado, mas nunca deve ser quente. Suave, sem aspereza ou adstringência.

\textbf{Comentários}: A mosturação por decocção desempenha um papel importante no desenvolvimento dos sabores, pois incrementam os sabores de malte que remetem a caramelo e reações de Maillard.

\textbf{História}: Originada na cidade de Einbeck, no norte da Alemanha, que era um centro cervejeiro e exportador popular nos dias da Liga Hanseática (séculos XIV a XVII). Recriada em Munique a partir do século XVII. A palavra “Bock” é traduzida como “bode” em alemão e é por isso que o referido animal é frequentemente usado em logotipos e anúncios.

\textbf{Ingredientes}: Maltes Munich e Vienna, raramente é usado um pouco de maltes torrados escuros para ajuste de cor, sem uso de quaisquer adjuntos não maltados. Variedades de lúpulo da Europa continental são usadas. Levedura lager alemã limpa.

\textbf{Comparação de Estilos}: Mais escura, com um sabor maltado mais rico e menos amargor aparente do que uma Helles Bock. Menos álcool e riqueza de malte do que uma Doppelbock. Sabores de malte mais fortes e teor alcoólico mais alto do que um Märzen. Mais rica, menos atenuada e menos lupulada do que uma Czech Amber Lager.

\begin{tabular}{@{}p{35mm}p{35mm}@{}}
  \textbf{Estatísticas}: & OG: 1,064 - 1,072 \\
  IBU: 20 - 27 & FG: 1,013 - 1,019 \\
  SRM: 14 - 22 & ABV: 6,3\% - 7,2\%
\end{tabular}

\textbf{Exemplos Comerciais}: Aass Bock, Einbecker Ur-Bock Dunkel, Kneitinger Bock, Lindeboom Bock, Schell’s Bock, Penn Brewery St. Nikolaus Bock.

\textbf{Última revisão}: Dunkles Bock (2015)

\textbf{Atributos de Estilo}: amber-color, bock-family, bottom-fermented, central-europe, high-strength, lagered, malty, traditional-style

