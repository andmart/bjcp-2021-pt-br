\phantomsection
\subsection*{2C. International Dark Lager}
\addcontentsline{toc}{subsection}{2C. International Dark Lager}
\textbf{Impressão Geral}: Uma versão mais escura, mais rica e um pouco mais doce da International Pale Lager, com um pouco mais de corpo e sabor, porém igualmente contida no amargor. O baixo amargor deixa o malte como elemento principal de sabor e os baixos níveis de lúpulo contribuem muito pouco para o equilíbrio. \\
\textbf{Aparência}: Cor de âmbar profundo a marrom muito escuro, com limpidez brilhante e reflexos rubi. Espuma com cor de bege a castanho claro que pode ser de baixa persistência. \\
\textbf{Aroma}: Aroma fraco de malte. Aroma de malte caramelo e/ou torrado médio-baixo é opcional. Aroma de lúpulo floral, herbal e/ou condimentado leve é opcional. Perfil de fermentação limpo. \\
\textbf{Sabor}: Dulçor de malte de baixo a médio. Sabor de malte caramelo e/ou torrado médio-baixo é opcional, possivelmente com notas de café, melaço, açúcar mascavo ou cacau. Sabor de lúpulo floral, herbal e/ou condimentado de baoxa intensidade é opcional. Amargor de baixo a médio. Pode apresentar um caráter frutado muito leve. Final moderadamente bem definido. O equilíbrio é tipicamente maltado. Sabores moderadamente fortes de malte torrado ou queimado são inapropriados. \\
\textbf{Sensação na Boca}: Corpo de baixo a médio-baixo. Suave, com leve cremosidade. Carbonatação de média a alta. \\
\textbf{Comentários}: Uma ampla variedade de lagers internacionais que são mais escuras do que clara/pale e não assertivamente amargas ou torradas. \\
\textbf{História}: Versões mais escuras das International Pale Lagers, geralmente criadas pelas mesmas grandes cervejarias industriais e destinadas a atrair um público amplo. Muitas vezes uma adaptação da lager industrial clara deste padrão com mais cor ou dulçor, ou uma versão mais acessível (e econômica) das lagers escuras mais tradicionais. \\
\textbf{Ingredientes}: Cevada de duas ou seis fileiras, com milho, arroz e/ou açúcares como adjuntos. Uso contido de malte caramelo e malte torrado mais escuro. Versões comerciais podem utilizar corante. \\
\textbf{Comparação de Estilos}: Menos sabor e riqueza do que a Munich Dunkel, Schwarzbier ou outras lagers escuras. Frequentemente utiliza adjuntos, como é típico de outras International Lagers \\
\begin{tabular}{@{}p{35mm}p{35mm}@{}}
  \textbf{Estatísticas}: & OG: 1,044 - 1,056 \\
  IBU: 8 - 20 & FG: 1,008 - 1,012 \\
  SRM: 14 - 30 & ABV: 4,2\% - 6\%
  \end{tabular}\\
\textbf{Exemplos Comerciais}: Baltika \#4 Original, Dixie Blackened Voodoo, Heineken Dark Lager, Saint Pauli Girl Dark, San Miguel Dark, Shiner Bock. \\
\textbf{Última Revisão}: International Dark Lager (2015) \\
\textbf{Atributos de Estilo}: bottom-fermented, dark-color, dark-lager-family, lagered, malty, standard-strength, traditional-style