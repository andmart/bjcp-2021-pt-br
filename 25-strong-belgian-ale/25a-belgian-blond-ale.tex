\phantomsection
\subsection*{25A. Belgian Blond Ale}
\addcontentsline{toc}{subsection}{25A. Belgian Blond Ale}
\textbf{Impressão Geral}: Uma cerveja belga dourada, teor alcoólico moderado, com agradável complexidade cítrica e condimentada oriundas da levedura, que deixa na boca um suave maltado, com final macio e seco.

\textbf{Aparência}: Cor amarelo profundo a um dourado profundo. Geralmente muito límpida. Colarinho quase branco, espesso, denso e cremoso. Boa retenção e com "rendado belga" (\textit{Belgian lace}).

\textbf{Aroma}: Maltado de intensidade de leve a moderada que remete a adocicado como cereais, levemente tostado ou biscoito água e sal. Perfil de levedura de sutil a moderado com ésteres que remete a cítrico e frutados (como limões e laranjas) e fenóis em segundo plano que remetem a picante e condimentados. Notas de lúpulos com perfil terroso ou condimentado são opcionais. Leve perfume alcoólico e indicações de um leve dulçor oriundo do malte, que pode ter um perfil de mel ou açúcar. Sutil, mas ainda assim complexo.

\textbf{Sabor}: Semelhante ao aroma, sendo percebido primeiro um sabor maltado doce de grãos que pode ser de leve a moderado. Pode ser percebido levemente na boca um adocicado que remete a açúcar caramelizado ou doçura de mel. Amargor médio, com o malte levemente superior no equilíbrio. Perfil de levedura de baixo a moderado, com ésteres que remetem a limões e laranjas, bem como leve fenólico picante e condimentado. Pode ter um perfil perfumado leve. Sabor leve de lúpulo, que pode ser terroso ou picante, complementando a levedura. Final de médio seco a seco, macio e suave, com álcool leve e a malte no retrogosto.

\textbf{Sensação na Boca}: Carbonatação de médio alta a alta, que pode dar a sensação borbulhante que enche a boca. Corpo médio. Leve a moderado aquecimento alcoólico, mas suave. Pode ser cremosa.

\textbf{Comentários}: A maioria dos exemplos comerciais estão na faixa de 6,5 a 7\% de ABV. Muitas vezes possui um caráter como de uma lager, com um perfil mais limpo quando comparado com outros estilos belgas. Os belgas que falam a língua flamenga usam o termo \textit{Blond}, ao passo que os que falam francês escrevem \textit{Blonde}. Muitas cervejarias belgas, monásticas ou artesanais, são chamadas de \textit{Blond}, mas estas não seriam representativas do estilo.

\textbf{História}: O desenvolvimento do estilo é relativamente recente para atender ao apelo de bebedores de Pilsen europeias, tornando-se mais popular em decorrência de sua forte comercialização, bem como a sua ampla distribuição. Apesar das alegações de que o estilo tenha surgido em 1200, ele somente foi criado após II Guerra Mundial e foi popularizado pela Leffe.

\textbf{Ingredientes}: Malte pilsen belga, maltes aromáticos, açúcar e outros adjuntos, cepas de leveduras belgas tipo Abbey, lúpulos continentais. Especiarias não são tradicionalmente utilizados, mas quando presentes, devem possuir apenas um caráter de fundo.

\textbf{Comparação de Estilos}: Semelhante a força alcoólica e o equilíbrio de Belgian Dubbel, como de coloração dourada e sem os sabores de maltes escuros. Semelhante ao perfil de uma Belgian Strong Golden Ale ou de Belgian Tripel, embora um pouco mais maltada, não tão amarga e mais baixo em relação ao álcool.

\begin{tabular}{@{}p{35mm}p{35mm}@{}}
  \textbf{Estatísticas}: & OG: 1,062 - 1,075 \\
  IBU: 15 - 30  & FG: 1,008 - 1,018  \\
  SRM: 4 - 6  & ABV: 6\% - 7,5\%
\end{tabular}

\textbf{Exemplos Comerciais}: Affligem Blond, Corsendonk Blond, Grimbergen Blond, La Trappe Blond, Leffe Blond, Val-Dieu Blond.

\textbf{Versão Anterior}: Belgian Blond Ale (2015)

\textbf{Atributos de Estilo}: balanced, high-strength, pale-color, top-fermented, traditional-style, western-europe