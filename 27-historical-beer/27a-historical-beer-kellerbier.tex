\phantomsection
\subsection*{27A. Historical Beer: Kellerbier}
\addcontentsline{toc}{subsection}{27A. Historical Beer: Kellerbier}
\textbf{Impressão Geral}: Uma lager alemã não filtrada, não pasteurizada e totalmente atenuada, tradicionalmente servida dos tanques de maturação. Pode ser um pouco mais rica, robusta e rústica do que os estilos base. Uma cerveja fresca sem defeitos de fermentação associados à cerveja jovem e verde (inacabada).

\textbf{Aroma}: Reflete o estilo base. Pode ter um caráter de pão e fermento adicionado pela levedura. Aroma limpo. Versões claras podem ter um caráter de lúpulo mais robusto. Versões escuras podem ter um perfil de malte mais rico.

\textbf{Aparência}: Reflete o estilo base. Pode ser turva ou levemente turva, nunca com turbidez excessiva. Provavelmente um pouco mais escura na aparência do que o estilo base.

\textbf{Sabor}: Reflete o estilo base. Pode ter um caráter de pão e fermento adicionado pela levedura. Versões claras podem ter um caráter de lúpulo mais robusto. As versões escuras podem ter um perfil de malte mais rico, mas nunca devem ter o caráter torrado. Pode ser um pouco mais amarga que o estilo base e um pouco mais pesada no final. Totalmente fermentada com um perfil de fermentação limpo; não deve apresentar defeitos que remetam a ovo, amanteigado, maçã ou similares.

\textbf{Sensação na Boca}: Reflete o estilo base. Pode ter um pouco mais de corpo e uma textura mais cremosa do que o estilo base. Carbonatação típica do estilo base, mas pode ser menor.

\textbf{Comentários}: Mais do que um estilo de cerveja, um estilo tradicional de servir, mas essas cervejas têm diferenças sensoriais das cervejas básicas. Devem ser julgadas como cervejas especiais; considere a variedade de \textit{kellerbiers} baseado da Helles a Märzen e a Dunkel como um espectro contínuo, portanto, permita que o cervejeiro escolha a mais próxima, sem ser muito exigente quanto à adesão estrita ao estilo base. O nome significa literalmente cerveja de adega e é de manuseio suave e natural de uma \textit{lagerbier} alemã de sabor fresco para serviço sazonal no local. Como as British Bitters é melhor apreciada localmente, pois os exemplares engarrafados podem não ter o frescor característico.

\textbf{História}: Originalmente se referia à cerveja lager maturada nas cavernas ou adegas sob a cervejaria e depois servida a partir delas. Adaptada pela primeira vez à âmbar lager da Francônia, depois aos estilos locais de Munique. Mais recentemente, usada internacionalmente para criar variantes especiais da Pils. Por tradição, um estilo de servir para uma peculiaridade popular de verão na Baviera, mas agora amplamente adaptada como um termo de propaganda para lagers não filtradas.

\textbf{Ingredientes}: Igual aos estilos básicos. Tradicionalmente carbonatada naturalmente. O \textit{dry-hopping} não é um método tradicional de fabricação de cerveja alemã, mas alguns exemplos claros modernos usam essa técnica - que é permitida nesse estilo, desde que seja equilibrada. Tradicionalmente maturada a frio e não filtrada, essas cervejas nunca foram feitas para serem embaladas para venda externa.

\textbf{Comparação de Estilos}: Mais rica ou mais robusta que o estilo base, possivelmente com um pouco mais de corpo e sensação na boca. Pode ser um pouco mais turva que a cerveja base.

\textbf{Instruções para Inscrição}: O participante deve especificar o estilo base: German Pils, Munich Helles, Märzen ou Munich Dunkel.

\textbf{Estatisticas}: As mesmas do estilo base.

\textbf{Exemplos Comerciais}: Märzen - Faust Kräusen Naturtrüb, Mahrs Bräu aU Ungespundet Kellerbier; Dunkel - Engel Kellerbier Dunkel, Paulaner Ur-Dunkel Naturtrüb; Helles - ABK Kellerbier Naturtrüb, Löwenbräu 1747 Original; Pils - Giesinger Feines Pilschen, Ketterer Zwickel-Pils

\textbf{Última Revisão}: Kellerbier: Pale Kellerbier (2015)

