\phantomsection
\subsection*{26A. Belgian Single}
\addcontentsline{toc}{subsection}{26A. Belgian Single}
\textbf{Impressão Geral}: Uma cerveja amarelo dourada, amarga, lupulada, muito seca e altamente carbonatada. O frutado e o condimentado são agressivos, oriundos da levedura belga, e o amargor elevado repercutem em seu equilíbrio, de forma suave, deixando na boca um perfil de malte adocicado de cereais e floral e condimentado de lúpulo.

\textbf{Aparência}: Cor de amarelo claro a dourado médio. Geralmente tem boa limpidez, com um colarinho lembrando merengue de coloração branca, de tamanho moderado, persistente, com característico rendado belga.

\textbf{Aroma}: Característica de levedura belga de intensidade média-baixa a média-alta, exibindo um caráter frutado e condimentado em conjunto com um perfil de intensidade média-baixo a médio de lúpulo condimentado ou floral, ocasionalmente realçadas por toques leves de especiarias herbais ou cítricas. Em segundo plano, o malte é de baixo a médio-baixo, com notas de pão, biscoito, cereais ou mel suave. O frutado pode varias bastante (maçã, pera, toranja, limão, laranja, pêssego, damasco). Fenólicos são tipicamente como pimenta preta ou cravo. Tutti-frutti é inapropriado.

\textbf{Sabor}: No início, o sabor maltado é leve com caráter de biscoito de mel, pão ou biscoito tipo água e sal. Possui sabor de grãos suave e um final lupulado, amargo e seco e bem definido. Na boca, sabor moderado de lúpulo com característica condimentada ou floral. Ésteres moderados semelhantes aos encontrados no aroma. Fenólicos condimentados de leves a moderados como os encontrados no aroma. Amargor de médio a alto, acentuado pela secura. O caráter de levedura e o de lúpulo permanecem no retrogosto.

\textbf{Sensação na Boca}: Corpo de médio-baixo a médio. Macia. Carbonatação de média-alta a alta, pode causar um pouco de formigamento. Não deve ter aquecimento alcoólico perceptível.

\textbf{Comentários}: Raramente é fabricada, quando é, geralmente não é rotulada e nem vendida fora dos mosteiros. Também pode ser chamada de cerveja dos monges, cerveja dos Irmãos ou simplesmente Blond (não usamos este termo para não causar confusão com um outro estilo, o Belgian Blond Ale). Altamente atenuada, geralmente com 85\% ou mais.

\textbf{História}: Embora as cervejarias de monastérios tenham a tradição de produzir cerveja com menor teor alcoólico como alimento diário de um monge (a Westmalle começou a produzir a sua em 1922), a cerveja clara e amarga que o estilo descreve é uma invenção relativamente moderna que reflete os gostos atuais. A Westvleteren produziu a deles pela primeira vez em 1999, mas substituiu produtos mais antigos de densidade mais baixa.

\textbf{Ingredientes}: Malte Pilsen. Levedura belga. Lúpulos continentais.

\textbf{Comparação de Estilos}: Como uma interpretação belga de alta fermentação de uma German Pils - clara, lupulada e bem atenuada, mas com um caráter forte da levedura belga. Tem menos dulçor, maior atenuação, menos caráter de malte e é mais centrada no lúpulo do que uma Belgian Pale Ale. Mais parecida com a uma versão altamente lupulada e mais leve de uma Belgian Tripel (com o seu amargor e a secura) do que uma versão mais menor de uma Belgian Blond Ale.

\begin{tabular}{@{}p{35mm}p{35mm}@{}}
  \textbf{Estatísticas}: & OG: 1,044 - 1,054 \\
  IBU: 25 - 45  & FG: 1,004 - 1,010  \\
  SRM: 3 - 5  & ABV: 4,8\% - 6\%
\end{tabular}

\textbf{Exemplos Comerciais}: Chimay Gold, La Trappe Puur, Russian River Redemption, St. Bernardus Extra 4, Westmalle Extra, Westvleteren Blond.

\textbf{Última Revisão}: Trappist Single (2015)

\textbf{Atributos de Estilo}: bitter, craft-style, hoppy, pale-color, standard-strength, top-fermented, western-europe
