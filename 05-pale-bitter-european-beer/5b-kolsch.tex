\phantomsection
\subsection*{5B. Kölsch}
\addcontentsline{toc}{subsection}{5B. Kölsch}
\textbf{Impressão Geral}: Uma cerveja sutil, clara, brilhante e límpida com um delicado equilíbrio entre o caráter de malte, fruta e lúpulo. Amargor moderado e bem atenuada, mas com o final suave. O frescor faz muita diferença a esta cerveja, pois o perfil delicado pode desaparecer rapidamente com o tempo. \\
\textbf{Aparência}: De amarelo médio a dourado claro. Limpidez brilhante. Possui uma delicada espuma branca que pode não persistir. \\
\textbf{Aroma}: Aroma de cereal maltado adocicado de muito-baixo a baixo. Aroma sutil de frutas (maçã, pera ou às vezes cereja) é opcional, mas bem-vindo. Aroma de lúpulo floral, condimentado e/ou herbal baixo é opcional. A intensidade dos aromas é sutil, mas geralmente equilibrada, limpa, fresca e agradável. \\
\textbf{Sabor}: Apresenta um delicado equilíbrio entre o sabor maltado, frutado, amargor e lúpulo, com final limpo e bem atenuado. Característa de malte lembrando cereais de médio a médio-baixo, pode apresentar notas muito leves de pão ou mel. O frutado pode apresentar um dulçor quase imperceptível. Amargor de médio-baixo a médio. Sabor de lúpulo floral, condimentado e/ou herbal de baixo a moderadamente alto; a maioria é de médio-baixo a médio. No início pode dar a impressão de um caráter de malte neutro a um leve dulçor de malte. Gosto suave e arredondado. O final é macio, seco e levemente crisp (bem definido), sem ser forte ou penetrante. Sem dulçor residual perceptível. Embora o equilíbrio entre os componentes de sabor possa variar, nenhum é forte. \\
\textbf{Sensação na Boca}: Corpo de médio-baixo a médio; a maioria é médio-baixo. Carbonatação de média a média-alta. Suave e macia, mas bem atenuada sem ser pesada. Sem aspereza. \\
\textbf{Comentários}: Uma cerveja tradicional, de alta fermentação e maturada a frio, da cidade de Colônia, Alemanha (Köln). As cervejarias de Colônia se diferenciam entre si pelo equilíbrio, então ao julgar permita uma gama de variações dentro do estilo. As versões mais secas podem parecer mais lupuladas ou mais amargas do que os níveis de IBU podem sugerir. O delicado perfil de sabor não envelhece bem, portanto deve-se ficar atento a defeitos oriundos de oxidação. Em Colônia é servida em um copo alto e estreito de 200 ml chamado Stange. \\
\textbf{História}: Colônia tem uma tradição cervejeira desde a Idade Média, mas a cerveja agora conhecida como Kölsch foi desenvolvida no final dos anos 1800s como uma alternativa as lagers claras. A baixa fermentação foi na realidade proibida em Colônia. Kölsch é uma denominação protegida pela \textit{Kölsch Konvention} (1986) e é restrita a cervejarias dentro e ao redor de Colônia. A \textit{Konvention} simplesmente define a cerveja como uma “Vollbier” leve, altamente atenuada, com lúpulo acentuado, límpida e de alta fermentação. \\
\textbf{Ingredientes}: Lúpulos tradicionais alemães. Malte alemão Pils, Pale e/ou Vienna. Levedura ale alemã limpa e atenuante. Ocasional uso de malte de trigo em pequenas quantidades. A prática comercial atual é fermentar em torno de 15°C, acondicionar a frio próximo ao congelamento por até um mês e servir fresca. \\
\textbf{Comparação de Estilos}: Pode ser confundida com uma Cream Ale ou com uma German Pils um tanto sutil. \\
\begin{tabular}{@{}p{35mm}p{35mm}@{}}
  \textbf{Estatísticas}: & OG: 1,044 - 1,050 \\
  IBU: 18 - 30  & FG: 1,007 - 1,011  \\
  SRM: 3,5 - 5  & ABV: 4,4\% - 5,2\%
\end{tabular}\\
\textbf{Exemplos Comerciais}: Früh Kölsch, Gaffel Kölsch, Mühlen Kölsch, Päffgen Kolsch, Reissdorf Kölsch, Sion Kölsch, Sünner Kölsch. \\
\textbf{Última revisão}: Kölsch (2015) \\
\textbf{Atributos de Estilo}: balanced, central-europe, lagered, pale-ale-family, pale-color, standard-strength, top-fermented, traditional-style
