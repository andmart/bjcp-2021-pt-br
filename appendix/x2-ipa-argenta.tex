\phantomsection
\subsection*{X2. IPA Argenta}
\addcontentsline{toc}{subsection}{X2. IPA Argenta}

\textit{Estilo sugerido para inscrição: Categoria 21 (IPA)}\\
\textbf{Impressão Geral}: Uma ale argentina decididamente lupulada e amarga, refrescante e moderadamente forte. O ponto principal do estilo está em ser fácil de beber, sem aspereza e com um bom equilíbrio.

\textbf{Aroma}: Aroma de lúpulo intenso com caráter cítrico e floral, oriundo dos lúpulos argentinos. Algum dulçor maltado e caramelo pode ser encontrado como nota de fundo, mas deve estar em menor intensidade que em exemplares ingleses. Um caráter frutado proveniente de ésteres e leve fenóis de fermentação de trigo podem também ser detectadas em algumas versões, embora um caráter de fermentação neutra seja mais comum. Um caráter alcoólico pode ser percebido em versões mais fortes. Sem DMS. O diacetil é um grande demérito, porque pode encobrir aromas de lúpulo, e nunca deve estar presente.

\textbf{Aparência}: Cor variando de dourada média a cobre avermelhada médio; algumas versões podem ter um tom alaranjado. Deve ser límpida, ainda que versões com \textit{dry-hopping} não filtradas ou com trigo não maltado possam ser um pouco turvas. Boa formação de espuma e persistente.

\textbf{Sabor}: Sabor de lúpulo de médio a alto, devendo refletir o caráter de lúpulo argentino: cítrico, toranja e casca de tangerina devem dominar. Pode ter um caráter floral, como flor de laranjeira, ou herbal e resinoso, embora isso seja menos comum e deve apenas adicionar complexidade. Amargor de lúpulo de médio-alto a muito alto, apesar da base de malte dar suporte ao forte caráter de lúpulo e prover melhor equilíbrio. Sabor de malte deve ser de baixo a médio e é geralmente limpo e doce como malte, ainda que algum caramelo ou sabores de especiarias do trigo, maltado ou não maltado, sejam aceitáveis em níveis baixos. Sem diacetil. Frutado baixo é aceitável, mas não é obrigatório. O amargor pode permanecer no retrogosto, mas não deve ser áspero. Final de meio seco a seco, refrescante. Algum sabor limpo de álcool pode ser notado em versões mais fortes.

\textbf{Sensação na Boca}: Corpo de médio-baixo a médio, sem adstringência derivada de lúpulo, ainda que a carbonatação de moderada a médio-alta, combinada com o trigo, possa dar uma sensação seca, mesmo com presença de dulçor de malte. Algum aquecimento alcoólico suave pode e deve ser sentido em versões mais fortes (mas não em todas). O corpo geralmente é mais baixo do que em uma English IPA e mais seca que uma American IPA

\textbf{História}: Uma versão argentina do estilo histórico inglês, desenvolvida em 2013, nos encontros da associação \textit{Somos Cerveceros}, momento em que suas características marcantes foram definidas. Difere de uma American IPA por ser produzida com trigo e usando lúpulos argentinos, que tem sabor e aroma com características únicas. O estilo busca que as características cítricas do lúpulo argentino estejam em harmonia com o trigo, como em uma Witbier. A pouca quantidade de trigo pode ser semelhante ao grist de uma Kölsch, em que também há um frutado proveniente da fermentação.

\textbf{Ingredientes}: Malte Pale Ale (bem modificado e adequado para mostura simples) com até 15\% de trigo, tanto maltado como não maltado; maltes Caramelo devem ser limitados, dando preferência ao trigo caramelo. Lúpulos argentinos, como Cascade, Mapuche e Nugget são típicos, embora Spalt, Victoria e Bullion também possam ser usados para adicionar complexidade. Levedura americana, que traz um perfil levemente frutado e limpo. Caráter da água varia de baixo a moderado teor de sulfatos.

\begin{tabular}{@{}p{35mm}p{35mm}@{}}
  \textbf{Estatísticas}: & OG: 1,055 – 1,065 \\
  IBU: 35 - 60 & 1,008 – 1,015 \\
  SRM: 6 - 15 & ABV: 5,0 - 5,5\%
\end{tabular}

\textbf{Exemplos Comerciais}: Antares Ipa Argenta, Kerze Ipa Argenta.
