\phantomsection
\subsection*{X1. Dorada Pampeana}
\addcontentsline{toc}{subsection}{X1. Dorada Pampeana}

\textit{Estilo sugerido para inscrição: Categoria 18 (Pale American Beer)}

\textbf{Impressão Geral}: Fácil de beber, maltada.

\textbf{Aroma}: Aroma adocicado de malte de leve a moderado. Aroma frutado de baixo a moderado é aceitável. Pode ter aroma de lúpulo de baixo a médio. Sem diacetil.

\textbf{Aparência}: Cor de amarelo claro a dourado profundo. De limpa a brilhante. Colarinho de baixo a médio, com boa retenção.

\textbf{Sabor}: ulçor maltado inicial suave. Sabores de caramelo são tipicamente ausentes. Sabor de lúpulo de leve a moderado (geralmente Cascade), mas não deve ser agressivo. Amargor de lúpulo de baixo a moderado, com o equilíbrio normalmente tendendo ao maltado. Final de meio seco a um pouco doce. Sem diacetil.

\textbf{Sensação na Boca}: Corpo de médio-baixo a médio. Carbonatação de média à alta. Macia, sem amargor áspero ou adstringência.

\textbf{Comentários}: É difícil alcançar o equilíbrio.

\textbf{História}: No início, os cervejeiros caseiros argentinos estavam muito limitados: não existiam extratos, só podiam usar malte Pils, lúpulos Cascade e levedura seca, geralmente Nottingham, Windsor ou Safale. Com esses ingredientes, os cervejeiros argentinos desenvolveram uma versão específica de Blond Ale, chamando-a de Dorada Pampeana.

\textbf{Ingredientes}: Geralmente só malte Pale ou Pils, embora possa incluir proporções baixas de maltes caramelizados. Comumente se utiliza lúpulo Cascade. Levedura americana, inglesa ou \textit{Kölsch} limpa, geralmente acondicionada a frio.

\begin{tabular}{@{}p{35mm}p{35mm}@{}}
  \textbf{Estatísticas} & OG: 1,042 – 1,054 \\
  IBU: 15 - 22 & FG: 1,009 – 1,013 \\
  SRM: 3 - 5 & ABV: 4,3 - 5,5\%
\end{tabular}