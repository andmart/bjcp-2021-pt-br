\phantomsection
\subsection*{7A. Vienna Lager}
\addcontentsline{toc}{subsection}{7A. Vienna Lager}
\textbf{Impressão Geral}: Uma lager âmbar da Europa continental de teor alcoólico moderado com maltado macio e suave e um amargor moderado para dar equilibrio, ainda assim com um final relativamente seco. O sabor de malte é limpo, rico como pão e um tanto tostado, com uma elegância oriunda não dos maltes especiais ou adjuntos, mas sim da qualidade dos maltes base e do processo. \\
\textbf{Aparência}: Cor de âmbar avermelhado claro a cobre. Limpidez brilhante. Colarinho espesso, quase branco e persistente. \\
\textbf{Aroma}: Aroma de malte moderadamente intenso com notas tostadas e de maltado rico. Aroma de lúpulo floral, condimentado de baixo a nenhum. Caráter lager limpo. Aromas significativo de caramelo, similar a biscoito e/ou torrado são inapropriados. \\
\textbf{Sabor}: Complexidade de malte suave, elegante em primeiro plano com amargor de lúpulo firme o suficiente para prover um final equilibrado. O sabor de malte tende em direção ao caráter rico e tostado, sem sabores significativos de caramelo, similar a biscoito ou torrado. Razoavelmente seca, com final macio, com tanto a riqueza de malte quanto o amargor de lúpulo presentes no retrogosto. Sabor de lúpulo floral, condimentado e/ou herbal pode ser de baixo a nenhum. Perfil limpo de fermentação. \\
\textbf{Sensação na Boca}: Corpo de médio-baixo a médio com uma delicada cremosidade. Carbonatação moderada. Macia. \\
\textbf{Comentários}: Uma cerveja do dia a dia de teor alcoólico padrão, não uma cerveja brassada para festivais. Muitos exemplares tradicionais se tornaram mais doces e carregadas de adjuntos, parecendo mais como uma International Amber ou Dark Lagers. \\
\textbf{História}: Desenvolvido por Anton Dreher em Vienna em 1841, se tornando popular no século XIX. O estilo foi levado para o México por Santiago Graf e outros cervejeiros imigrantes austríacos no final do século XIX. Parece ter sido adotada como uma estilo artesanal moderno em outros países. \\
\textbf{Ingredientes}: Tradicionalmente malte Vienna da melhor qualidade, mas também pode ser usado Malte Pils ou Munich. Lúpulos continentais europeus tradicionais. Levedura lager alemã limpa. Pode usar pequenas quantidades de maltes especiais para cor e dulçor. \\
\textbf{Comparação de Estilo}: Sabor de malte similar a uma Märzen, mas com intensidade e corpo mais leve, com um toque a mais de amargor e secura no equilíbrio. Menos alcoólica que uma Märzen ou Festbier. Menos rica, maltada e lupulada que uma Czech Amber Lager. \\
\begin{tabular}{@{}p{35mm}p{35mm}@{}}
  \textbf{Estatísticas}: & OG: 1,048 - 1,055 \\
  IBU: 18 - 30 & FG: 1,010 - 1,014 \\
  SRM: 9 - 15 & ABV: 4,7\% - 5,5\%
\end{tabular}\\
\textbf{Exemplos Comerciais}: Chuckanut Vienna Lager, Devils Backbone Vienna Lager, Figueroa Mountain Red Lager, Heavy Seas Cutlass, Ottakringer Wiener Original, Schell’s Firebrick, Theresianer Vienna. \\
\textbf{Última Revisão}: Vienna Lager (2015) \\
\textbf{Atributos de Estilo}: amber-color, amber-lager-family, balanced, bottom-fermented, central-europe, lagered, standard-strength, traditional-style
