\phantomsection
\subsection*{29A. Fruit Beer}
\addcontentsline{toc}{subsection}{29A. Fruit Beer}

\textbf{Impressão Geral}: Uma combinação harmoniosa entre frutas e cerveja, mas ainda assim reconhecível como cerveja. O caráter de frutas deve estar em equilíbrio com a cerveja e ser evidente, mas não proeminente a ponto de sugerir um produto artificial.

\textbf{Aparência}: Varia de acordo com o estilo base e ingredientes especiais utilizados. Cervejas de cor mais clara, incluindo o colarinho, podem apresentar cores vindas da adição. A cor da fruta na cerveja é geralmente mais clara do que a cor da polpa da própria fruta e pode ter tons ligeiramente diferentes. Limpidez variável, ainda que a turbidez seja geralmente não desejável. Alguns ingredientes podem impactar na retenção da espuma.

\textbf{Aroma}: Varia de acordo com o estilo base. O caráter de fruta deve ser notável no aroma; no entanto, algumas frutas (p. ex., framboesas e cerejas) têm aromas mais intensos e distintos do que outras (p. ex., mirtilos e morangos) – o que permite uma gama de caráter e intensidade de frutas, de sutil a agressivo. Aroma de lúpulo pode ser mais baixo do que o esperado para o estilo base, para mostrar melhor o caráter da fruta. A fruta deve adicionar complexidade, mas não deve estar proeminente a ponto de desequilibrar o produto final.

\textbf{Sabor}: Varia de acordo com o estilo base. Como no aroma, um distinto sabor de fruta deve ser notável e pode variar em intensidade, de sutil a agressivo; porém, o caráter da fruta não deve ser tão artificial ou inapropriadamente dominante a ponto de sugerir um "suco de frutas". Amargor, sabores de lúpulo e malte, teor alcoólico e subprodutos de fermentação, como ésteres, devem ser apropriados para o estilo base, porém harmoniosos e equilibrados com os distintos sabores de frutas presentes. A fruta geralmente adiciona sabor, mas não dulçor; normalmente, o açúcar encontrado em frutas é totalmente fermentado e contribui para sabores mais leves e um final mais seco. Entretanto, dulçor residual não é necessariamente uma caraterística negativa, a não ser que traga um caráter cru, de não fermentado. Algumas frutas podem adicionar acidez, amargor e/ou taninos, que devem estar em equilíbrio no perfil final de sabor.

\textbf{Sensação na boca}: Varia de acordo com o estilo base. A fruta muitas vezes diminui o corpo e faz a cerveja parecer mais leve no palato. Algumas frutas menores e mais escuras podem adicionar uma profundidade tânica, mas essa adstringência não deve sobrepujar a cerveja base.

\textbf{Comentários}: A descrição da cerveja é crítica para avaliação. Os juízes devem pensar mais no conceito declarado do que tentar detectar individualmente cada ingrediente utilizado. Equilíbrio, facilidade de beber e execução do tema (do estilo proposto) são os fatores mais importantes. A fruta deve complementar o estilo base, não sobrecarregá-lo. Os atributos do estilo base serão diferentes após a adição da fruta, ou seja, não espere que a cerveja tenha o sabor idêntico ao do estilo base inalterado. Fruit Beers com base de Estilos Clássicos devem ser inscritas neste estilo, exceto Lambic - existe um estilo especial para Fruit Lambic (23F). Cervejas com frutas, ácidas ou de fermentação mista, sem base de Estilo Clássico devem ser inscritas em 28C Wild Specialty Beer. Versões com fruta de Estilos Clássicos ácidos (p. ex., Flanders Red Ale, Oud Bruin, Gose, Berliner Weisse) devem ser inscritas em 29A Fruit Beer. Versões com fruta de Estilos Clássicos em que especiarias são uma parte inerente da sua definição (p. ex., Witbier, Gose) não contam como Spice Beer para fins de inscrição.

\textbf{Instruções para inscrição}: O participante deve especificar o(s) tipo(s) de fruta(s) utilizado(s). O participante deve fornecer uma descrição da cerveja, identificando o estilo base e os ingredientes, estatísticas e/ou caráter desejado para a cerveja. Uma descrição geral da natureza especial da cerveja pode abranger todos os itens necessários.

\textbf{Estatísticas}: OG, FG, IBU, SRM e ABV vão variar de acordo com a cerveja base, mas a fruta poderá impactar na coloração.

\textbf{Exemplos comerciais}: 21st Amendment Hell or High Watermelon, Anderson Valley Blood Orange Gose, Avery Liliko'i Kepolo, Ballast Point Grapefruit Sculpin, Bell's Cherry Stout, Founders Rübæus.

\textbf{Última Revisão}: Fruit Beer (2015)

\textbf{Atributos do Estilo}: fruit, specialty-beer
