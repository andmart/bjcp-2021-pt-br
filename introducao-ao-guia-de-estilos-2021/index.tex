\clearpage
\phantomsection
\divisorLine
\section*{Introdução ao Guia de Estilos 2021}
\addcontentsline{toc}{section}{Introdução ao Guia de Estilos 2021}
\textit{O Guia de estilos do BJCP 2021 é uma pequena revisão da edição de 2015 e uma grande atualização da edição de 2008. Os objetivos da edição 2015 eram a de melhor descrever os estilos de cerveja ao redor do mundo como encontrados em seus mercados locais, se manter em dia com as tendências emergentes no mercado da cerveja artesanal, descrever cervejas históricas que passam a ter apreciadores, descrever de forma melhor as características sensoriais de ingredientes modernos, se valer de novas pesquisas e referências e ajudar organizadores de competições a administrar a complexidade de seus eventos de forma melhor. Estes objetivos não mudaram na edição 2021.}

\textit{No guia de 2015, diversos estilos foram adicionados, alguns estilos foram divididos em múltiplas categorias e alguns simplesmente renomeados. Os estilos são organizados e agrupados tendo por base estilos com características de julgamento similares ao invés de uma origem ou nome de família em comum. Não faça presunções que a mesma característica primária (por exemplo cor, teor alcoólico, equilíbrio, sabor predominante, país de origem) foi utilizado para determinar o agrupamento de cada categoria – o motivo é mais detalhado.}

\textit{Se você possui familiaridade com o guia de estilos de 2015 mudamos alguns nomes e alguns estilos foram movidos de categorias provisórias, históricas ou estilos locais para a parte principal do guia de estilos. De forma intencional nós trabalhamos para manter as mudanças de estilos, numerações e adições ao mínimo.}

\begin{multicols*}{2}
\import{./}{estilos-e-categorias.tex}
\import{./}{nomeclatura-dos-estilos-e-categorias.tex}
\import{./}{utilizando-o-guia-de-estilos.tex}
\import{./}{formato-da-descricao-de-estilos.tex}
\import{./}{linguagem-de-descricao-dos-estilos.tex}
\end{multicols*}