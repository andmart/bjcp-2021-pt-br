\subsection*{Estilos e Categorias}
\addcontentsline{toc}{subsection}{Estilos e Categorias}
O guia de estilos do BJCP utiliza alguns termos específicos com significado particular: Categoria, Subcategoria e Estilo. Ao pensar em estilos de cerveja, hidromel e sidra, a subcategoria é a indicação mais importante - subcategoria se refere essencialmente à mesma coisa que estilo e identifica as características principais de um tipo de cerveja, hidromel ou sidra. Cada estilo tem uma descrição bem definida que é a ferramenta básica utilizada durante o julgamento.

Quando as descrições de um estilo de cerveja não convencional se referirem à um estilo clássico, nós estamos nos referindo a um estilo (ou subcategoria) nomeado anteriormente à sessão de estilos de especialidade do guia de estilos.

As categorias mais amplas são agrupamentos arbitrários de estilos de cerveja, hidromel e sidra, normalmente com características sensoriais similares. Subcategorias não são necessariamente relacionadas umas com as outras dentro de uma mesma categoria. A proposta por trás da estrutura das categorias é agrupar estilos de cerveja, hidromel e sidra de forma a facilitar o julgamento durante as competições. Não tente deduzir significados adicionais dos agrupamentos - nenhuma associação histórica ou geográfica é implicada ou intencionada.

Competições podem criar suas próprias categorias de premiação que são distintas das categorias de estilos contidas neste guia de estilo. Não existe nenhum requerimento para que as competições utilizem as categorias de estilos como categorias de premiação! Estilos individuais podem ser agrupados de qualquer maneira para criar as categorias de premiação dentro de uma competição; como por exemplo, para equilibrar e melhor distribuir o número de amostras dentro de cada categoria de premiação.

Enquanto categorias de estilos são a forma mais vantajosa para efeitos de julgamento, pois agrupam as cervejas com características sensoriais semelhantes, nós reconhecemos que esta talvez não seja a melhor forma para aprender sobre os estilos de cerveja. Para estudo, os estilos podem ser agrupados em famílias de estilos para que possam ser comparados e contrastados. As cervejas também podem ser agrupadas por país de origem para melhor entendimento da história da cerveja em um país, ou para aprender sobre o mercado local. Qualquer um destes agrupamentos é perfeitamente aceitável, os estilos só foram agrupados desta forma para facilitar o julgamento de competições. Veja o apêndice A para formas alternativas de agrupamentos.