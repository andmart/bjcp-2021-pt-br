\phantomsection
\subsection*{28A. Brett Beer}
\addcontentsline{toc}{subsection}{28A. Brett Beer}

\textbf{Impressão Geral}: Uma cerveja frequentemente mais seca e frutada do que o estilo base sugere. Notas frutadas ou \textit{funky} variam de baixas a altas, dependendo da idade da cerveja e cepas de \textit{brettanomyces} utilizadas. Pode apresentar leve acidez que não seja láctica.

\textbf{Aparência}: Varia com o estilo base. Limpidez variável e depende do estilo base e ingredientes utilizados. Turbidez baixa não necessariamente é uma falha.

\textbf{Aroma}: Varia com o estilo base. Brett Beers jovens apresentam notas frutadas (por exemplo, frutas tropicais, frutas de caroço ou frutas cítricas), no entanto a intensidade varia de acordo com as cepas de \textit{brettanomyces} utilizadas. Brett Beers mais velhas apresentam leve \textit{funky} conforme passa o tempo (por exemplo, celeiro, palha úmida, leve terroso e notas defumadas), no entanto essa característica não deve dominar.

\textbf{Sabor}: Varia com o estilo base. O caráter de \textit{brettanomyces} varia de mínimo a agressivo. Pode ser bastante frutado (por exemplo, frutas tropicais, bagas, frutas de caroço ou frutas cítricas) ou ter algum defumado, terroso ou notas de celeiro. Não deve ter um \textit{funky} desagradável como Band-aid, fétido, removedor de esmalte, queijo etc. Sempre frutado quando a cerveja é jovem e ganhando mais \textit{funky} com o envelhecimento. Pode não ser láctica. Sabores de malte são frequentemente menos pronunciados do que no estilo base, deixando a cerveja frequentemente seca e com final bem definido devido a alta atenuação da \textit{brettanomyces}.

\textbf{Sensação na Boca}: Varia com o estilo base. Geralmente tem um corpo baixo, mais leve do que o esperado para o estilo base, embora um corpo muito baixo seja uma falha. Carbonatação usualmente de moderada a alta. Retenção de espuma é variável, mas frequentemente menor que a do estilo base.

\textbf{Comentários}: O estilo base descreve a maior parte das características destas cervejas, mas a adição de \textit{brettanomyces} garante um produto mais seco, mais leve e frequentemente mais frutado e com \textit{funky}. Versões jovens são mais vividas e frutadas, enquanto as mais velhas possuem \textit{funk} mais complexo e podem perder mais do caráter do estilo base. O caráter de \textit{brettanomyces} deve sempre combinar com o estilo base; essas cervejas nunca devem ser uma 'bomba de Brett'. Embora a \textit{brettanomyces} possa produzir baixos níveis de ácidos orgânicos, não é um método primário de acidificação de cerveja.

\textbf{História}: Interpretações modernas artesanais ou experimentações inspiradas em Belgian wild ales ou cervejas inglesas com \textit{brettanomyces}. As ditas cervejas 100\% Brett ganharam popularidade após o ano 2000, mas foi quando se imaginava que a Saccharomyces Trois fosse uma cepa de \textit{brettanomyces} (o que não é). Atualmente \textit{brettanomyces} usada em conjunto com uma fermentação com \textit{saccharomyces} é uma prática padrão.

\textbf{Ingredientes}: Virtualmente qualquer estilo de cerveja (exceto os que já utilizam a co-fermentação \textit{saccharomyces} ou \textit{brettanomyces}), depois finalizada com uma ou mais cepas de \textit{brettanomyces}. Uma alternativa é uma fermentação mista com \textit{saccharomyces} e uma ou mais cepas de \textit{brettanomyces}. Sem \textit{lactobacillus}.

\textbf{Comparação de Estilos}: Comparada com o mesmo estilo de cerveja sem \textit{brettanomyces}, uma Brett Beer será mais seca, mais atenuada, mais frutada, com corpo mais leve e levemente com mais \textit{funky} conforme envelhece. Menos acidez e profundidade do que Belgian wild ales.

\textbf{Instruções para Inscrição}: O participante do concurso deve especificar um estilo base ou fornecer uma descrição dos ingredientes ou características desejadas. O participante deve especificar as cepas de \textit{brettanomyces} utilizadas.

\textbf{Estatísticas}:: Varia com o estilo base.

\textbf{Commercial Examples}: Boulevard Saison Brett, Hill Farmstead Arthur, Logsdon Seizoen Bretta, Lost Abbey Brett Devo, Russian River Sanctification, The Bruery Saison Rue.

\textbf{Atributos de Estilo}: craft-style, north-america, specialty-beer, wild-fermentation

