\phantomsection
\subsection*{23D. Lambic}
\addcontentsline{toc}{subsection}{23D. Lambic}
\textbf{Impressão Geral}: Uma cerveja de trigo belga, selvagem, bastante ácida, muitas vezes com notas moderadas \textit{funky}, com a acidez tomando o lugar do amargor do lúpulo no equilíbrio. Tradicionalmente servida sem gás como um drink de café.

\textbf{Aparência}: Cor amarela clara a dourada profunda, a idade tende a escurecer a cerveja. Limpidez é de boa a turva. As versões mais novas costumam ser turvas, enquanto as mais antigas geralmente são límpidas. A espuma, de cor branca, geralmente tem pouca retenção.

\textbf{Aroma}: As versões jovens podem ser bastante ácidas e frutadas, mas com o tempo podem desenvolver um sabor de curral, terroso, de cabra, de feno, de cavalo ou de manta de cavalo. O caráter de fruta pode assumir uma leve qualidade de frutas cítricas, casca de frutas cítricas, frutas de pomar ou ruibarbo, tornando-se mais complexo com o tempo. O malte pode ter um leve caráter de pão, cereais, mel ou trigo, quando perceptível. Não deve apresentar defeitos entéricos, defumados, semelhantes a charutos ou de queijo. Sem lúpulo

\textbf{Sabor}: As versões jovens costumam ter uma forte acidez láctica com sabores frutados (mesmos descritores do aroma), enquanto as versões envelhecidas são mais equilibradas e complexas. Notas funky podem se desenvolver ao longo do tempo, com os mesmos descritores de aroma. Malte de baixo teor como pão e cereais. Amargor geralmente abaixo do limiar sensorial; a acidez fornece o equilíbrio. Sem sabor de lúpulo. Final seco, aumentando com o tempo. Não deve apresentar defeitos entéricos, defumados, semelhantes a charutos ou de queijo.

\textbf{Sensação na Boca}: Corpo de baixo a médio-baixo; não deve ser aguada. Tem uma acidez de média a alta, trazendo a capacidade de franzir o rosto sem ser acentuadamente adstringente. As versões tradicionais são virtualmente a completamente não carbonatadas, mas os exemplares engarrafados podem adquirir carbonatação moderada com o tempo.

\textbf{Comentários}: Uma cerveja de lote único, sem mistura, refletindo o caráter da casa da cervejaria. Geralmente servida jovem (6 meses) do barril. Versões mais jovens tendem a ser unidimensionais ácidas, já que o caráter complexo de Brett leva um ano ou mais para se desenvolver. Um caráter avinagrado ou de cidra perceptível é considerado um defeito pelos cervejeiros belgas. Normalmente é engarrafada apenas quando completamente fermentada. A Lambic adoçada com açúcar mascavo na hora do serviço é conhecida como Faro.

\textbf{História}: As cervejas "selvagens" fermentadas espontaneamente da área de Bruxelas e arredores (também conhecidas como Vale do Senne e Pajottenland) derivam de uma tradição cervejeira de fazenda com vários séculos de idade.

\textbf{Ingredientes}: Malte Pilsner, trigo não maltado. Lúpulo envelhecido (mais de 3 anos) usado mais como conservante do que para amargor. Fermentada espontaneamente com leveduras e bactérias naturais, em barris de carvalho neutro e bem usados.

\textbf{Comparação de Estilos}: Muitas vezes tem uma acidez mais simples e menos complexidade do que um Gueuze, mas apresenta mais variação de lote para lote. Tradicionalmente servida sem gás e em jarros, enquanto a Gueuze é engarrafada e altamente carbonatada.

\begin{tabular}{@{}p{35mm}p{35mm}@{}}
  \textbf{Estatísticas}: & OG: 1,040 - 1,054 \\
  IBU: 0 - 10  & FG: 1,001 - 1,010  \\
  SRM: 3 - 6  & ABV: 5\% - 6,5%
\end{tabular}

\textbf{Exemplos Comerciais}: Cantillon Grand Cru Bruocsella. In the Brussels area, many specialty cafés have draught lambic from Boon, De Cam, Cantillon, Drie Fonteinen, Lindemans, Timmermans, Girardin and others.

\textbf{Última Revisão}: Lambic (2015)

\textbf{Atributos de Estilo}: pale-color, sour, standard-strength, traditional-style, western-europe, wheat-beer-family, wild-fermented
