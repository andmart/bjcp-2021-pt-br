\phantomsection
\subsection*{18B. American Pale Ale}
\addcontentsline{toc}{subsection}{18B. American Pale Ale}

\textbf{Impressão Geral}: Uma cerveja artesanal americana clara, de força média, orientada para o lúpulo, com suporte suficiente do malte para tornar a cerveja equilibrada e fácil de beber. Uma clara presença de lúpulo pode refletir variedades clássicas ou modernas de lúpulos americanos ou do Novo Mundo, com uma ampla gama de características.

\textbf{Aparência}: Cor de dourado claro ao âmbar. Colarinho moderadamente volumoso, de branco a quase branco, com boa retenção. Geralmente bastante límpida.

\textbf{Aroma}: Aroma de lúpulo de moderado a moderadamente alto de variedades de lúpulos americanos ou do Novo Mundo, com uma ampla gama de características possíveis, incluindo cítricos, florais, de pinho, resinoso, condimentado, de frutas tropicais, de frutas de caroço, de frutas vermelhas ou de melão. Nenhuma dessas características específicas é necessária, mas um aroma lupulado deve ser aparente. De baixo a moderado aroma maltado neutro a de grãos dá suporte a apresentação do lúpulo e pode mostrar pequenas quantidades de caráter de malte especial (por exemplo, pão, torrada, biscoito, caramelo). Ésteres frutados são opcionais, até moderados em força. Aroma fresco de \textit{dry-hopping} é opcional.

\textbf{Sabor}: Caráter de lúpulo e malte semelhante ao aroma (aplicam-se as mesmas intensidades e descritores). Os sabores de caramelo geralmente estão ausentes ou muito contidos, mas são aceitáveis, desde que não entrem em conflito com o lúpulo. Amargor de moderado a alto. Perfil limpo de fermentação. Os ésteres de levedura frutados podem ser de moderados a nenhum, embora muitas variedades de lúpulo sejam bastante frutadas. Final de médio a seco. O equilíbrio é tipicamente para o lúpulo com adição tardia e amargor; a presença de malte deve ser como suporte, não uma distração. O sabor e o amargor do lúpulo geralmente permanecem no final, mas o sabor residual geralmente deve ser limpo e não áspero. Sabor fresco de \textit{dry-hopping} x  é opcional.

\textbf{Sensação na Boca}: Corpo de médio leve a médio. Carbonatação de moderada a alta. Acabamento geral macio, sem adstringência ou aspereza.

\textbf{Comentários}: As versões americanas modernas geralmente são apenas IPAs de menor densidade. Tradicionalmente era um estilo que permitia a experimentação com maneiras de utilização e variedades de lúpulo, que agora podem ser encontrados como adaptações internacionais em países com um mercado emergente de cerveja artesanal. Os juízes devem considerar as características dos lúpulos americanos modernos ou do Novo Mundo à medida que são desenvolvidos e lançados.

\textbf{História}: Uma adaptação moderna do mercado americano de cerveja artesanal da English pale ale, refletindo ingredientes regionais. A Sierra Nevada Pale Ale foi produzida pela primeira vez em 1980 e ajudou a popularizar o estilo. Antes da explosão de popularidade das IPAs, esse estilo era o mais conhecido e o mais popular das cervejas artesanais americanas.

\textbf{Ingredientes}: Malte pale ale neutro. Lúpulo americano ou do Novo Mundo. Levedura ale americana neutra ou inglesa levemente frutada. Pequenas quantidades de vários maltes especiais.

\textbf{Comparação de Estilos}: Normalmente de cor mais clara, mais limpo no perfil de fermentação e com menos sabores de caramelo do que as equivalentes ingleses. Pode haver alguma sobreposição de cores entre a American Pale Ale e a American Amber Ale. A American Pale Ale geralmente será mais limpa, terá um perfil de malte menos caramelizado, menos corpo e muitas vezes mais lúpulos de finalização. Menos amargor no equilíbrio e teor alcoólico que uma American IPA. Mais maltada, mais equilibrada, fácil de beber, e menos intensamente focada no lúpulo e amarga do que as American IPAs com força \textit{session} x  (também conhecidas como Session IPAs). Mais amarga e lupulada que uma Blonde Ale.

\begin{tabular}{@{}p{35mm}p{35mm}@{}}
  \textbf{Estatísticas}: & OG: 1,045 - 1,060 \\
  IBU: 30 - 50  & FG: 1,010 - 1,015  \\
  SRM: 5 - 10  & ABV: 4,5\% - 6,2\%
\end{tabular}

\textbf{Exemplos Comerciais}: Deschutes Mirror Pond Pale Ale, Half Acre Daisy Cutter Pale Ale, Great Lakes Burning River, La Cumbre Acclimated APA, Sierra Nevada Pale Ale, Stone Pale Ale 2,0.

\textbf{Última Revisão}: American Pale Ale (2015)

\textbf{Atributos de Estilo}: bitter, craft-style, hoppy, north-america, pale-ale-family, pale-color, standard-strength, top-fermented
