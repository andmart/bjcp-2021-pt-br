\phantomsection
\subsection*{9C. Baltic Porter}
\addcontentsline{toc}{subsection}{9C. Baltic Porter}
\textbf{Impressão Geral}: Uma cerveja forte, escura e maltada com diferentes interpretações dentro da região do Báltico. Suave, aquecedora e ricamente maltada, com sabores complexos de frutas escuras e um sabor torrado sem notas queimadas.

\textbf{Aparência}: De cobre avermelhado escuro a marrom escuro opaco, mas não preto. Espuma espessa e persistente de cor castanho claro. Límpida, embora as versões mais escuras possam ser opacas.

\textbf{Aroma}: Maltado rico, normalmente contendo notas de caramelo, toffee, nozes, tostado profundo e/ou alcaçuz. Perfis de álcool e éster complexos e de intensidade moderada, lembrando ameixa, ameixa seca, passas, cereja e/ou groselha, ocasionalmente com uma qualidade como vinho do Porto. Notas profundas de malte como chocolate amargo, café e/ou melaço, mas nunca queimado. Sem lúpulo. Sem acidez. Impressão suave, não acentuada.

\textbf{Sabor}: Assim como no aroma, tem um maltado rico com uma combinação complexa e profunda de malte, ésteres de frutas secas e álcool. O malte pode ter uma complexidade de caramelo, toffee, nozes, melaço e/ou alcaçuz. Sabor torrado proeminente porém suave semelhante ao de uma Schwarzbier, que não chega a ser queimado. Leves toques de cassis negro e frutas secas escuras. No palato, suave e com final cheio. O sabor começa com dulçor de malte, mas os sabores de malte mais escuros dominam rapidamente e persistem durante o final um tanto seco, deixando um toque de café torrado e/ou alcaçuz e frutas secas no retrogosto. Amargor de médio-baixo a médio, apenas para dar equilíbrio e não parecer enjoativa. Sabor de lúpulo levemente condimentado varia de nenhum a médio-baixo. Perfil de fermentação limpo.

\textbf{Sensação na Boca}: Geralmente bastante encorpada e suave, com aquecimento alcoólico envelhecido que pode enganar. Carbonatação de média a média-alta fazendo com que pareça com preenchimento de boca ainda mais envolvente. Não é pesada na língua devido ao nível de carbonatação.

\textbf{Comentários}: A maioria das versões comerciais está na faixa de 7 a 8,5\% ABV. Os melhores exemplares têm teor alcoólico mascarado, que os torna perigosamente fáceis de beber. O caráter dessas cervejas varia de acordo com o país de origem, portanto tenha cuidado ao generalizar com base em um único exemplar. Algumas cervejas são mais fiéis às suas raízes inglesas enquanto outras são mais ao estilo popularizado primeiramente na Polônia.

\textbf{História}: Desenvolvida de forma nativa (e independente) em vários países que faziam fronteira com o Mar Báltico após a interrupção da importação das populares porters e stouts inglesas, no início do século XIX. Historicamente de alta fermentação, muitas cervejarias adaptaram as receitas com levedura de baixa fermentação junto com o restante de sua produção. O nome Baltic Porter é recente (desde a década de 1990) e descreve a coletânea moderna de cervejas com um perfil um tanto semelhante desses países e não versões históricas.

\textbf{Ingredientes}: Geralmente levedura lager (fermentada a frio se utilizar levedura ale, como é obrigatório quando fabricada na Rússia). Malte escuro sem casca. Malte base Munich ou Vienna. Lúpulos da Europa continental. Pode conter malte cristal ou adjuntos. Malte brown ou amber são comuns em receitas históricas. Como uma coletânea de cervejas regionais são esperadas diferentes formulações.

\textbf{Comparação de Estilos}: Combina o corpo, o maltado, a riqueza e a suavidade de uma Doppelbock, o caráter de malte mais escuro de uma English Porter, os sabores torrados de uma Schwarzbier e o álcool e o frutado de uma Old Ale. Muito menos torrado e frequentemente com menor teor alcoólico do que uma Imperial Stout.

\begin{tabular}{@{}p{35mm}p{35mm}@{}}
  \textbf{Estatísticas}: & OG: 1,060 - 1,090 \\
  IBU: 20 - 40 & FG: 1,016 - 1,024 \\
  SRM: 17 - 30 & ABV: 6,5\% - 9,5\%
\end{tabular}

\textbf{Exemplos Comerciais}: Aldaris Mežpils Porteris, Baltika \#6 Porter, Devils Backbone Danzig, Okocim Mistrzowski Porter, Sinebrychoff Porter, Zywiec Porter.

\textbf{Última Revisão}: Baltic Porter (2015)

\textbf{Atributos de Estilo}: any-fermentation, dark-color, eastern-europe, high-strength, lagered, malty, porter-family, traditional-style
