\phantomsection
\subsection*{21B. Specialty IPA: Brut IPA}
\addcontentsline{toc}{subsection}{21B. Specialty IPA: Brut IPA}

\textbf{Impressão Geral}: Uma variante de American IPA muito clara e lupulada, com um final seco, carbonatação muito alta e um nível de amargor moderado. Pode lembrar um espumante branco ou champanhe. O caráter do lúpulo é moderno e enfatiza diferentes camadas de sabor e aroma.

\textbf{Aparência}: Cor muito clara, variando de palha muito clara a ouro muito claro. Cristalina, mas um toque de turbidez é aceitável. A carbonatação, de alta a muito alta, gera um colarinho branco sólido, denso, persistente, com bolhas firmes e constantes.

\textbf{Aroma}: Aroma de lúpulo de moderadamente alto a intenso, muito vívida e orientada para o lúpulo no equilíbrio. As variedades modernas de lúpulo americano e do Novo Mundo fornecem uma ampla gama de características possíveis, como tropical, de fruta de caroço, cítrica ou de uva branca, não pode ser gramínea, vegetal ou herbal. O malte é sutil, neutro e vem em segundo plano, mas nunca como caramelo ou excessivamente doce. Uma nota de álcool leve e limpa é opcional. Caráter de fermentação muito limpo; não deve trazer notas de levedura.

\textbf{Sabor}: Sabor de lúpulo de alto a muito alto, mesmos descritores do aroma. Caráter de malte neutro, de baixo a muito baixo, sendo sutil no equilíbrio. Sem sabores fortes de malte, sem caramelo. O amargor percebido é de baixo a muito baixo devido ao final seco e à carbonatação muito alta. Perfil de fermentação de neutro a levemente frutado. Sem diacetil. Final de seco a muito seco, com um retrogosto fresco, lupulado e um amargor limpo.

\textbf{Sensação na Boca}: Corpo de leve a muito leve, com carbonatação vigorosa (de alta a muito alta), lembrando um vinho branco espumante. Sem adstringência, amargor ou aspereza derivada do lúpulo. O aquecimento alcoólico pode estar presente, mas nunca deve ser quente.

\textbf{Comentários}: 'Brut' é um termo do mundo do vinho que indica secura. O conceito original era uma IPA semelhante a um vinho espumante, embora o caráter do lúpulo agora varie mais amplamente. Densidade final muito baixa e alta carbonatação tornam o equilíbrio crítico, muitas vezes exigindo um amargor na receita surpreendentemente baixo. As enzimas adicionadas podem causar diacetil quando usadas incorretamente, o que é sempre uma falha.

\textbf{História}: Um estilo moderno de cerveja artesanal originado em 2017 na Social Kitchen \& Brewery em San Francisco (que agora está fechada), como uma reação da costa oeste à crescente tendência das hazy e juicy IPAs da costa leste, bem como as chamadas milkshake IPAs espessas e doces. O estilo ainda está evoluindo e mudando (e talvez morrendo, já que a cerveja estava bastante na moda em 2018-2019 nos EUA). A maioria das versões parece estar se transformando em IPAs de baixa caloria.

\textbf{Ingredientes}: Maltes base pilsen ou muito claros com até 40\% de adjuntos. Sem malte crystal ou lactose. Enzimas, como a amiloglucosidase. Lúpulos altamente aromáticos americanos ou do Novo Mundo, com bastante quantidade de óleos, usados com diferentes técnicas de adições tardias ou pós-fervura, para enfatizar o aroma e o sabor do lúpulo e minimizar o amargor. Levedura neutra.

\textbf{Comparação de Estilos}: Menos sabor de malte, amargor e cor do que uma American IPA, muito mais seca e mais carbonatada. Dry-hopping similar ao de uma American IPA. Aroma e sabor semelhantes a uma Hazy IPA, mas sem o dulçor e com muito menos turbidez. Muito clara, altamente carbonatada e seca como uma Belgian Golden Strong Ale, mas com teor alcoólico não tão alto e sem o caráter de levedura belga.

\begin{tabular}{@{}p{35mm}p{35mm}@{}}
  \textbf{Estatísticas}: & OG: 1,046 - 1,057 \\
  IBU: 20 - 30  & FG: 0,990 - 1,004 \\
  SRM: 2 - 4  & ABV: 6\% - 7,5\%
\end{tabular}

\textbf{Exemplos Comerciais}: Drake's Brightside Extra Brut IPA, Fair State Brewing Co-Op The Brut Squad, Ommegang Brut IPA.

\textbf{Atributos de Estilo}: bitter, craft-style, high-strength, hoppy, ipa-family, north-america, pale-color, specialty-family, top-fermented
