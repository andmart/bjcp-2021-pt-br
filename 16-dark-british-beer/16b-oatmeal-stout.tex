\phantomsection
\subsection*{16B. Oatmeal Stout}
\addcontentsline{toc}{subsection}{16B. Oatmeal Stout}
\textbf{Impressão Geral}: Uma stout escura, torrada e encorpada com doçura suficiente para sustentar a base de aveia. O dulçor, equilíbrio e impressão de aveia podem variar consideravelmente.

\textbf{Aparência}: Cor marrom a preta. Colarinho espesso, cremoso e persistente de cor castanho claro a marrom. Límpida, se não opaco.

\textbf{Aroma}: Caráter leve de cereais, torrado, semelhante ao café, com um leve dulçor de malte que pode dar uma impressão de café com creme. Frutado baixo a médio-alto. Médio-baixo aroma de lúpulo terroso ou floral opcional. Um leve aroma de nozes e cereais vindos da aveia é opcional. Diacetil médio-baixo opcional, mas normalmente ausente.

\textbf{Sabor}: Semelhante ao aroma, com um sabor suave de café torrado, chocolate ao leite ou café com creme, frutado baixo a moderadamente alto. A aveia pode adicionar um sabor de nozes tostadas, cereais ou terroso. Amargor médio. Final meio doce a meio seco, o que afeta a percepção de equilíbrio. Retrogosto maltado, torrado e de nozes. Sabor de lúpulo terroso ou floral médio-baixo opcional. Diacetil médio-baixo opcional, mas normalmente ausente.

\textbf{Sensação na Boca}: Corpo médio-cheio a cheio, com um toque macio, sedoso, aveludado, às vezes quase oleosa da aveia. Cremosa. Carbonatação média a média-alta. Versões com teor alcoólico mais alto podem aquecer levemente.

\textbf{Comentários}: Ao julgar, permita diferenças de equilíbrio e interpretação. As versões americanas tendem a ser mais lupuladas, menos doces e menos frutadas do que as inglesas. O amargor, o dulçor e a impressão de aveia variam. O uso leve de aveia pode dar uma certa sedosidade ao corpo e riqueza de sabor, enquanto o uso pesado de aveia pode ser bastante intenso em sabor com uma sensação na boca quase oleosa e final seco.

\textbf{História}: Uma variante de stouts usando aveia no conjunto de cereais, nutritiva ou para enfermos, surgiu por volta de 1900, semelhante, mas independente do desenvolvimento da sweet stout que usava lactose. A versão original escocesa usava uma quantidade significativa de malte de aveia. Mais tarde, passou por uma fase sombria em que alguns cervejeiros ingleses jogavam um punhado de aveia em suas stouts feitas pelo método part-gyled para produzir legalmente uma Oatmeal Stout 'saudável' para fins de marketing. Mais popular na Inglaterra entre as Guerras Mundiais, foi revivido na era da cerveja artesanal para exportação, o que ajudou a levar à sua adoção como um estilo popular de cerveja artesanal americana moderna que usa uma quantidade notável (não simbólica) de aveia.

\textbf{Ingredientes}: Maltes pale, caramelo e torrados escuros (geralmente chocolate) e cereais. Aveia ou aveia maltada (5-20\% ou mais). Lúpulo principalmente para amargor. Pode usar açúcares cervejeiros ou xaropes. Levedura de cerveja inglesa.

\textbf{Comparação de Estilos}: A maioria é como um cruzamento entre uma Irish Extra Stout e uma Sweet Stout com adição de aveia. Existem muitas variações, com as versões mais doces mais parecidas com uma Sweet Stout com aveia em vez de lactose, e as versões mais secas, como uma Irish Extra Stout mais saborosa e com nozes. Ambos tendem a enfatizar o corpo e a sensação na boca.

\begin{tabular}{@{}p{35mm}p{35mm}@{}}
  \textbf{Estatísticas}: & OG: 1,045 - 1,065 \\
  IBU: 25 - 40  & FG: 1,010 - 1,018 \\
  SRM: 22 - 40  & ABV: 4,2\% - 5,9\%
\end{tabular}

\textbf{Exemplos Comerciais}: Anderson Valley Barney Flats Oatmeal Stout, St-Ambroise Oatmeal Stout, Samuel Smith Oatmeal Stout, Broughton Stout Jock, Summit Oatmeal Stout, Young's London Stout.

\textbf{Revisão anterior}: Oatmeal Stout (2015)

\textbf{Atributos de Estilo}: balanced, british-isles, dark-color, roasty, standard-strength, stout-family, top-fermented, traditional-style
