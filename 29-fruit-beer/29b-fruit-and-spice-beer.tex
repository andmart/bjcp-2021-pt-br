\phantomsection
\subsection*{29B. Fruit and Spice Beer}
\addcontentsline{toc}{subsection}{29B. Fruit and Spice Beer}
\textit{Utilize as definições de Frutas no preâmbulo da Categoria 29 e de Especiarias no preâmbulo da Categoria 30; qualquer combinação de ingredientes válida nos Estilos 29A e 30A é permitida nesta categoria. Para este estilo, a palavra "especiaria" significa "qualquer SHV" – sigla em inglês para especiarias, ervas e vegetais.}\\
\textbf{Impressão Geral}: Uma saborosa união entre frutas, especiarias e cerveja, mas ainda assim reconhecível como cerveja. O caráter de frutas e de especiarias devem estar em equilíbrio com a cerveja e ser evidentes, mas não proeminentes a ponto de sugerir um produto artificial.

\textbf{Aparência}: Varia de acordo com o estilo base e ingredientes especiais utilizados. Cervejas de cor mais clara, incluindo o colarinho, podem apresentar cores vindas da adição. A cor da fruta na cerveja é geralmente mais clara do que a cor da polpa da própria fruta e pode ter tons ligeiramente diferentes. Limpidez variável, ainda que a turbidez seja geralmente não desejável. Alguns ingredientes podem impactar na retenção da espuma.

\textbf{Aroma}: Varia de acordo com o estilo base. O caráter de fruta e de especiaria deve ser notável no aroma; no entanto, algumas frutas e especiarias (p. ex., framboesas, cerejas, canela, gengibre) têm aromas mais intensos e distintos do que outras (p. ex., mirtilos e morangos) - o que permite uma gama de caráter e intensidade de frutas, de sutil a agressivo. Aroma de lúpulo pode ser mais baixo do que o esperado para o estilo base, para mostrar melhor o caráter de especialidade. Os ingredientes especiais devem adicionar complexidade, mas não devem estar proeminentes a ponto de desequilibrar o produto final.

\textbf{Sabor}: Varia de acordo com o estilo base. Como no aroma, um distinto sabor de fruta e de especiaria deve ser notável e pode variar de intensidade, de sutil a agressivo, porém o caráter da fruta não deve ser tão artificial ou inapropriadamente dominante a ponto de sugerir um "suco de frutas com especiarias". Amargor, sabores de lúpulo e malte, teor alcoólico e subprodutos de fermentação, como ésteres, devem ser apropriados para o estilo base, porém harmoniosos e equilibrados com os distintos sabores de frutas e especiarias presentes. A fruta geralmente adiciona sabor, mas não dulçor; normalmente, o açúcar encontrado em frutas é totalmente fermentado e contribui para sabores mais leves e um final mais seco. Entretanto, dulçor residual não é necessariamente uma caraterística negativa, a não ser que traga um caráter cru, de não fermentado. Alguns ingredientes podem adicionar acidez, amargor e/ou taninos, que devem estar em equilíbrio no perfil final de sabor.

\textbf{Sensação na Boca}: Varia de acordo com o estilo base. A fruta muitas vezes diminui o corpo e faz a cerveja parecer mais leve no palato. Algumas frutas menores e mais escuras podem adicionar uma profundidade tânica, mas essa adstringência não deve sobrepujar a cerveja base. Alguns SHVs podem adicionar um pouco de adstringência, ainda que um caráter de especiaria "crua" não seja desejável.

\textbf{Comentários}: A descrição da cerveja é crítica para avaliação. Os juízes devem pensar mais no conceito declarado do que tentar detectar individualmente cada ingrediente utilizado. Equilíbrio, facilidade de beber e execução do tema (do estilo proposto) são os fatores mais importantes. Os ingredientes especiais devem complementar o estilo base, não sobrecarregá-lo. Os atributos do estilo base serão diferentes após a adição de fruta e especiaria, ou seja, não espere que a cerveja tenha o sabor idêntico ao do estilo base inalterado.

\textbf{Instruções para Inscrição}: O participante deve especificar o tipo de fruta e o tipo de especiarias, ervas e vegetais utilizados. Ingredientes individuais de especiarias, ervas e vegetais não necessitam serem especificados, caso façam parte de uma combinação conhecida de especiarias (p. ex., torta de maçã com especiarias). O participante deve fornecer uma descrição da cerveja, identificando o estilo base e os ingredientes, estatísticas e/ou caráter desejado para a cerveja. Uma descrição geral da natureza especial da cerveja pode abranger todos os itens necessários.

\textbf{Estatísticas}: OG, FG, IBU, SRM e ABV vão variar de acordo com a cerveja base, mas a fruta poderá impactar na coloração.

\textbf{Exemplos Comerciais}: Cigar City Margarita Gose, Firestone Walker Chocolate Cherry Stout, Golden Road Spicy Mango Cart, Kona Island Colada Cream Ale, New Glarus Blueberry Cocoa Stout, Sun King Orange Vanilla Sunlight.

\textbf{Última Revisão}: Fruit and Spice Beer (2015)

\textbf{Atributos de Estilo}: fruit, specialty-beer, spice
