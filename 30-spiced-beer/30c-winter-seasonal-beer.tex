\phantomsection
\subsection*{30C. Winter Seasonal Beer}
\addcontentsline{toc}{subsection}{30C. Winter Seasonal Beer}
\textit{Winter Seasonal Beer são cervejas que sugerem clima frio e a temporada de festas de Natal (do hemisfério Norte), e podem incluir especiairias, açúcares especiais e outros produtos que lembram a época festiva.}
\textbf{Impressão Geral}: Uma cerveja mais forte, escura e condimentada, que muitas vezes possui um corpo rico e um final que aquece, sugerindo um bom acompanhamento para a temporada fria de inverno.

\textbf{Aparência}: Cor de âmbar médio a marrom muito escuro (as versões mais escuras são mais comuns). Límpida, se não for opaca; normalmente límpida, embora versões mais escuras podem ser virtualmente opacas. Colarinho de quase branco à castanho, com boa formação e persistente.

\textbf{Aroma}: Maltado, condimentado, frutado e equilibrado. Uma ampla faixa é possível, desde que recorde o tema de férias de inverno (ou festividades de fim de ano, do Hemisfério Norte). Os ingredientes declarados e o conceito definem a expectativa. Muitas vezes com caráter de frutas escuras ou secas. Lúpulos são geralmente sutis. Álcool muitas vezes presente, porém suave e de suporte. Aromas maltados ou de açúcar tendem a trazer para si o equilíbrio geral e servem de suporte para as especiarias. Os componentes devem estar bem integrados e criar uma apresentação coerente. Veja a seção de Sabor para os perfis de especiaria, malte, açúcar e fruta.

\textbf{Sabor}: Maltado, condimentado e equilibrado. Deve-se permitir a criatividade do cervejeiro para atingir o objetivo do tema. Especiarias adocicadas ou aquecedoras são comuns. Sabores ricos e tostados de malte são comuns e podem incluir sabores de caramelo, pão tostado, nozes ou chocolate. Podem incluir sabores de frutas secas ou de suas cascas, como passas, ameixa, figo, cereja, casca de laranja ou limão. Podem incluir característicos sabores de açúcares, como melaço, mel ou açúcar mascavo. Os ingredientes especiais devem dar suporte e equilíbrio, sem ofucar a cerveja base. Amargor e sabor de lúpulo são normalmente restritos, para não interferirem no perfil especial. Geralmente, o final é cheio e satisfatório, ocasionalmente com um leve sabor de álcool. Características de malte torrado são raras e, normalmente, não são mais fortes do que as de chocolate.

\textbf{Sensação na Boca}: Corpo geralmente de médio a alto, muitas vezes viscosa. Carbonatação de moderadamente baixa a moderadamente alta. Caráter de envelhecimento e aquecimento alcoólico são permitidos.

\textbf{Comentários}: Utilizando o perfil sensorial de ingredientes que sugerem a temporada de de férias de fim de ano (do Hemisfério Norte), como biscoitos de Natal, biscoitos de gengibre, bolo inglês de Natal, bolo de rum, gemada, pinheiros, mix de ervas secas ou especiarias de vinho quente, em equilíbrio com uma cerveja base de suporte, muitas vezes maltada, aquecedora e escura. A descrição da cerveja é crítica para avaliação. Os juízes devem pensar mais no conceito declarado do que tentar detectar individualmente cada ingrediente utilizado. Equilíbrio, facilidade de beber e execução do tema (do estilo proposto) são os fatores mais importantes.

\textbf{História}: A temporada de festas de inverno é uma época tradicional no Hemisfério Norte, em que velhos amigos se encontram e as cervejas um pouco mais alcoólicas e ricas são servidas. Muitos cervejeiros oferecem produtos sazonais que podem ser mais escuros, fortes, condimentados, ou com mais personalidade que as cervejas feitas durante todo o ano. Versões condimentadas são uma tradição americana ou belga, uma vez que cervejarias alemãs ou inglesas tradicionalmente não utilizam especiarias em suas cervejas. Muitos cervejeiros artesanais americanos foram inspirados pela Anchor Our Special Ale, produzida pela primeira vez em 1975.

\textbf{Ingredientes}: Especiarias são necessárias e, muitas vezes, aquelas que remetem ao Natal do hemisfério Norte (p. ex., pimenta da Jamaica, noz-moscada, canela, cravo e gengibre), porém qualquer combinação é possível e a criatividade é incentivada. Cascas de frutas (p. ex., laranja e limão) podem ser utilizadas, assim como adições sutis de outras frutas (muitas vezes secas ou escuras). Adjuntos que fornecem sabor são comuns (p. ex., melaço, açúcar invertido, açúcar mascavo, mel, xarope de bordo). Normalmente ales, embora existam lagers fortes e escuras.

\textbf{Instruções para Inscrição}: O participante deve especificar o tipo de especiarias, açúcares, frutas ou fermentáveis utilizados, mas ingredientes individuais não necessitam serem especificados, caso façam parte de uma combinação conhecida de especiarias (p. ex., especiarias para vinho quente). O participante deve fornecer uma descrição da cerveja, identificando o estilo base e os ingredientes, estatísticas e/ou caráter desejado para a cerveja. Uma descrição geral da natureza especial da cerveja pode abranger todos os itens necessários.

\textbf{Estatísticas}: OG, FG, IBU, SRM e ABV vão variar dependendo do estilo base. ABV é geralmente acima de 6\% e a maioria dos exemplos apresenta cor mais escura.

\textbf{Exemplos Comerciais}: Anchor Christmas Ale, Great Lakes Christmas Ale, Harpoon Winter Warmer, Rogue Santa's Private Reserve, Schlafly Christmas Ale, Troeg's The Mad Elf.

\textbf{Última revisão}: Winter Seasonal Beer (2015)

\textbf{Atributos de Estilo}: specialty-beer, spice
