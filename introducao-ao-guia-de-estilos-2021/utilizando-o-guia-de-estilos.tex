\subsection*{Utilizando o Guia de Estilos}
\addcontentsline{toc}{subsection}{Utilizando o Guia de Estilos}
Quando publicamos as versões anteriores do guia de estilos, não tínhamos ideia de quão prevalentes e difundidas elas se tornariam.  Nós acreditávamos que estávamos criando uma padronização de um conjunto de descrições de estilos para competições de cervejeiros caseiros, mas depois descobrimos que elas foram amplamente adotadas à nível mundial para descrever cervejas de forma geral. Diversos países com mercados de cerveja artesanal emergentes passaram a usar o guia como um manual do que produzir. Consumidores e grupos comerciais passaram a utilizar os estilos para descrever seus produtos. Infelizmente, muitos fizeram suposições surpreendentes muito além de nossa intenção original, consequentemente utilizando o guia de estilos como uma espécie de “Rosetta Stone” universal para cerveja.

Ao passo que entendemos que o guia de estilos pode ter sido utilizado de forma errônea para fins além de nossa intenção original, também notamos ele ser utilizado de forma errada em competições e para outras propostas dentro do BJCP como preparação para exames e correção de exames. Algumas pessoas interpretaram de forma errada o guia de estilos e, sem querer, passaram adiante o mal-uso dele. Nossa esperança é que a informação contida nesta sessão ajude a evitar más interpretações e o uso equivocado no futuro. Caso você encontre alguém utilizando o guia de estilos de forma incorreta, por favor indique esta sessão.

As diretrizes a seguir expressam nossas intenções originais e foram feitas para limitar o mal-uso do guia, não para evitar que o guia seja adotado para novos usos:
\begin{enumerate}
\item \textbf{O Guia de estilos do BJCP é um guia e não uma especificação.} Entenda e não se esqueça destas palavras. O guia de estilos tem o propósito de descrever as características gerais e os exemplares mais comuns, serve para auxiliar no julgamento; não foi criado para ser uma especificação que deve ser aplicada de forma rigorosa e punir exemplares um pouco fora do esperado. São sugestões, não limites expressos. Permita alguma flexibilidade no julgamento para que amostras bem executadas possam ser recompensadas. As diretrizes são escritas em detalhes para facilitar o processo de avaliação estruturada de cervejas conforme exercitado no julgamento de competições caseiras, não use cada detalhe individual na descrição de estilos como um motivo para desqualificar uma amostra.
\item \textbf{O guia de estilos foi escrito pensando nas competições de cervejas caseiras.} Descrições individuais de estilos são escritas primordialmente para auxiliar no julgamento. Nós temos, em alguns casos, buscado definir parâmetros claros entre os estilos para criar categorias de julgamento que não se sobrepõem. Nós entendemos que alguns estilos podem se sobrepor no mercado, alguns exemplares comerciais podem ir além de parâmetros do estilo. Nós organizamos as categorias de estilo com o propósito de organizar competições de cervejas caseiras, não para descrever e comunicar os estilos do mundo para outro público.
\item \textbf{Sabemos que muitas pessoas usam nosso guia de estilos.} Entendemos que outras organizações e grupos usam nosso guia de estilos para propostas que vão muito além de nossa proposta inicial. No sentindo que estes grupos enxergam valor em nosso trabalho, estamos felizes em ter nosso guia de estilos utilizado. Permitimos o uso de nossa nomenclatura e sistema de numeração de forma livre por outras pessoas. No entanto, não faça presunções precipitadas a respeito da natureza da cerveja e dos estilos baseado na aplicação do guia do estilo além de sua finalidade original. Nós também sabemos que alguns cervejeiros artesanais estão utilizando o guia de estilos para descobrir estilos históricos, ou para produzir estilos não nativos a seus respectivos países – estamos muito entusiasmados em poder ajudar a cerveja artesanal a avançar desta forma. Porém se lembre que esta não é nossa missão original, é apenas um efeito colateral bom.
\item \textbf{Estilos mudam com o tempo.} Estilos de cerveja mudam com o passar dos anos, alguns estilos estão abertos para interpretação e debate. Só porque o nome do estilo não mudou com os anos, isto não significa que as cervejas em si também não mudaram. Cervejeiros comerciais sujeitos à regulamentação governamental e forças do mercado definitivamente mudam seus produtos ao longo do tempo. Por exemplo, porque temos agora uma cerveja conhecida como porter não significa que ela sempre foi feita desta forma ao longo de sua história. Nossos estilos de cerveja geralmente têm o propósito de descrever cervejas modernas que estão disponíveis atualmente, a menos que descrito de forma contrária (por exemplo, a categoria de cervejas históricas).
\item \textbf{Nem toda cerveja comercial se enquadra em nossos estilos.} Não assuma que toda cerveja se enquadra perfeitamente em alguma de nossas categorias. Algumas cervejarias se deleitam em criar exemplares que não se enquadram em nosso (ou em qualquer) guia de estilos. Algumas cervejarias criam cervejas chamadas pelo nome de um estilo que deliberadamente não se enquadram em nosso guia de estilos. É perfeitamente aceitável que uma cerveja comercial não se enquadre em um de nossos estilos, nós não tentamos categorizar todas cervejas comerciais – não é nossa intenção, tampouco nossa missão.
\item \textbf{Nós não definimos todos estilos de cervejas possíveis.} É claro que sabemos de estilos de cervejas que não estão definidos em nosso guia. Talvez por serem obscuros ou pouco populares, ou porque cervejeiros caseiros não estão fazendo estes estilos, não existem exemplares suficientes ou material de pesquisa que os definem de forma adequada aos nossos padrões, ou estão em uma parte do mundo que não visitamos de forma extensiva. Talvez por serem estilos históricos que não estejam mais sendo feitos, ou porque acreditamos que sejam estilos que são uma moda passageira. Independente de nossas razões, não acredite que nosso guia de estilos represente uma categorização de toda cerveja já feita – ele não é. O guia, no entanto, descreve as cervejas mais comumente feitas atualmente por cervejeiros caseiros e muitas cervejarias artesanais.
\item \textbf{Exemplares comerciais mudam com o tempo.} Assim como estilos de cerveja mudam, exemplares individuais também mudam. Uma cerveja que já foi outrora um grande exemplar do estilo pode deixar de permanecer como um. Algumas vezes a cerveja muda (possivelmente com uma mudança de proprietário da marca) ou as vezes com uma mudança de tendência do estilo que o exemplar não acompanha. Por exemplo, a Anchor Liberty ajudou a definir o estilo American IPA quando foi criada, mas hoje se assemelha mais com uma típica American Pale Ale.
\item \textbf{Ingredientes mudam com o tempo.} Lúpulos são um ótimo exemplo atual, novas variedades estão chegando ao mercado com características únicas. Cervejeiros buscando se diferenciar podem rapidamente adotar (ou abandonar) ingredientes. É difícil dizer que o perfil de um estilo de cerveja é fixo quando os ingredientes típicos mudam com frequência. Permita estas mudanças ao julgar cervejas. Por exemplo, nem todos lúpulos americanos serão cítricos ou remeterão à pinho. Não seja rígido ao julgar se baseando no que era comumente utilizado no período em que o estilo foi escrito; entenda que ingredientes são tipicamente utilizados e adapte seu julgamento baseado nos perfis que estão evoluindo.
\item \textbf{A maioria dos estilos são razoavelmente abrangentes.} Alguns acreditam que nossos estilos proíbem a criatividade do cervejeiro por definir barreiras rígidas. Esta não é nossa intenção – nós acreditamos que a criatividade leva à inovação, e esta interpretação por parte dos cervejeiros deve ser permitida. No entanto, nem toda inovação é uma ótima ideia ou resulta em uma cerveja que é reconhecível no mesmo agrupamento de outras que levam o mesmo nome. Então, estilos devem ser interpretados como tendo espaço para flexibilidade, mas tudo dentro da razão.
\item \textbf{O guia de estilos não é “os 10 mandamentos”.} As palavras neste documento não são provenientes da inspiração divina –foram escritas por pessoas tentando realizar um esforço de boa-fé para descrever as cervejas conforme são percebidas. Não trate o guia como um tipo de escritura sagrada. Não se perca tanto interpretando palavras individuais de forma que você perca a intenção geral. A parte mais importante de qualquer estilo é o equilíbrio e a impressão geral; isto é, se a cerveja te lembra do estilo e é um produto facilmente bebível. Se perder nas descrições individuais fará com que você perca a essência do estilo. O mero fato de que as descrições de estilo podem mudar de uma edição do guia para a próxima deve ilustrar de forma muito clara que as palavras em si não são sagradas.
\item \textbf{Nossas diretrizes são extensíveis.} Entendemos que nossas diretrizes de estilos irão mudar no futuro e que podem haver diversos anos entre os ciclos de revisão. A principal missão do BJCP é conduzir exames e alterações constantes tornam estudar quase impossível. Então, optamos por adotar um meio de campo: nós listamos estilos provisórios em nosso website que podem ser usados da mesma forma que os estilos presentes neste guia. Isto permite com que possamos adicionar mudanças entre as edições. Nós também temos uma lista de sugestões de enquadramento de inscrições em nosso website para melhor entendimento de qual a melhor forma de inscrever estilos não definidos em nossas diretrizes ou como estilos provisórios. Estes recursos, junto da extensão de alguns estilos como por exemplo Specialty IPA e Historical Beer, permitem estilos definidos por cervejeiros serem utilizados em competições.  Combinadas, estes três recursos permitem que o guia de estilos evolua entre atualizações maiores.
\item \textbf{Não somos a polícia da cerveja.} Categorizamos e descrevemos estilos de cerveja que percebemos que existem e que são usados. De forma alguma estamos falando para os cervejeiros comerciais o que eles podem produzir, ou falando que eles estão errados caso seus produtos não se adequem às nossas diretrizes. Nós também não criamos estilos na esperança de que se tornem populares. O estado geral do mercado de cerveja de qualquer país não é nossa preocupação.
\item \textbf{Diferentes formatos existem.} Nossas diretrizes de estilos aparecem em diversos locais de terceiros, diversas plataformas móveis (mobile) e também são traduzidos para outras línguas. Infelizmente, nem todas estas versões contêm o texto completo de nosso guia de estilos, ou possuem traduções completamente fieis. Seja cuidadoso ao utilizar formatos fornecidos por alguém que não seja o BJCP de forma direta, quando em dúvida, sempre consulte a fonte original.
\item \textbf{O BJCP não organiza competições.} Às vezes competições utilizam softwares que limitam comentários, ou tornam difícil de seguir as instruções de inscrição presentes no guia de estilos. Comunique problemas aos organizadores da competição e aos fornecedores do software. Nosso intuito é que toda informação permitida e requerida pelo guia de estilos seja fornecida pelos cervejeiros, seja aceita pelas competições e fornecida aos juízes
\end{enumerate}