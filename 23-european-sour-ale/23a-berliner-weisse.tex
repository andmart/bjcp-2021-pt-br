\phantomsection
\subsection*{23A. Berliner Weisse}
\addcontentsline{toc}{subsection}{23A. Berliner Weisse}

\textbf{Impressão Geral}: Uma cerveja de trigo alemã muito clara, refrescante, de baixo teor alcoólico, com uma acidez lática limpa e um nível de carbonatação muito alto. No malte, um leve sabor de massa de pão sustenta a acidez, que não deve parecer artificial. Um frutado suave aparece nos melhores exemplares.

\textbf{Aparência}: De coloração palha, pode ser bem clara. A limpidez varia de transparente a um tanto turva. Colarinho branco alto, denso e com retenção pobre. Altamente efervescente.

\textbf{Aroma}: Um caráter nitidamente ácido de moderado a moderadamente-alto é dominante. Pode chegar até um moderadamente frutado, geralmente limão, maçã azeda, pêssego ou damasco e uma nota floral leve. Sem aroma de lúpulo. Nas versões mais frescas o trigo pode ser notado como massa de pão crua e, combinado com a acidez, pode sugerir pão de fermentação natural.

\textbf{Sabor}: Acidez lática limpa domina e pode ser bastante forte. Um sabor complementar que remete a massa de pão, pão ou cereais como trigo é geralmente perceptível. Amargor de lúpulo indetectável; a acidez supre o equilíbrio no lugar dos lúpulos. Nunca avinagrado. Frutado nítido, apesar de restrito, pode ser detectado como pêssego-damasco, limão-cítrico ou maçã azeda. Final muito seco. O equilíbrio é dominado pela acidez, mas algum sabor de malte deve estar presente. Sem sabor de lúpulo. Sem THP.

\textbf{Sensação na Boca}: Corpo leve, mas nunca ralo. Carbonatação muito alta. Sem sensação de álcool. Acidez que cessa rapidamente (\textit{crisp}).

\textbf{Comentários}: Qualquer aspecto de Bretta é contido e tipicamente expressado na forma de notas florais e frutadas, nunca textit{funky}. Versões envelhecidas podem denotar características de cidra, mel, feno ou suave flor silvestre e, por vezes, acidez elevada. Na Alemanha é classificada como uma \textit{Schankbier}, ou seja, uma cerveja leve com densidade inicial na faixa de 7-8°P. Versões com adição de fruta ou especiarias devem ser inscritas como 29A Fruit Beer, como 30A Spice, Herb, or Vegetable Beer ou como 29B Fruit and Spice Beer.

\textbf{História}: Uma especialidade regional de Berlim. Em 1809 era referida como "a Champagne do Norte" pelas tropas de Napoleão devido à sua qualidade vibrante e elegante. Em dado momento era defumada e costumava haver uma versão de intensidade \textit{Märzen} (14°P). Cada vez mais rara na Alemanha, mas agora produzida em diversos outros países.

\textbf{Ingredientes}: Malte Pilsen. Usualmente malte de trigo, geralmente metade dos grãos, no mínimo. Uma co-fermentação simbiótica com levedura de alta fermentação e LAB (bactéria de ácido lático, da sigla em inglês) fornece a acidez acentuada, que pode ser melhorada pela mistura (durante a fermentação) de cervejas com tempos de envelhecimento distintos e pela maturação a frio. Decocção na mostura com \textit{mash hopping} (adição de lúpulos durante a mosturação) é tradicional. Cientistas alemães da brassagem acreditam que Bretta é essencial para obter o perfil de sabor frutado-floral correto.

\textbf{Comparação de Estilo}: Comparada à Lambic, possui uma acidez lática limpa com Bretta entre contida a abaixo do limiar sensorial. Também possui teor alcoólico mais baixo. Comparada à Straight Sour Beer e à Catharina Sour, apresenta densidade menor e pode conter Bretta.

\begin{tabular}{@{}p{35mm}p{35mm}@{}}
  \textbf{Estatísticas}: & OG: 1,028 - 1,032 \\
  IBU: 3 - 8  & FG: 1,003 - 1,006  \\
  SRM: 2 - 3  & ABV: 2,8\% - 3,8\%
\end{tabular}

\textbf{Exemplos Comerciais}: Bayerischer Bahnhof Berliner Style Weisse, Berliner Berg Berliner Weisse, Brauerei Meierei Weiße, Lemke Berlin Budike Weisse, Schell's Brewing Company Schelltheiss, Urban Chestnut Ku’damm.

\textbf{Última Revisão}: Berliner Weisse (2015)

\textbf{Atributos de Estilo}: central-europe, pale-color, session-strength, sour, top-fermented, traditional-style, wheat-beer-family
