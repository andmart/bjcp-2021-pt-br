\phantomsection
\subsection*{11A. Ordinary Bitter}
\addcontentsline{toc}{subsection}{11A. Ordinary Bitter}

\textbf{Impressão Geral}: Baixa densidade, álcool e carbonatação fazem desta uma cerveja session fácil de beber. O perfil do malte pode variar em sabor e intensidade, mas nunca deve sobrepor a impressão geral de amargor. A facilidade em beber é um componente crítico do estilo.

\textbf{Aparência}: Cor de âmbar claro a cobre claro. Limpidez de boa a brilhante. Colarinho de branco a quase branco de baixo a moderado. Pode ter muito pouco colarinho devido à baixa carbonatação.

\textbf{Aroma}: Aroma de malte de baixo a moderado, muitas vezes (mas nem sempre) com uma leve qualidade de caramelo. Complexidade de malte como pão, biscoito ou levemente tostado é comum. Frutado de leve a moderado. O aroma de lúpulo pode variar de moderado a nenhum, tipicamente com um caráter floral, terroso, resinoso ou frutado. Geralmente sem diacetil, embora níveis muito baixos sejam permitidos.

\textbf{Sabor}: Amargor de médio a moderadamente alto. Ésteres frutados de moderadamente baixos a moderadamente altos. Sabor de lúpulo de baixo a moderado, tipicamente com caráter terroso, resinoso, frutado ou floral. Maltosidade de baixa a média com final seco. O perfil de malte é tipicamente de pão, biscoito ou levemente tostado. Sabores de caramelo ou toffee de baixo a moderado são opcionais. O equilíbrio é muitas vezes decididamente amargo, embora o amargor não deva dominar completamente o sabor do malte, ésteres e lúpulo. Geralmente sem diacetil, embora níveis muito baixos sejam permitidos.

\textbf{Sensação na Boca}: Corpo de leve a médio-leve. Carbonatação baixa, embora exemplos engarrafados possam ter carbonatação moderada.

\textbf{Comentários}: A membra de menor densidade da família British Bitter, normalmente conhecida pelos consumidores simplesmente como “bitter” (embora os cervejeiros tendam a se referir a ela como Ordinary Bitter para distingui-la de outras membras da família).

\textbf{História}: Veja os comentários na introdução da categoria.

\textbf{Ingredientes}: Maltes pale ale, amber ou crystal. Pode usar um toque de malte escuro para ajuste de cor. Pode usar adjuntos como açúcar, milho ou trigo. Lúpulos de finalização ingleses são os mais tradicionais, mas qualquer lúpulo é um jogo justo; se forem usados lúpulos americanos, é necessário um leve toque. Levedura britânica com caráter presente.

\textbf{Comparação de Estilos}: Algumas variantes modernas são fabricadas exclusivamente com malte claro e são conhecidas como Golden Ales, Summer Ales ou Golden Bitters. A ênfase está na adição de lúpulo de amargor, em oposição ao agressivo lúpulo de adição tardia visto em ales americanas.

\begin{tabular}{@{}p{35mm}p{35mm}@{}}
  \textbf{Estatísticas}: & OG: 1,030 - 1,039 \\
  IBU: 25 - 35  & FG: 1,007 - 1,011  \\
  SRM: 8 - 14  & ABV: 3,2\% - 3,8\%
\end{tabular}

\textbf{Exemplos Comerciais}: Bateman’s XB, Brains Bitter, Brakspear Gravity, Fuller's Chiswick Bitter, Greene King IPA, Tetley’s Original Bitter.

\textbf{Última Revisão}: Ordinary Bitter (2015)

\textbf{Atributos de Estilo}: amber-ale-family, amber-color, bitter, british-isles, session-strength, top-fermented, traditional-style