\phantomsection
\subsection*{22C. American Barleywine}
\addcontentsline{toc}{subsection}{22C. American Barleywine}

\textbf{Impressão Geral}: Uma cerveja americana com teor alcoólico muito alto, maltada, lupulada e amarga; com paladar rico, sensação na boca cheia e retrogosto quente. Adequada para se degustar de maneira contemplativa.

\textbf{Aparência}: A cor varia de âmbar a cobre médio, raramente, até marrom claro. Reflexos rubis são comuns. Colarinho de quase branco a castanho claro, de moderadamente baixo a volumoso; pode ter baixa retenção. Limpidez de boa a brilhante, podendo ter algum \textit{chill haze}. A cor pode parecer ter grande profundidade, como se fosse vista através de uma lente de vidro grossa. Possibilidade de lágrimas.

\textbf{Aroma}: Aroma forte de malte e lúpulo dominam. O lúpulo é de moderado a assertivo, mostrando uma variedade de características americanas, do novo mundo ou inglesas. Cítrico, frutado ou resinoso são atributos clássicos, mas outros são possíveis, incluindo os de lúpulos modernos. Forte riqueza de malte neutro, grãos, pão, tostado ou caramelo claro; tipicamente sem aspectos de caramelo mais escuro, torrado ou frutas intensas. Ésteres e álcool de baixa a moderadamente forte, contribuindo menos para o equilíbrio do que o malte e o lúpulo. As intensidades diminuem com a idade.

\textbf{Sabor}: Descritores de sabores de malte e de lúpulo semelhantes aos do aroma (aplicam-se os mesmos descritores). Amargor moderadamente forte a agressivo, amenizado por um paladar rico e maltado. Sabor de lúpulo de moderado a alto. Ésteres de baixos a moderados. Álcool perceptível, mas não como solvente. Dulçor do malte moderadamente de baixo a moderadamente alto, com um final de um pouco maltado a seco, mas cheio. A idade geralmente seca a cerveja e suaviza os sabores. O equilíbrio é maltado, mas sempre é amarga.

\textbf{Sensação na Boca}: Encorpada e altamente viscosa, com uma textura aveludada e envolvente, diminuindo com a idade. Um aquecimento suave do álcool deve ser perceptível, mas não deve queimar. A carbonatação pode ser baixa a moderada, dependendo da idade e do acondicionamento.

\textbf{Comentários}: Às vezes rotulado como “Barley Wine” ou “Barleywine-style ale”. Recentemente, muitas cervejarias dos EUA parecem ter descontinuado suas Barleywines, transformando em cervejas envelhecidas em barril ou renomeando como alguma forma de IPA.

\textbf{História}: Tradicionalmente, a cerveja com teor alcoólico mais alto oferecida por uma cervejaria, muitas vezes associada à temporada de inverno e datada por safra. Tal como acontece com muitos estilos de cervejas artesanais americanas, uma adaptação de um estilo inglês, usando ingredientes e equilíbrio americanos. Uma das primeiras versões de cerveja artesanal americana foi a Anchor Old Foghorn, fabricada pela primeira vez em 1975. Sierra Nevada Bigfoot, fabricada pela primeira vez em 1983, estabeleceu o padrão orientado para o lúpulo, o padrão mais observado atualmente. A história diz que quando a Sierra Nevada enviou a Bigfoot pela primeira vez para análise de laboratório, o laboratório ligou e disse: “sua Barleywine está muito amarga” – ao que Sierra Nevada respondeu: “obrigado”.

\textbf{Ingredientes}: Malte pale com alguns maltes especiais. Maltes escuros usados com grande moderação. Muitas variedades de lúpulo podem ser usadas, mas normalmente inclui lúpulo americano. Levedura ale americana ou inglesa.

\textbf{Comparação de Estilos}: Maior ênfase no amargor, sabor e aroma do lúpulo do que o English Barley Wine, muitas vezes apresentando variedades de lúpulos americanos. Normalmente mais clara do que as English Barley Wines mais escuras e sem seus sabores de malte mais intensos, mas mais escura do que as English Barley Wines douradas. Diferencia-se de uma Double IPA, pois os lúpulos não são extremos, é mais orientada para o malte e o corpo é mais cheio e muitas vezes mais rico. A American Barleywine normalmente tem mais dulçor residual do que Double IPA, o que afeta a facilidade em beber (degustar x beber).

\begin{tabular}{@{}p{35mm}p{35mm}@{}}
  \textbf{Estatísticas} & OG: 1,080 - 1,120  \\
  IBU: 50 - 100  & FG: 1,016 - 1,030  \\
  SRM: 9 - 18 & ABV: 8\% - 12\%
\end{tabular}

\textbf{Exemplos Comerciais}: Anchor Old Foghorn, Bell's Third Coast Old Ale, East End Gratitude, Hair of the Dog Doggie Claws, Sierra Nevada Bigfoot.

\textbf{Revisões Anteriores}: American Barleywine (2015)

\textbf{Atributos de Estilo}: amber-color, bitter, craft-style, hoppy, north-america, strong-ale-family, top-fermented, very-high-strength
