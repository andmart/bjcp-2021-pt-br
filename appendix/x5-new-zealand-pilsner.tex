\phantomsection
\subsection*{X5. New Zealand Pilsner}
\addcontentsline{toc}{subsection}{X5. New Zealand Pilsner}

\textit{Estilo sugerido para inscrição: Categoria 12 (Pale Commonwealth Beer)}

\textbf{Impressão Geral}: Uma cerveja clara, seca, de cor dourada, fermentação limpa, apresentando as características tropicais, cítricas, frutadas e gramíneas dos lúpulos neozelandeses. Corpo médio, sensação na boca leve, com paladar e final suaves. A base de malte, de neutra à características que lembram pão, fornece o suporte para que esta cerveja seja fácil de beber, refrescante e tenha o lúpulo em destaque.

\textbf{Aroma}: Aroma de lúpulo de médio a alto, refletindo as variedades modernas de lúpulo do Novo Mundo, frequentemente apresentando caráter de frutas tropicais, frutas cítricas (limão, toranja branca), \textit{gooseberry}, melão, com um leve toque de pimentão verde ou aspecto gramíneo. Malte de médio-baixo a médio, para dar suporte, com perfil de neutro à pão ou biscoito água e sal. DMS muito baixo aceitável, mas não obrigatório. Caráter de levedura neutro e limpo, opcionalmente com um perfil sulfuroso muito leve. O caráter de lúpulo deve ser mais proeminente no equilíbrio, mas algum caráter de malte deve ser evidente.

\textbf{Aparência}: Cor de amarelo palha a dourado profundo, mas a maioria dos exemplares são amarelo-ouro. Geralmente bastante límpida a cristalina; turbidez é uma falha. Espuma branca, cremosa e duradoura.

\textbf{Sabor}: Amargor de lúpulo de médio a alto e limpo, não áspero, mais proeminente no equilíbrio e duradouro no retrogosto. Sabor de lúpulo de médio a alto, com características semelhantes ao aroma (tropical, cítrico, \textit{gooseberry}, melão, gramíneo). Sabor de malte de médio-baixo a médio, remetendo a dulçor de cereais, pão ou biscoito. Perfil de fermentação limpo (ésteres de fermentação são uma falha). Final de meio seco a seco, limpo, suave e amargo, mas não áspero, no retrogosto. O malte pode sugerir uma impressão de dulçor, mas a cerveja não deve ser literalmente doce. O final pode ser seco, mas não bem definido ou cortante. O equilíbrio deve ser sempre em direção ao amargor, mas o sabor do malte deve ser perceptível.

\textbf{Sensação na Boca}: Corpo de médio-baixo a médio. Carbonatação de média à média-alta. Suavidade é a impressão mais proeminente. Nunca áspera, nem adstringente.

\textbf{Comentários}: Os aromas do lúpulo geralmente têm uma qualidade semelhante à de muitos vinhos Sauvignon Blanc da Nova Zelândia, com aromas de frutas tropicais, gramíneas, melão e limão. Frequentemente produzida como um estilo híbrido na Nova Zelândia, usando uma levedura de cerveja neutra em temperaturas mais baixas. Limitar o teor de enxofre do produto acabado é importante, pois pode entrar em conflito com o caráter do lúpulo.

\textbf{História}: Definida em grande parte pela cerveja original, criada na cervejaria Emerson em meados da década de 1990, a New Zealand Pilsner expandiu em perfil à medida que as variedades de lúpulo da Nova Zelândia aumentaram em número e popularidade.

\textbf{Ingredientes}: Variedades de lúpulo da Nova Zelândia, como Motueka, Riwaka, Nelson Sauvin, muitas vezes com Pacific Jade para amargor. Outras variedades do Novo Mundo, da Austrália ou dos EUA, podem ser usadas, se tiverem características semelhantes. Malte base claro, Pilsner ou Pale, talvez com uma pequena porcentagem de malte de trigo. Água com baixo teor de minerais, geralmente com mais cloretos que sulfatos. Levedura lager limpa ou levedura ale muito neutra.

\textbf{Comparação de Estilos}: Comparada com uma German Pils, não é tão bem definida e seca no final, com uma presença mais suave e maltada e um corpo mais cheio. Em comparação com a Czech Premium Pale Lager, apresenta menor complexidade de malte e uma fermentação mais limpa. Similar em equilíbrio a uma Kölsch ou British Golden Ale, mas com um aroma mais lupulado. Comparada com qualquer um desses estilos alemães, apresenta variedades de lúpulo da Nova Zelândia com características tropicais, cítricas, frutadas e gramíneas, frequentemente com um caráter de vinho branco. Não deve ser tão lupulada ou amarga quanto um IPA.

\textbf{Estatísticas}:: OG: 1,044 – 1,056
FG: 1,009 – 1,014
IBU: 25 – 45
SRM: 2 – 6
ABV: 4,5 – 5,8%

\begin{tabular}{@{}p{35mm}p{35mm}@{}}
  \textbf{Estatísticas}: & OG: 1.044 – 1.056\\
  IBU: 25 - 45 & FG: 1.009 - 1.014 \\
  SRM: 2 - 6  & ABV: 4.5 - 5.8\%
\end{tabular}

\textbf{Exemplos Comerciais}: Croucher New Zealand Pilsner, Emerson’s Pilsner, Liberty Halo Pilsner, Panhead Port Road Pilsner, Sawmill Pilsner, Tuatara Mot Eureka

\textbf{Atributos do Estilo}: bitter, pale-color, standard-strength, bottom-fermented, hoppy, pilsner-family, lagered, craft-style, pacific
