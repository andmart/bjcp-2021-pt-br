\phantomsection
\subsection*{16D. Foreign Extra Stout}
\addcontentsline{toc}{subsection}{16D. Foreign Extra Stout}
\textbf{Impressão Geral}: Uma stout muito escura, rica, com teor alcoólico moderadamente alto, bastante seca, com sabores torrados proeminentes.

\textbf{Aparência}: Cor marrom muito profundo a preto. Limpidez geralmente obscurecida pela cor escura. Límpida, se não for opaca. Colarinho volumoso, castanho a marrom, com boa retenção.

\textbf{Aroma}: Torra moderada a alta, como café, chocolate amargo ou cereais levemente queimados. Frutado baixo a médio. Pode ter um aroma doce, melaço, alcaçuz, frutas secas ou vínico. Versões com teor alcoólico mais alto podem ter um aroma sutil e limpo de álcool. Aroma baixo de lúpulo terroso, herbal ou floral opcional. Diacetil baixo opcional.

\textbf{Sabor}: Torra moderada a alta, como café, chocolate amargo ou cereais levemente queimados, embora sem uma intensidade forte. Ésteres de baixo a médio. Amargor médio a alto. Final moderadamente seco. Sabor moderado de lúpulo terroso, herbal ou floral opcional. Diacetil médio-baixo opcional.

\textbf{Sensação na Boca}: Corpo médio-cheio a cheio, muitas vezes com um caráter suave, às vezes cremoso. Pode ter um aquecimento alcoólico, mas não pode queimar. Carbonatação moderada a moderadamente alta.

\textbf{Comentários}: Também conhecida como Foreign Stout, Export Stout e Foreign Export Stout. Versões históricas (antes da Primeira Guerra Mundial, pelo menos) tinham a mesmo OG que as Extra Stouts domésticas, mas dependendo da cervejaria poderia ter um ABV mais alto porque tinha uma fermentação secundária longa e com Brett. A diferença entre as versões nacionais e estrangeiras foram a lupulagem e o tempo de maturação.

\textbf{História}: Stouts com teor alcóolico mais alto fabricadas para o mercado de exportação hoje, mas com uma história que remonta aos séculos 18 e 19, quando eram versões mais lupuladas de stouts de exportação com teor alcoólico mais alto. Cerveja originalmente maturada em barricas de madeira*, mas a Guinness interrompeu essa prática na década de 1950. A Guinness Foreign Extra Stout (originalmente, West India Porter, mais tarde Foreign Extra Double Stout) foi fabricada pela primeira vez em 1801 de acordo com a Guinness com “lúpulo extra para dar um sabor distinto e uma vida útil mais longa em climas quentes”.

\textbf{Ingredientes}: Malte pale, maltes torrados escuros e cereais, historicamente também poderiam ter usado maltes brown e âmbar. Lúpulo principalmente para amargor, variedades tipicamente inglesas. Pode usar adjuntos e açúcar para aumentar a densidade.

\textbf{Comparação de Estilos}: Semelhante em equilíbrio a uma Irish Extra Stout, mas com mais álcool. Com teor alcoólico não tão alto nem tão intensa quanto uma Imperial Stout. Sem o forte amargor e os lúpulos de adição tardia das American Stout. densidade semelhante à Tropical Stout, mas com final mais seco e mais amargo.

\begin{tabular}{@{}p{35mm}p{35mm}@{}}
  \textbf{Estatísticas}: & OG: 1,056 - 1,075 \\
  IBU: 50 - 70  & FG: 1,010 - 1,018  \\
  SRM: 30 - 40   & ABV: 6,3\% - 8\%
\end{tabular}

\textbf{Exemplos Comerciais}: Coopers Best Extra Stout, Guinness Foreign Extra Stout, The Kernel Export Stout London 1890, La Cumbre Malpais Stout, Pelican Tsunami Export Stout, Ridgeway Foreign Export Stout, Southwark Old Stout.

\textbf{Última Revisão}: Foreign Extra Stout (2015)

\textbf{Atributos de Estilo}: balanced, british-isles, dark-color, high-strength, roasty, stout-family, top-fermented, traditional-style
*Versão original usa o termo Vatted
