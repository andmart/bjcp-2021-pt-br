\section*{29. Fruit Beer}
\addcontentsline{toc}{section}{29. Fruit Beer}
\textit{A categoria Fruit Beer é feita para cervejas produzidas com qualquer fruta ou combinações de frutas que estejam dentro de suas definições. Aqui, é utilizada a definição \textbf{culinária da fruta, não a botânica} - polpa, estruturas associadas à semente de plantas que são doces ou azedas e comestíveis no estado cru. Exemplos incluem frutas de pomar (maçã, pera, marmelo), frutas de caroço (cereja, ameixa, pêssego, damasco, manga etc), frutas vermelhas (qualquer fruta com a palavra 'berry' - em inglês - no nome), groselhas, frutas cítricas, frutas secas (tâmara, ameixa, uva passa etc), frutas tropicais (banana, abacaxi, manga, goiaba, maracujá, papaia etc), figo, romã, e assim por diante. Não estão abrangidos aqui especiarias, ervas e vegetais como definidos na Categoria 30 - especialmente frutos botânicos que são tratadas como vegetais culinários (legumes). Se você precisar justificar uma fruta utilizando a palavra "tecnicamente" como parte da descrição, então não é isso que nós queremos dizer.}\textit{Veja a introdução da seção Cervejas de Especialidade para comentários adicionais, particularmente sobre a avaliação do equilíbrio dos ingredientes adicionados à cerveja base.}
