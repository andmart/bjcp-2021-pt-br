\phantomsection
\subsection*{10C. Weizenbock}
\addcontentsline{toc}{subsection}{10C. Weizenbock}
\textbf{Impressão Geral}: Uma cerveja de trigo alemã forte e maltada, combinando os melhores sabores de trigo e levedura de uma Weissbier com a riqueza maltada, potência e corpo de uma Bock. As variações do estilo vão da potência de Bock até Doppelbock, com variações para cor clara e escura.

\textbf{Aparência}: Colarinho muito espesso, como mousse e duradouro. Pode ser turva e ter um brilho derivado do trigo e da levedura embora isso possa sedimentar com o tempo.

Versões escuras vão de cor âmbar escuro a marrom-rubi escuro, com colarinho castanho claro.
Versões claras vão de cor dourado a âmbar, com colarinho muito branco a quase branco
\textbf{Aroma}: Maltado rico de médio-alto a alto com caráter significativo de trigo como pão e cereais. Caráter de levedura weizen de médio-baixo a médio-alto, geralmente banana e cravo. Opcionalmente com notas de baunilha. Sem lúpulos. Álcool de baixo a moderado, nunca quente e/ou solvente. Malte, levedura e álcool são bem equilibrados, complexos e convidativos. Bubblegum/Tuttifrutti (morango com banana), acidez e defumado são falhas.

Versões escuras tem riqueza de malte profundo, altamente tostado e como pão, com presença significativa de produtos de Maillard, similar a uma Dunkles Bock ou uma Doppelbock escura. Elas também podem apresentar caramelo e ésteres lembrando frutas escuras como ameixa, ameixa seca, uvas escuras, couro de frutas e passas, principalmente com o envelhecimento.
Versões claras tem uma riqueza maltada com adocicado de cereais, pão e tostado, similar a uma Helles Bock ou uma Doppelbock clara.
\textbf{Sabor}: Maltado rico de médio-alto a alto com sabor significativo de trigo como pão e cereais. Caráter de levedura como banana e condimentado (cravo, baunilha) de baixo a moderado. Sem sabor de lúpulo. Amargor de baixo a médio-baixo pode dar uma impressão ligeiramente doce, mas a cerveja tipicamente tem um final seco. Uma leve nota alcoólica pode realçar esse caráter. A interação entre o malte, levedura e álcool adiciona complexidade e deixa a cerveja interessante, o que muitas vezes é aprimorado com o tempo. Bubblegum/Tuttifrutti, acidez e defumado são falhas.

Versões escuras tem sabores de malte mais profundos e ricos como pão ou tostado e provenientes da reação de Maillard, opcionalmente com caramelo ou chocolate leve, mas não torrado. Pode ter ésteres lembrando frutas escuras como ameixa, ameixa seca, uvas escuras, couro de fruta e/ou passas, principalmente conforme envelhecem.
Versões claras tem um maltado rico como pão, tostado e adocicado de cereais.
\textbf{Sensação na Boca}: Corpo de médio-alto a alto. Textura macia, suave, fofa ou cremosa. Leve aquecimento alcoólico. Carbonatação de moderada a alta.

\textbf{Comentários}: Uma Weissbier brassada com a potência de uma bock ou uma doppelbock, embora a Schneider também produza uma versão Eisbock. Existem versões claras e escuras, mas a versão escura é a mais comum. Produtos de Maillard levemente oxidados podem produzir alguns sabores e aromas ricos e intensos que muitas vezes são vistos em produtos comerciais importados envelhecidos, versões mais frescas não terão esse peril. Exemplares bem envelhecidos podem ter uma leve complexidade como xerez. Versões claras, assim como suas primas doppelbocks, vão ter menos complexidade rica de malte e muitas vezes ser mais inclinadas para o lúpulo. Entretanto, versões que tem uma carga de lúpulos tardios significativa ou que levam dry-hopping devem ser inscritas na categoria 34B Mixed-Style Beer.

\textbf{História}: A Aventinus com potência de uma Doppelbock foi criada em 1907 na Schneider Weisse Brauhaus em Munique. Versões claras são uma invenção muito mais recente

\textbf{Ingredientes}: Pelo menos metade da lista de grãos de malte de trigo. Malte Munich, Vienna e/ou Pilsner. Maltes para cor podem ser usados com moderação. Mostura por decocção é tradicional. Levedura Weizen e fermentação a temperatura fria.

\textbf{Comparação de Estilo}: Mais forte e mais rica que uma Weissbier ou uma Dunkles Weissbier, mas com perfil de levedura similar. Mais diretamente comparável com o estilo Doppelbock com variações claras e escuras. Pode variar muito em potência, mas a maioria está entre uma Bock e uma Doppelbock.

\textbf{Instruções para Inscrição}: O participante deve especificar se a amostra é uma versão clara (SRM entre 6-9) ou escura (SRM entre 10-25).

\begin{tabular}{@{}p{35mm}p{35mm}@{}}
  \textbf{Estatísticas}: & OG: 1,064 - 1,090 \\
  IBU: 15 - 30 & FG: 1,015 - 1,022 \\
  SRM: 6 - 25 & ABV: 6,5\% - 9\%
\end{tabular}

\textbf{Exemplos Comerciais}: Escura – Plank Bavarian Dunkler Weizenbock, Penn Weizenbock, Schalchner Weisser Bock, Schneider Weisse Aventinus; Clara – Ayinger Weizenbock, Distelhäuser Weizen Bock, Ladenburger Weizenbock Hell, Weihenstephaner Vitus

\textbf{Última Revisão}: Weizenbock (2015)

\textbf{Atributos de Estilo}: amber-color, central-europe, high-strength, malty, pale-color, top-fermented, traditional-style, wheat-beer-family