\phantomsection
\subsection*{15C. Irish Extra Stout}
\addcontentsline{toc}{subsection}{15C. Irish Extra Stout}
\textbf{Impressão geral}: Uma cerveja preta mais encorpada com um sabor torrado pronunciado, muitas vezes semelhante ao café e chocolate amargo com alguma complexidade maltada. O equilíbrio pode variar de moderadamente agridoce a amargo, com as versões mais equilibradas apresentando uma riqueza de malte moderada e as versões amargas sendo bastante secas.

\textbf{Aparência}: Muito preta. Opaca. Um colarinho espesso, cremoso e persistente é característico.

\textbf{Aroma}: Aroma de café moderado a moderadamente alto, muitas vezes com notas secundárias leves de chocolate amargo, cacau, biscoito, baunilha ou cereais torrados. Ésteres médio-baixo opcionais. Aroma de lúpulo baixo a nenhum, pode ser levemente terroso ou condimentado, mas normalmente está ausente. Malte e torra dominam o aroma.

\textbf{Sabor}: Sabor moderado a moderadamente alto de cereal escuro torrado ou malte com amargor médio a médio-alto. O final pode ser seco e como café a moderadamente equilibrado com caramelo moderado ou dulçor de malte. Normalmente tem sabores de café torrado, mas também muitas vezes tem um caráter de chocolate escuro no paladar, durando até o final. Os sabores de café mocha ou biscoito, em segundo plano, geralmente estão presentes e adicionam complexidade. Frutado médio-baixo opcional. Sabor médio terroso ou condimentado de lúpulo opcional. O nível de amargor é um pouco variável, assim como o caráter torrado e a secura do final; permitindo a interpretação pelos cervejeiros.

\textbf{Sensação na Boca}: Corpo médio-cheio a cheio, com um caráter um tanto cremoso. Carbonatação moderada. Muito suave. Pode ter uma leve adstringência dos cereais torrados, embora a aspereza seja indesejável. Pode ser detectado um leve aquecimento.

\textbf{Comentários}: Tradicionalmente, um produto engarrafado
com teor alcoólico mais alto com uma gama de interpretações
possíveis igualmente válidas, variando mais frequentemente
no sabor torrado e dulçor. Os exemplos comerciais irlandeses
mais tradicionais estão na faixa de 5,6 a 6,0\% de ABV.

\textbf{História}: Mesmas raízes que a Irish Stout, mas como um teor alcoólico mais alto. Guinness Extra Stout (Extra Superior Porter, mais tarde Double Stout) foi fabricada pela primeira vez em 1821, e era principalmente um produto engarrafado.

\textbf{Ingredientes}: Semelhante a Irish Stout. Adicionalmente, pode ter malte crystal escuro ou açúcares escuros.

\textbf{Comparação de Estilos}: No meio do caminho, entre uma Irish Stout e uma Foreign Extra Stout em força e intensidade de sabor, embora com equilíbrio semelhante. Mais corpo, riqueza e muitas vezes complexidade de malte do que uma Irish Stout. Preto na cor, não marrom como uma English Porter.

\begin{tabular}{@{}p{35mm}p{35mm}@{}}
  \textbf{Estatísticas}: & OG: 1,052 - 1,062 \\
  IBU: 35 - 50  & FG: 1,010 - 1,014 \\
  SRM: 30 - 40  & ABV: 5\% - 6,5\%
\end{tabular}

\textbf{Exemplos Comerciais}: Guinness Extra Stout, O'Hara's Leann Folláin, Porterhouse XXXX, Sheaf Stout.

\textbf{Última Revisão}: Irish Extra Stout (2015)

\textbf{Atributos de Estilo}: bitter, british-isles, dark-color, high-strength, roasty, stout-family, top-fermented, traditional-style
