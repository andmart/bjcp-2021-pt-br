\phantomsection
\subsection*{1A. American Light Lager}
\addcontentsline{toc}{subsection}{1A. American Light Lager}
\textbf{Impressões Gerais}: Altamente carbonatada e de corpo muito baixo, uma lager quase sem sabor, criada para ser consumida bem gelada. Muito refrescante e para matar a sede.

\textbf{Aroma}: Aroma de malte de baixo a ausente, pode ser percebido como remetendo à cereais, como milho, adocicado. Aroma de lúpulo de caráter picante ou floral é opcional. Apesar de ser desejável um caráter limpo de fermentação, um leve perfil de levedura não é uma falha.

\textbf{Aparência}: Cor de palha a amarelo claro. Colarinho branco, não muito persistente. Límpida.

\textbf{Sabor}: Relativamente neutra no palato, com um final fresco e seco. Sabor de cereais ou milho de muito baixo a baixo, que pode ser percebido como dulçor, devido ao baixo amargor. Sabor de lúpulo de baixo a ausente, podendo ter perfil floral, picante ou herbal, embora seja raramente forte o suficiente para ser detectado. Amargor de lúpulo de baixo a muito baixo. Equilíbrio pode variar de ligeiramente maltado a ligeiramente amargo, mas é comumente equilibrada. A alta carbonatação pode realçar a sensação de frescor e o final seco. Caráter limpo de fermentação lager.

\textbf{Sensação na Boca}: Corpo muito leve, às vezes aguado. Carbonatação muito alta, gera sensação de carbonatação picante na língua.

\textbf{Comentários}: Desenvolvida para atrair o maior público possível. Sabores fortes significam falha na cerveja. Com pouco sabor de malte ou lúpulo, o caráter de levedura muitas vezes é o que diferencia as diversas marcas.

\textbf{História}: A Coors produziu uma Light Lager por alguns anos, na década de 1940. Versões modernas foram produzidas inicialmente por Rheingold, em 1967, para atender os consumidores que faziam dieta, mas, somente em 1973 se tornou popular, após a cervejaria Miller adquirir a receita e fazer uma grande ação de \textit{marketing} entre os praticantes de esportes com o slogan "tastes great, less filling" \textit{(em português, algo como "mais sabor, menos calorias")}. As cervejas deste estilo se tornaram as mais vendidas nos EUA na década de 1990.

\textbf{Ingredientes}: Cevada de duas ou seis fileiras com até 40\% de adjuntos (arroz ou milho). Enzimas adicionais podem ser utilizadas para reduzir o corpo e a quantidade de carboidratos. Levedura Lager. Pouco uso de lúpulos.

\textbf{Comparação de Estilos}: Uma versão com menor corpo, menos álcool e menos calorias que uma American Lager. Menos caráter de lúpulo e amargor do que na German Leitchbier.

\begin{tabular}{@{}p{35mm}p{35mm}@{}}
  \textbf{Estatísticas}: & OG: 1,028 - 1,040 \\
  IBU: 8 - 12  & FG: 0,998 - 1,008 \\
  SRM: 2 - 3  & ABV: 2,8\% - 4,2\%
\end{tabular}

\textbf{Exemplos Comerciais}: Bud Light, Coors Light, Grain Belt Premium Light American Lager, Michelob Light, Miller Lite, Old Milwaukee Light.

\textbf{Última Revisão}: American Light Lager (2015)

\textbf{Atributos de Estilo}: balanced, bottom-fermented, lagered, north-america, pale-color, pale-lager-family, session-strength, traditional-style.