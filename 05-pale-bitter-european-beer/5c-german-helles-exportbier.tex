\phantomsection
\subsection*{5C. German Helles Exportbier}
\addcontentsline{toc}{subsection}{5C. German Helles Exportbier}
\textbf{Impressão Geral}: Uma lager alemã dourada, que equilibra perfil de malte suave e caráter amargo e lupulado em uma cerveja de teor alcoólico e corpo levemente acima da média. \\
\textbf{Aparência}: De amarelo médio a dourado profundo. Límpida. Colarinho branco persistente. \\
\textbf{Aroma}: Aroma de lúpulo floral, condimentado e/ou herbal de médio-baixo a médio. Aroma de malte lembrando adocicado de cereais de intensidade moderada, possivelmente com notas leves de tostado, pão ou massa de pão. Perfil de fermentação limpo. Lúpulo e malte são perceptíveis e, geralmente, em equilíbrio. \\
\textbf{Sabor}: Malte e lúpulo em intensidade moderada e equilibrados, amargor dando suporte. Sabores de malte e lúpulo similar ao aroma (com os mesmos descritores e intensidades). Amargor perceptível de média intensidade, cheia na boca, com final médio-seco. Perfil limpo de fermentação. Retrogosto com malte e lúpulos geralmente em equilíbrio. Caráter mineral é normalmente percebido mais como um arredondamento, aumento de percepção de sabor e uma secura nítida no final, do que como sabor mineral evidente. Sulfato de fundo é opcional. \\
\textbf{Sensação na Boca}: Corpo de médio a médio-cheio. Carbonatação média. Macia e suave na boca. Aquecimento alcoólico muito leve pode ser percebido em versões mais fortes. \\
\textbf{Comentários}: Também conhecida como Dortmunder Export, Dortmunder, Export ou simplesmente Dort. Chamada de Export dentro da Alemanha e em outros lugares, frequentemente, de Dormunder. A palavra Export é também um descritor da potencia da cerveja, segundo a tradição cervejeira alemã e também pode ser aplicada a outros estilos. Fica entre uma German Pils e uma Munich Helles em vários aspectos: cor, equilibro entre lúpulo e malte, final e amargor. \\
\textbf{História}: Desenvolvida em Dortmund, na região industrial de Ruhr, em meados de 1870, como resposta as cervejas claras tipo Pilsner. Tornou-se muito popular depois da Segunda Guerra Mundial, mas perdeu popularidade em meados de 1970. Outras cervejas para exportação se desenvolveram independentemente e refletiam versões levemente mais fortes de cervejas existentes. \\
\textbf{Ingredientes}: Água mineralizada com altos níveis de sulfatos, carbonatos e cloretos. Lúpulos tradicionais alemães ou tchecos. Malte Pilsner. Levedura lager alemã. \\
\textbf{Comparação de Estilo}: Menos lúpulos de finalização e mais corpo do que uma German Pils. Mais amarga e seca do que uma Munich Helles. Mais forte, seca e menos lupulada do que uma Czech Premium Pale Lager. \\
\begin{tabular}{@{}p{35mm}p{35mm}@{}}
  \textbf{Estatísticas}: & OG: 1,050 - 1,058 \\
  IBU: 20 - 30  & FG: 1,008 - 1,015  \\
  SRM: 4 - 6  & ABV: 5\% - 6\%
\end{tabular}\\
\textbf{Exemplos Comerciais}: Chuckanut Export Dortmunder Lager, DAB Dortmunder Export, Dortmunder Kronen, Landshuter Edel Hell, Müllerbräu Export Gold, Schönramer Gold. \\
\textbf{Última Revisão}: German Helles Exportbier (2015) \\
\textbf{Atributos de Estilo}: balanced, bottom-fermented, central-europe, lagered, pale-color, pale-lager-family, standard-strength, traditional-style
