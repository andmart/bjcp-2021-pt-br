\phantomsection
\subsection*{27A. Historical Beer: Piwo Grodziskie}
\addcontentsline{toc}{subsection}{27A. Historical Beer: Piwo Grodziskie}
\textbf{Impressão Geral}: Uma cerveja de trigo histórica da região central europeia com densidade baixa, amarga e defumada por carvalho, com um perfil de fermentação limpo e sem acidez. Altamente carbonatada, seca, com final bem definido e refrescante.

\textbf{Aroma}: Defumado por madeira de carvalho de baixo a moderado é o componente de aroma mais proeminente, mas pode ser sutil e difícil detectar. Aroma de lúpulo condimentado, de ervas ou floral de baixa intensidade está tipicamente presente e deve ser em intensidade menor ou igual ao defumado. Notas de cereais comotrigo também são detectados nos melhores exemplares. Fora essas características, o aroma é limpo, embora esteres leves de frutas de pomar (especialmente maçã vermelha madura ou pera) são bem vindos. Sem acidez. Leve enxofre é opcional.

\textbf{Aparência}: Cor de amarelo palha a dourado, com excelente limpidez. Um colarinho alto, largo, bem coeso, com excelente retenção é característico. Turbidez alta é uma falha.

\textbf{Sabor}: Sabor defumado por carvalho de moderadamente baixo a médio em primeiro plano e se mantém até o final; o defumado pode ser mais forte no sabor do que no aroma. O caráter de defumado é delicado e não deve ser acre e pode levar a impressão de dulçor. Amargor de moderado a forte é evidente e permanece até o final. O equilíbrio geral é para o lúpulo. Sabor de lúpulo condimentado, de ervas ou floral baixo, mas perceptível. Caráter de cereais como trigo de intensidade baixa como uma nota de fundo. Esteres leves como frutas de pomar (maçã vermelha ou pera) pode estar presente. Final seco e bem definido. Sem acidez.

\textbf{Sensação na Boca}: Corpo leve, com um final bem definido e seco. Carbonatação é bem alta e pode adicionar uma leve picância carbônica ou uma sensação de formigamento. Sem aquecimento alcoólico.

\textbf{Comentários}: Pronunciada em inglês 'pivo grow-JEES-kee-uh' (significa: Cerveja de Grodzisk). Conhecida como \textit{Grätzer} (pronunciado 'GRATE-sir') em países de língua alemã e em alguma literatura de cerveja. Tradicionalmente feita por uma brassagem por rampas, uma longa fervura (~2 horas) e múltiplas cepas de levedura ale. A cerveja nunca é filtrada, mas é usado Isinglass para clarificar a cerveja antes da refermentação na garrada. Servida tradicionalmente em copos cônicos altos para acomodar a vigorosa espuma gerada.

\textbf{História}: Desenvolvida séculos atrás como um estilo único na cidade polonesa de Grodzizk (conhecida como Grätz quando governada pela Prússia e Alemanha). Sua fama e popularidade se estendeu rapidamente para outras partes do mundo no fim do século 19 e início do século 20. A produção comercial regular diminui depois da Segunda Guerra Mundial e acabou nos anos 90. Essa descrição de estilo descreve a versão tradicional durante o seu período de maior popularidade.

\textbf{Ingredientes}: Malte de trigo defumado em carvalho, que tem um caráter menos intenso de desumado do que o Rauchmalz alemão e uma qualidade mais seca e \textit{crisp}. Sabor defumado como de bacon ou presunto é inapropriado. Lúpulos tradicionais poloneses, tchecos ou alemães. Água com dureza moderada de sulfatos. Levedura ale limpa e atenuante. Levedura Weizen é inapropriada.

\textbf{Comparação de Estilo}: Similar em força a uma Berliner Weisse, mas nunca deve ser ácida e é muito mais amarga. Tem um caráter defumado mas com intensidade mais baixa do que em uma Rauchbier. Densidade mais baixa que uma Lichtenhainer mas mais amarga e não ácida. Mais amarga que uma Gose mas sem sal e condimentos.

\begin{tabular}{@{}p{35mm}p{35mm}@{}}
  \textbf{Estatísticas}: & OG: 1,028 - 1,032 \\
  IBU: 20 - 35  & 1,006 - 1,012  \\
  SRM: 3 - 6 & ABV: 2,5\% - 3,3\%
\end{tabular}

\textbf{Exemplos Comerciais}: Live Oak Grodziskie.

\textbf{Última Revisão}: Historical Beer: Piwo Grodziskie (2015)

\textbf{Atributos de Estilo}: bitter, central-europe, historical-style, pale-color, session-strength, smoke, top-fermented, wheat-beer-family
