\phantomsection
\subsection*{17C. Wee Heavy}
\addcontentsline{toc}{subsection}{17C. Wee Heavy}
\textbf{Impressão Geral}: Malte rico e doce, com profundidade de caramelo, \textit{toffee} e sabores frutados. Corpo cheio e mastigável, com aquecimento alcoólico. Amargor contido, mas não enjoativa ou xaroposa.

\textbf{Aparência}: Cor de cobre claro a marrom escuro, muitas vezes com reflexos rubis profundos. Límpida. Geralmente tem um volumoso colarinho castanho, que pode não persistir. Lágrimas podem ser evidentes em versões com teor alcoólico mais alto.

\textbf{Aroma}: Malte tostado forte, com alto aspecto de caramelo e \textit{toffee}. Uma ampla gama de aromas de açúcar caramelizado e pão torrado como suporte são possíveis (migalhas de pão tostadas, biscoitos \textit{champagne}, biscoitos ingleses, bolachas, torrone, caramelo etc). Às vezes, nota-se um leve toque de torra. Ésteres de frutas escuras ou secas e álcool de baixo a moderado. Lúpulos terrosos, florais, cítricos alaranjarados ou picantes muito baixos são opcionais.

\textbf{Sabor}: Malte rico e tostado, muitas vezes cheio e doce no paladar com sabores de caramelo e \textit{toffee}, mas equilibrado pelo álcool e uma pitada de grãos torrados no final. O malte geralmente tem açúcar caramelizado e sabores tostados do mesmo tipo descrito no aroma. Álcool e ésteres de baixos a médios (ameixas, passas, frutos secos etc). Amargor baixo no equilíbrio, dando um final de meio seco a doce. Sabor de lúpulo médio baixo é opcional, com descritores semelhantes ao aroma.

\textbf{Sensação na Boca}: De medio alto a alto, às vezes com uma viscosidade espessa, mastigável, às vezes cremosa. Um aquecimento alcoólico suave, geralmente está presente e é desejável, pois equilibra a doçura do malte. Carbonatação moderada.

\textbf{Comentários}: Uma ampla gama em relação a força é permitida; nem todas as versões tem teor alcoólico muito alto. Também conhecida como “Strong Scotch Ale”, o termo \textit{wee heavy} significa “forte pequena” e remonta à cerveja que tornou o termo famoso, a Fowler’s Wee Heavy, uma 12 Guinea Ale (Moeda antiga da Inglaterra).

\textbf{História}: Descendente das Edinburgh Ales, uma cerveja maltada com maior teor alcoólico fabricada numa variedade em relação à força alcoólica, semelhante à Burton Ale (embora com metade da proporção de lúpulo). As versões modernas têm duas variantes principais, uma cerveja mais modesta com 5\% de ABV e a mais conhecida cerveja com 8-9\% de ABV. Como as densidades diminuíram ao longo do tempo, algumas das variações deixaram de ser produzidas.

\textbf{Ingredientes}: Malte escocês pale ale, uma ampla gama de outros ingredientes é possível, incluindo adjuntos. Alguns podem usar malte crystal ou grãos mais escuros para conferir cor. Sem malte defumado com turfa.

\textbf{Comparação de Estilos}: Um pouco semelhante a uma English Barley Wine, mas muitas vezes mais escura e mais caramelizada.

\begin{tabular}{@{}p{35mm}p{35mm}@{}}
  \textbf{Estatísticas}: & OG: 1,070 - 1,130 \\
  IBU: 17 - 35  & FG: 1,018 - 1,040  \\
  SRM: 14 - 25  & ABV: 6,5\% - 10\%
\end{tabular}

\textbf{Exemplos Comerciais}: Belhaven Wee Heavy, Broughton Old Jock, Gordon Highland Scotch Ale, Inveralmond Blackfriar, McEwan's Scotch Ale, Orkney Skull Splitter, Traquair House Ale, The Duck-Rabbit Wee Heavy Scotch-Style Ale.

\textbf{Última Revisão}: Wee Heavy (2015)

\textbf{Atributos de Estilo}: amber-color, british-isles, high-strength, malty, strong-ale-family, top-fermented, traditional-style
