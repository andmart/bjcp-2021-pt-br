\phantomsection
\subsection*{27A. Historical Beer: Pre-Prohibition Porter}
\addcontentsline{toc}{subsection}{27A. Historical Beer: Pre-Prohibition Porter}
\textbf{Impressão Geral}: Uma adaptação americana histórica da English Porter por imigrantes alemães usando ingredientes americanos, incluindo adjuntos.

\textbf{Aparência}: Cor de marrom médio a escuro, embora alguns exemplares possam ser quase pretos, com reflexos rubi ou de cor mogno. Relativamente clara. Espuma de cor castanha, de baixa a média formação, persistente.

\textbf{Aroma}: Aroma de malte como cereais, com baixos níveis de chocolate, caramelo, biscoito, açúcar queimado, alcaçuz ou malte levemente queimado. Baixo aroma de lúpulo. Nível de baixo a moderadamente baixo de milho ou DMS é aceitável. Ésteres de muito baixos a nenhum. Diacetil de baixo a nenhum. Perfil de fermentação lager limpo é aceitável.

\textbf{Sabor}: Sabor moderado de malte que remete a de cereais e de pão, com baixos níveis de chocolate, malte queimado, açúcar queimado, caramelo, biscoito, alcaçuz, melaço ou tosta. Sabor de milho ou DMS é aceitável em níveis baixos a moderados. Amargor de médio baixo a moderado. Baixo sabor de lúpulo floral, condimentado ou terroso é opcional. O equilíbrio é tipicamente nivelado entre o malte e o lúpulo, com um final moderadamente seco. Perfil de fermentação limpo, entretanto, ésteres leves são permitidos.

\textbf{Sensação na Boca}: Corpo de médio baixo a médio. Carbonatação moderada. Cremosidade de baixa a moderada. Pode ter uma leve adstringência de malte escuro.

\textbf{Comentários}: Também às vezes conhecida como Pennsylvania Porter ou East Coast Porter. Este estilo não descreve produtos da era colonial.

\textbf{História}: Produzida comercialmente na Filadélfia durante o período revolucionário como uma adaptação da cerveja inglesa. Evoluiu mais tarde, quando os imigrantes alemães aplicaram métodos de fabricação de cerveja lager durante a segunda metade dos anos 1800. A Lei Seca acabou com a maioria da produção de cerveja Porter nos Estados Unidos, exceto em alguns estados regionais do Nordeste e do Meio-Atlântico, onde ela era mais popular.

\textbf{Ingredientes}: Malte de duas ou seis fileiras. Baixas porcentagens de maltes escuros, incluindo malte black, chocolate e brown (cevada torrada normalmente não é usada). Adjuntos são aceitáveis, incluindo milho, alcaçuz cervejeiro, melaço e porterine. Versões mais históricas terão até vinte por cento de adjuntos. Levedura lager ou ale. Lúpulos de amargor americanos históricos ou tradicionais, lúpulos americanos ou alemães para final de fervura.

\textbf{Comparação de Estilos}: Mais suave e com menos amargor de lúpulo do que uma American Porter (moderna). Menos caráter de caramelo e mais suave do que uma English Porter, com mais caráter de adjunto ou de uma cerveja lager. Mais amargor e torrado do que uma International Dark Lager.

\begin{tabular}{@{}p{35mm}p{35mm}@{}}
  \textbf{Estatísticas}: & OG: 1,046 - 1,060  \\
  IBU: 20 - 30  & FG: 1,010 - 1,016  \\
  SRM: 20 - 30 & ABV: 4,5\% - 6\%
\end{tabular}

\textbf{Exemplos Comerciais}: Stegmaier Porter, Yuengling Porter.

\textbf{Última Revisão}: Historical Beer: Pre-Prohibition Porter (2015)

\textbf{Atributos de Estilo}: any-fermentation, dark-color, historical-style, malty, north-america, porter-family, standard-strength
