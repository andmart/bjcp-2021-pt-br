\phantomsection
\subsection*{19A. American Amber Ale}
\addcontentsline{toc}{subsection}{19A. American Amber Ale}
\textbf{Impressão Geral}: Uma cerveja artesanal americana âmbar, lupulada e de força moderada, com um sabor de caramelo maltado. O equilíbrio pode variar um pouco, com algumas versões sendo bastante maltadas e outras agressivamente lupuladas. As versões lupuladas e amargas não devem ter sabores conflitantes com o perfil do malte caramelo.

\textbf{Aparência}: Cor de âmbar profundo a marrom acobreado, às vezes com um tom avermelhado. Colarinho quase branco e moderadamente volumoso, com boa retenção. Geralmente bastante límpida.

\textbf{Aroma}: Aroma de lúpulo de baixo a moderado, refletindo variedades de lúpulos americanos ou do Novo Mundo (cítrico, floral, de pinho, resinoso, condimentado, de frutas tropicais, de frutas de caroço, de frutas vermelhas ou de melão). Um caráter cítrico de lúpulo é comum, mas não é obrigatório. Malte de moderadamente baixo a moderadamente alto, geralmente com um caráter moderado de caramelo, que pode dar suporte, equilibrar ou às vezes mascarar a apresentação do lúpulo. Os ésteres variam de nenhum a moderados.

\textbf{Sabor}: Sabor de lúpulo de moderado a alto com características semelhantes ao aroma. Os sabores de malte são de moderados a fortes e geralmente mostram um dulçor maltado inicial seguido de um sabor moderado de caramelo e, às vezes, sabores de malte tostado ou biscoito em quantidades menores. Sabores de maltes escuros ou torrados ausentes. Amargor de moderado a moderadamente alto. O equilíbrio pode variar de um pouco maltado a um pouco amargo. Ésteres frutados podem ser de moderados a nenhum. O dulçor do caramelo, o sabor do lúpulo e o amargor podem permanecer um pouco no final, com intensidade de média a alta, embora ainda seja seco.

\textbf{Sensação na Boca}: Corpo de médio a médio cheio. Carbonatação de média a alta. Acabamento geral macio e sem adstringência. Versões com teor alcoólico mais alto podem ter um leve aquecimento alcoólico.

\textbf{Comentários}: Pode se sobrepor na cor com as American Pale Ales mais escuras, mas com um sabor e equilíbrio de malte diferente. Existe uma gama de equilíbrio neste estilo, desde equilibrado e maltado, até mais agressivamente lupulado.

\textbf{História}: Um estilo moderno de cerveja artesanal americana desenvolvido como uma variação das American Pale Ales. A Mendocino Red Tail Ale foi produzida pela primeira vez em 1983 e era conhecida regionalmente como Red Ale. Ela serviu como progenitora das Double Reds (American Strong Ale), Red IPAs e outras cervejas lupuladas e caramelizadas.

\textbf{Ingredientes}: Malte Pale Ale neutro. Maltes crystal médio a escuro. Lúpulos americanos ou do Novo Mundo são comuns, muitas vezes com sabores cítricos, mas outros também podem ser usados. Levedura de neutra a que produz leve esterificação.

\textbf{Comparação de Estilos}: Mais escura, mais caramelizada, mais encorpada e geralmente menos amarga no equilíbrio do que as American Pale Ales. Menos álcool, amargor e caráter de lúpulo do que as Red IPAs. Menor teor alcóolico, malte e caráter de lúpulo do que American Strong Ales. Menos chocolate e caramelo escuro do que uma American Brown Ale.

\begin{tabular}{@{}p{35mm}p{35mm}@{}}
  \textbf{Estatísticas}: & OG: 1,045 - 1,060 \\
  IBU: 25 - 40  & FG: 1,010 - 1,015  \\
  SRM: 10 - 17  & ABV: 4,5\% - 6,2\%
\end{tabular}

\textbf{Exemplos Comerciais}: Anderson Valley Boont Amber Ale, Bell’s Amber Ale, Full Sail Amber, North Coast Red Seal Ale, Saint Arnold Amber Ale, Tröegs Hopback Amber Ale.

\textbf{Última Revisão}: American Amber Ale (2015)

\textbf{Atributos de Estilo}: amber-ale-family, amber-color, balanced, craft-style, hoppy, north-america, standard-strength, top-fermented
