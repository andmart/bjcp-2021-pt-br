\clearpage
\phantomsection
\divisorLine
\section*{Introdução às Cervejas de Especialidade}
\addcontentsline{toc}{section}{Introdução às Cervejas de Especialidade }
\textit{\textbf{Cerveja de especialidade} é um termo amplo que se refere às categorias de estilo 28 à 34 e são contrastantes com as cervejas de Estilos Clássicos, das categorias 1 até 27. Os estilos clássicos possuem descrições completas e singulares, mas os estilos de especialidade compreendem uma transformação de um Estilo Base usando ou um processo ou por adição de um ou mais \textbf{Ingredientes especiais} (levedura/bactéria, defumação, madeira, fruta, especiaria, cereais ou açúcar). Descrições de estilos de cervejas de especialidade normalmente descrevem como os ingredientes ou processos especiais modificam o estilo base.}\\
\textit{Quando o \textbf{Estilo Base} para as cervejas de especialidade for necessário, um Estilo Clássico ou uma família de estilos amplas (por exemplo, IPA, Blond Ale, Stout) pode ser usado. De maneira geral, cervejas de especialidade não podem ser usadas como um Estilo Base para outras cervejas de especialidade a não ser que as instruções de inscrição para o estilo permitam expressamente. Alguns estilos de cerveja de especialidade não necessitam de um Estilo Base declarado – leia cuidadosamente a sessão de Instruções de Inscrição de cada estilo.}\\
\textit{Consulte a página dos Estilos Provisórios no website do BJCP para adições na lista de estilos. Esses Estilos Provisórios podem ser citados como o Estilo Base ao inscrever uma cerveja de especialidade. A página de Sugestões de Inscrições de Estilos do website do BJCP esclarece onde inscrever alguns estilos atualmente sem definições.}\\
\textit{Normalmente ao inscrever uma Cerveja de Especialidade que utiliza algum ingrediente que envolve comida, utilize a definição culinária deste ingrediente e não a botânica. Veja o prefácio de cada categoria de estilo para uma lista detalhada de ingredientes comuns. }\\
\textit{Presume-se que esta introdução incorporada à todas as descrições de estilos das cervejas de especialidade. Ela descreve como inscrever uma Cerveja de Especialidade de modo geral. Instruções específicas par a inscrição de cada estilo estão contidas nas descrições de estilos individuais.}\\
\begin{multicols}{2}
\import{./}{inscrevendo-cervejas-de-especialidade.tex}
\import{./}{julgando-cervejas-de-especialidade.tex}
\end{multicols}