\phantomsection
\subsection*{28D. Straight Sour Beer}
\addcontentsline{toc}{subsection}{28D. Straight Sour Beer}
\textit{Destina-se a cervejas fermentadas com Sacch e Lacto, com ou sem envelhecimento em carvalho, produzidas com qualquer técnica (ex.:co-fermentação tradicional, acidificação rápida de kettle sour).}\\
\textbf{Impressão Geral}: Uma cerveja azeda, clara e refrescante com um acidez lático limpo. Suave sabor de malte claro de apoio para a acidez, como limão e com moderados ésteres frutados.

\textbf{Aroma}: Um caráter de acidez acentuada é dominante (de moderadamente alto a alto). Pode ter um caráter frutado moderado (muitas vezes pêssego, damasco, limão ou maçã ácida). Sem aroma de lúpulo. Maltado claro de apoio, normalmente abiscoitado. Fermentação limpa

\textbf{Aparência}: Cor muito clara. Limpidez varia de muito límpida até um pouco turva. Colarinho branco, denso e alto, com baixa retenção. Efervescente.

\textbf{Sabor}: Acidez lático limpo dominante e pode ser bastante forte. Algum sabor complementar de pão, biscoito ou como cereais é geralmente notável. Amargor de lúpulo é indetectável. Nunca avinagrado ou com uma acidez picante. Caráter de fruta branca pode ser moderado, incluindo cítrico como limão ou maçã ácida que é geralmente notável. Final é quase seco a seco. O equilíbrio é dominado pelo acidez, porém alguns sabores de malte e ésteres frutados devem estar presentes. Sem sabor de lúpulo. Limpa.

\textbf{Sensação na Boca}: Corpo leve. Carbonatação moderada a alta. Nunca quente, embora exemplos de densidade inicial maior podem ter um caráter de aquecimento alcoólico. Acidez bem definida.

\textbf{Comentários}: Uma versão mais forte de cervejas como Berliner Weisse com uma combinação de grãos menos restrita, e sem Brett. Esse estilo de cerveja é tipicamente utilizado como base para cervejas modernas que são altamente saborizadas com frutas, especiarias, açúcares, etc., essas devem ser inscritas em 28C Wild Specialty Beer.

\textbf{História}: O cientista cervejeiro alemão, Otto Francke, desenvolveu o que se tornou conhecido como processo de acidificação Francke, que possibilitou o tradicional processo de fermentação mista da Berliner Weisse se tornar mais rápido e consistente; também é conhecido como \textit{kettle sour} ou acidificação na panela. Muitas cervejas azedas modernas utilizam esse método para produção rápida e é uma alternativa para os complexos processos de produção em barris.

\textbf{Ingredientes}: A maior parte, ou toda combinação de grãos, é de maltes Pale, Pilsen ou maltes de trigo em qualquer combinação. Maltes levemente tostados podem ser utilizados para maior profundidade maltada. Maltes como Carapils podem ser utilizados para o corpo. Açúcar claro pode ser utilizado para aumentar a densidade inicial sem aumentar o corpo. Sem lactose ou maltodextrina. Pode ser produzida por meio de kettle souring, culturas de cofermentação (levedura ou bactéria produtora de ácido lático), ou utilizando levedura especial para produção de ácido lático. Sem Brett.

\textbf{Comparação de Estilos}: Exemplos de densidade mais baixa podem parecer muito com uma Berliner Weisse sem Brett. Comparados a Lambic, geralmente não são tão ácidos e com um acidez de fermentação lática limpo, com \textit{funk} restrito abaixo do limiar de percepção. Maior teor alcoólico do que ambas.

\begin{tabular}{@{}p{35mm}p{35mm}@{}}
  \textbf{Estatísticas}: & OG: 1,048 - 1,065 \\
  IBU: 3 - 8  & FG: 1,006 - 1,013 \\
  SRM: 2 - 3  & ABV: 4,5\% - 7\%
\end{tabular}

\textbf{Exemplos Comerciais}: Raramente são encontrados, esse estilo é tipicamente utilizado como base para outros estilos especiais (Specialty-Type Beers).

\textbf{Atributos de Estilo}: pale-color, sour, top-fermented
