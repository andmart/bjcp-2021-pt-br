\phantomsection
\subsection*{22D. Wheatwine}
\addcontentsline{toc}{subsection}{22D. Wheatwine}

\textbf{Impressão Geral}: Uma cerveja com uma textura rica, de alto teor alcoólico, com um sabor significativo de pão, cereais, e um corpo elegante. A ênfase é primeiramente nos sabores de pão e trigo, juntamente com malte, lúpulo, levedura frutada e complexidade alcoólica.

\textbf{Aparência}: Cor variando do ouro ao âmbar profundo, muitas vezes com reflexos granada ou rubi. Colarinho quase branco, de baixo a médio. A espuma pode ter textura cremosa e boa retenção. *Chill haze* é permitido, mas geralmente desaparece à medida que a cerveja fica mais quente. O alto teor alcoólico e a viscosidade podem ser visíveis como lágrimas (pernas).

\textbf{Aroma}: O aroma de lúpulo é suave e pode representar praticamente qualquer variedade de lúpulo. Caráter de malte de trigo, de moderado a moderadamente forte, muitas vezes com uma complexidade adicional de malte, como mel e caramelo. Pode se notar um leve aroma de álcool limpo. Notas frutadas de baixas a médias podem ser aparentes. Diacetil muito baixo é opcional. O caráter de levedura Weizen, como banana e cravo, é inadequado.

\textbf{Sabor}: Sabor de malte de trigo de moderado a moderadamente alto, dominante no equilíbrio sobre qualquer caráter de lúpulo. Notas de malte tostado, caramelo, biscoito ou mel, de baixo a moderado, podem adicionar uma complexidade que é bem-vinda, mas não são obrigatórias. Sabor de lúpulo de baixo a médio, refletindo qualquer variedade. Frutado de moderado a moderadamente alto, muitas vezes com um caráter de frutas secas. Amargor de baixo a moderado, criando um equilíbrio com o malte. Não deve ser xaroposa ou sub-atenuada.

\textbf{Sensação na Boca}: Corpo de médio cheio a cheio. Altamente viscosa, muitas vezes com uma textura envolvente e aveludada. Carbonatação de baixa a moderada. Aquecimento alcoólico macio, leve a moderado, é opcional.

\textbf{Comentários}: Grande parte da cor surge de uma longa fervura. Alguns exemplos comerciais podem ser mais fortes do que as estatísticas vitais.

\textbf{História}: Um estilo de cerveja artesanal americana que foi fabricada pela primeira vez na Rubicon Brewing Company em 1988. Geralmente de lançamento sazonal no inverno ou datada por safra (*vintage*) ou de lançamento único.

\textbf{Ingredientes}: Normalmente fabricada com uma combinação de malte americano de duas fileiras e trigo americano. O estilo geralmente usa 50\% ou mais de malte de trigo. Uso restrito de maltes escuros. Qualquer variedade de lúpulo pode ser usada. Pode ser envelhecida em carvalho.

\textbf{Comparação de Estilos}: Mais do que simplesmente uma Barleywine à base de trigo, muitas versões têm notas frutadas e lupuladas muito expressivas, enquanto outras desenvolvem complexidade através do envelhecimento em carvalho. Menos ênfase no lúpulo do que na American Barleywine. Tem raízes na American Wheat Beer ao invés de qualquer estilo de trigo alemão, então não deve ter nenhum caráter de levedura Weizen.

\textbf{Estatísticas}: IBU: 30 - 60
SRM: 6 - 14
OG: 1,080 - 1,120
FG: 1,016 - 1,030
ABV: 8% - 12%

\textbf{Exemplos Comerciais}: The Bruery White Oak, Castelain Winter Ale, Perennial Heart of Gold, Two Brothers Bare Tree.

\textbf{Última Revisão}: Wheatwine (2015)

\textbf{Atributos de Estilo}: amber-color, balanced, craft-style, hoppy, north-america, strong-ale-family, top-fermented, very-high-strength, wheat-beer-family
