\phantomsection
\subsection*{21B. Specialty IPA: White IPA}
\addcontentsline{toc}{subsection}{21B. Specialty IPA: White IPA}
\textbf{Impressão Geral}: Uma versão frutada, condimentada e refrescante de uma American IPA, mas com uma cor mais clara, menos corpo e apresentando as características distintivas de levedura ou especiarias típicas de uma Witbier. \\
\textbf{Aparência}: Cor de dourado claro a dourado profundo. Tipicamente turva. Colarinho branco de moderado a alto, denso e persistente. \\
\textbf{Aroma}: Ésteres moderados, geralmente que remetem a laranja, toranja, damasco ou, às vezes, banana. Condimentos leves são opcionais, habitualmente que remetem a coentro, casca de laranja, pimenta ou cravo. Aroma de lúpulo de médio baixo a médio, usualmente de frutas de caroço, de frutas cítricas ou tropicais. Ésteres e especiarias podem reduzir a percepção do aroma do lúpulo. Baixo teor de malte neutro, de grãos ou de pão. Leve aroma de álcool é opcional. \\
\textbf{Sabor}: Ésteres de moderados a altos, sabor de lúpulo de médio baixo a médio alto e especiarias leves, todos com os mesmos descritores do aroma. Sabor de leve de malte, talvez um pouco de pão. Alto amargor. Final moderadamente seco e refrescante. Sabor de álcool, em segundo plano, é opcional. \\
\textbf{Sensação na Boca}: Corpo médio leve. Carbonatação de média a média alta. Adstringência leve de especiarias é opcional. Leve aquecimento alcoólico é opcional. \\
\textbf{Comentários}: Uma interpretação pelos cervejeiros artesanais de uma American IPA cruzada com uma Witbier. A impressão de especiarias pode vir da levedura belga, adições de especiarias ou de ambos. \\
\textbf{História}: Os cervejeiros artesanais americanos desenvolveram o estilo como uma cerveja sazonal para o final do inverno ou primavera, visando atrair os consumidores de Witbier e de IPA. \\
\textbf{Ingredientes}: Maltes claros e de trigo, levedura Witbier belga, lúpulos cítricos do tipo americano. Coentro e casca de laranja são opcionais. \\
\textbf{Comparação de Estilos}: Amarga, lupulada e com teor alcoólico mais alto como uma American IPA, mas frutada, picante e leve como uma Witbier. Normalmente, os lúpulos de adição tardia não são tão proeminentes quanto em uma American IPA. \\
\begin{tabular}{@{}p{35mm}p{35mm}@{}}
  \textbf{Estatísticas}: & OG: 1,056 - 1,065 \\
  IBU: 40 - 70  & FG: 1,010 - 1,016 \\
  SRM: 5 - 6  & ABV: 5,5\% - 7\%
\end{tabular}\\
\textbf{Exemplos Comerciais}: Lagunitas A Little Sumpin' Sumpin' Ale, New Belgium Accumulation. \\
\textbf{Última Revisão}: Specialty IPA: White IPA (2015) \\
\textbf{Atributos de Estilo}: bitter, craft-style, high-strength, hoppy, ipa-family, north-america, pale-color, specialty-family, spice, top-fermented
