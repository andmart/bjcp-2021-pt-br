\phantomsection
\subsection*{21B. Specialty IPA: Black IPA}
\addcontentsline{toc}{subsection}{21B. Specialty IPA: Black IPA}
\textbf{Impressão geral}: Uma cerveja com a secura, o equilíbrio de lúpulo e as características de sabor de uma American IPA, mas com a cor mais escura. Maltes mais escuros adicionam um sabor suave e um suporte, não um caráter fortemente torrado ou queimado.

\textbf{Aroma}: Aroma de lúpulo de moderado a alto, muitas vezes com um caráter de frutas do caroço, tropical, cítrico, resinoso, de pinho, de frutas vermelhas ou de melão. Malte de muito baixo a moderado, possivelmente com notas leves de chocolate, café ou tosta, bem como um dulçor de caramelo em segundo plano. Perfil de fermentação limpo, mas ésteres leves são aceitáveis.

\textbf{Aparência}: Cor de marrom escuro a preta. Limpa, quando não for opaca. Enovado leve é permitido, mas não deve ser turvo. Colarinho moderado, persistente, de castanho claro a castanho.

\textbf{Sabor}: Sabor de lúpulo de médio-baixo a alto, com os mesmos descritores do aroma. Sabor de malte de baixo a médio, com notas contidas de chocolate ou café, mas não queimadas ou como de cinzas. As notas torradas não devem colidir com os lúpulos. Caramelo claro ou toffee é opcional. Amargor de médio-alto a muito alto. Final de seco a ligeiramente seco, com um retrogosto amargo, mas não desagradável. Muitas vezes com um leve sabor de torra que pode contribuir para a impressão seca. Ésteres de baixos a moderados são opcionais. Sabor de álcool em segundo plano é opcional.

\textbf{Sensação na Boca}: Macia. Corpo de médio-leve a médio. Carbonatação média. Leve cremosidade é opcional. Leve aquecimento alcoólico é opcional.

\textbf{Comentários}: A maioria dos exemplos são de teor alcoólico padrão. Exemplarers com teor alcoólico mais alto podem parecer uma porter lupulada se feita muito no extremo do estilo, o que pode prejudicar sua facilidade de ser bebida.

\textbf{História}: Uma variante da American IPA produzida comercialmente pela primeira vez por Greg Noonan como Blackwatch IPA, por volta de 1990. Popularizada a partir de meados dos anos 2000 no noroeste do Pacífico e no sul da Califórnia dos Estados Unidos, quando foi popular até o início de 2010 antes de cair na obscuridade no EUA.

\textbf{Ingredientes}: Maltes torrados sem casca (\textit{debittered}). Qualquer perfil de lúpulo americano ou do novo mundo é aceitável; novas variedades de lúpulo continuam a ser lançadas e, por isso, não se deve restringir o estilo às características de lúpulo listadas como exemplos.

\textbf{Comparação de Estilos}: Equilíbrio e impressão geral de uma American ou de uma Double IPA, com torra contida, semelhante a encontrada na Schwarzbier. Não é tão rico e torrado como American Stout e Porter e com menos corpo, maior suavidade e maior facilidade de beber.

\begin{tabular}{@{}p{35mm}p{35mm}@{}}
  \textbf{Estatísticas}: & OG: 1,050 - 1,085 \\
  IBU: 50 - 90  & FG: 1,010 - 1,018 \\
  SRM: 25 - 40  & ABV: 5,5\% - 9\%
\end{tabular}

\textbf{Exemplos Comerciais}: 21st Amendment Back in Black, Duck-Rabbit Hoppy Bunny ABA, Stone Sublimely Self-Righteous Black IPA.

\textbf{Última Revisão}: Specialty IPA: Black IPA (2015)

\textbf{Atributos de Estilo}: bitter, craft-style, dark-color, high-strength, hoppy, ipa-family, north-america, specialty-family, top-fermented
