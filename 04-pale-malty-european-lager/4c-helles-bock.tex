\phantomsection
\subsection*{4C. Helles Bock}
\addcontentsline{toc}{subsection}{4C. Helles Bock}
\textbf{Impressão Geral}: Uma lager alemã relativamente clara, forte e maltada com um final bem atenuado que aumenta a facilidade de beber. O caráter de lúpulo é geralmente mais aparente e o caráter de malte menos profundamente rico do que em outras Bocks.

\textbf{Aparência}: Cor de dourado profundo a âmbar claro. Limpidez de brilhante a limpa. Espuma volumosa, cremosa, persistente e de cor branca.

\textbf{Aroma}: Aroma de malte adocicado como cereias de moderado a forte, muitas vezes com qualidade levemente tostada e baixos níveis de produtos de reação de Maillard. Aroma de lúpulo condimentado, herbal e/ou floral moderadamente baixo é opcional. Perfil de fermentação limpo. Opcionalmente pode apresentar ésteres frutados de baixa intensidade. Álcool muito leve é opcional.

\textbf{Sabor}: Sabor de malte de moderado a moderadamente forte, predominância de adocicado lembrando cereais, massa de pão, pão e/ou levemente tostado com alguns produtos ricos de reação de Maillard adicionando complexidade. Pouco sabor de caramelo é opcional. Sabor de lúpulo de baixo a moderado de perfil condimentado, herbal, floral, apimentado é opcional, mas presente nos melhores exemplares. Amargor de lúpulo moderado, mais no equilíbrio que em outras Bocks. Perfil de fermentação limpo. Bem atenuada, não é enjoativa e com um final moderadamente seco que pode ter sabor tanto de malte quanto de lúpulo.

\textbf{Sensação na Boca}: Corpo médio. Carbonatação de moderada a moderadamente alta. Suave e limpa, sem aspereza ou adstringência, apesar do amargor de lúpulo mais alto. Leve aquecimento alcoólico é opcional.

\textbf{Comentários}: Também conhecida como Maibock. Em comparação com cervejas Bock mais escuras, os lúpulos compensam o baixo nível de produtos de reação de Maillard no equilíbrio.

\textbf{História}: Um desenvolvimento bem recente se comparado aos outros membros da família bock. A disponibilidade da Maibock é uma oferta sazonal associada à primavera, ao mês de maio e pode apresentar uma gama de coloração e lupulagem mais ampla do que é visto nos produtos para exportação.

\textbf{Ingredientes}: Uma combinação de maltes Pils, Vienna e Munich. Sem adjuntos. É possível o uso de malte tipo crystal claro em pequenas quantidades. Lúpulos alemães tradicionais. Levedura de lager limpa. Brassagem por decocção é tradicional, mas a fervura é menor do que em uma Dunkles Bock para conter o desenvolvimento da cor. Água mole.

\textbf{Comparação de Estilo}: Pode ser pensada tanto como uma versão mais clara de uma Dunkles Bock ou como uma Munich Helles ou Festbier com a força de uma bock. Embora muito maltada, essa cerveja geralmente tem menos riqueza e sabores de malte escuro e pode ser mais seca, mais lupulada e mais amarga do que uma Dunkles Bock. Menos forte do que uma Doppelbock clara, mas com sabores similares.

\begin{tabular}{@{}p{35mm}p{35mm}@{}}
  \textbf{Estatísticas}: & OG: 1,064 - 1,072 \\
  IBU: 23 - 35  & FG: 1,011 - 1,018  \\
  SRM: 6 - 9  & ABV: 6,3\% - 7,4\%
\end{tabular}

\textbf{Exemplos Comerciais}: Altenmünster Maibock, Ayinger Maibock, Chuckanut Maibock, Einbecker Mai-Urbock, Hacker-Pschorr Hubertus Bock, Hofbräu Maibock, Mahr’s Heller Bock.

\textbf{Última Revisão}: Helles Bock (2015)

\textbf{Atributos de Estilo}: bock-family, bottom-fermented, central-europe, high-strength, lagered, malty, pale-color, traditional-style.
