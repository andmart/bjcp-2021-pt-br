\clearpage
\phantomsection
\divisorLine
\section*{Introdução ao Guia de Estilos 2021}
\addcontentsline{toc}{section}{Introdução ao Guia de Estilos 2021}
\textit{O Guia de estilos do BJCP 2021 é uma pequena revisão da edição de 2015 e uma grande atualização da edição de 2008. Os objetivos da edição 2015 eram a de melhor descrever os estilos de cerveja ao redor do mundo como encontrados em seus mercados locais, se manter em dia com as tendências emergentes no mercado da cerveja artesanal, descrever cervejas históricas que passam a ter apreciadores, descrever de forma melhor as características sensoriais de ingredientes modernos, se valer de novas pesquisas e referências e ajudar organizadores de competições a administrar a complexidade de seus eventos de forma melhor. Estes objetivos não mudaram na edição 2021.}

\textit{No guia de 2015, diversos estilos foram adicionados, alguns estilos foram divididos em múltiplas categorias e alguns simplesmente renomeados. Os estilos são organizados e agrupados tendo por base estilos com características de julgamento similares ao invés de uma origem ou nome de família em comum. Não faça presunções que a mesma característica primária (por exemplo cor, teor alcoólico, equilíbrio, sabor predominante, país de origem) foi utilizado para determinar o agrupamento de cada categoria – o motivo é mais detalhado.}

\textit{Se você possui familiaridade com o guia de estilos de 2015 mudamos alguns nomes e alguns estilos foram movidos de categorias provisórias, históricas ou estilos locais para a parte principal do guia de estilos. De forma intencional nós trabalhamos para manter as mudanças de estilos, numerações e adições ao mínimo.}

\begin{multicols*}{2}
\subsection*{Estilos e Categorias}
\addcontentsline{toc}{subsection}{Estilos e Categorias}
O guia de estilos do BJCP utiliza alguns termos específicos com significado particular: Categoria, Subcategoria e Estilo. Ao pensar em estilos de cerveja, hidromel e sidra, a subcategoria é a indicação mais importante - subcategoria se refere essencialmente à mesma coisa que estilo e identifica as características principais de um tipo de cerveja, hidromel ou sidra. Cada estilo tem uma descrição bem definida que é a ferramenta básica utilizada durante o julgamento.

Quando as descrições de um estilo de cerveja não convencional se referirem à um estilo clássico, nós estamos nos referindo a um estilo (ou subcategoria) nomeado anteriormente à sessão de estilos de especialidade do guia de estilos.

As categorias mais amplas são agrupamentos arbitrários de estilos de cerveja, hidromel e sidra, normalmente com características sensoriais similares. Subcategorias não são necessariamente relacionadas umas com as outras dentro de uma mesma categoria. A proposta por trás da estrutura das categorias é agrupar estilos de cerveja, hidromel e sidra de forma a facilitar o julgamento durante as competições. Não tente deduzir significados adicionais dos agrupamentos - nenhuma associação histórica ou geográfica é implicada ou intencionada.

Competições podem criar suas próprias categorias de premiação que são distintas das categorias de estilos contidas neste guia de estilo. Não existe nenhum requerimento para que as competições utilizem as categorias de estilos como categorias de premiação! Estilos individuais podem ser agrupados de qualquer maneira para criar as categorias de premiação dentro de uma competição; como por exemplo, para equilibrar e melhor distribuir o número de amostras dentro de cada categoria de premiação.

Enquanto categorias de estilos são a forma mais vantajosa para efeitos de julgamento, pois agrupam as cervejas com características sensoriais semelhantes, nós reconhecemos que esta talvez não seja a melhor forma para aprender sobre os estilos de cerveja. Para estudo, os estilos podem ser agrupados em famílias de estilos para que possam ser comparados e contrastados. As cervejas também podem ser agrupadas por país de origem para melhor entendimento da história da cerveja em um país, ou para aprender sobre o mercado local. Qualquer um destes agrupamentos é perfeitamente aceitável, os estilos só foram agrupados desta forma para facilitar o julgamento de competições. Veja o apêndice A para formas alternativas de agrupamentos.
\subsection*{Nomenclatura dos Estilos e Categorias}
\addcontentsline{toc}{subsection}{Nomenclatura dos Estilos e Categorias}

Escolhemos nomes e títulos para melhor representar estilos e agrupamentos em nosso sistema de categorização.  Não deixe com que estes nomes interfiram no seu entendimento real das descrições do estilo – este é o ponto principal do guia de estilos. Nós não estamos dizendo para as cervejarias como elas devem chamar seus produtos, estamos apenas fornecendo nomes que podem ser usados para uma fácil referência.\\
Entendemos que muitos dos estilos estabelecidos por nós podem ter nomes alternativos, ou serem chamados por nomes diferentes em outras (e as vezes nas mesmas) partes do mundo. No passado, usamos diversos nomes em títulos de estilos para evitar demonstrar alguma preferência, mas isso levava ao uso equivocado de todas versões do nome de forma simultânea. Então, decidimos escolher um nome não ambíguo para cada estilo.\\
Adicionamos o país ou região de origem à alguns títulos de estilos para diferenciar entre estilos usando um nome em comum (como Porter). Selecionamos os títulos com a intenção que sejam únicos e descritivos, mas não necessariamente são a forma como eles são chamados em seus mercados locais. Não estamos buscando nenhuma apropriação ou domínio baseado nos nomes que selecionamos, também pedimos desculpas por qualquer ofensa não intencional que possa ocorrer na esfera política, ética ou social.\\
Alguns nomes que usamos são protegidos por marcas registradas ou apelações de origem controlada. Não estamos dizendo que estes não devem ser respeitados, ou que cervejarias comerciais podem usar estes nomes. Pelo contrário, estas são as formas mais apropriadas de utilizar os nomes quando tratando de alguns estilos. Caso este conceito seja difícil de entender, apenas assuma que existe uma designação implícita de “-estilo” em todo nome de estilo. Nós não queríamos utilizar “-estilo” na forma como nomeamos visto que este é um guia de estilos, portanto tudo é um estilo.\\
(Nota do tradutor): O significado se perde um pouco na tradução. O importante é entender que alguns estilos são protegidos por denominação de origem controlada ou podem conter algum registro sobre a nomenclatura. Por exemplo cervejas Lambic ou Trapistas. Cervejas não podem utilizar estes nomes seus rótulos sem permissão. O “-estilo” funciona como “tipo”, por exemplo cerveja “tipo” Lambic.
\subsection*{Utilizando o Guia de Estilos}
\addcontentsline{toc}{subsection}{Utilizando o Guia de Estilos}
Quando publicamos as versões anteriores do guia de estilos, não tínhamos ideia de quão prevalentes e difundidas elas se tornariam.  Nós acreditávamos que estávamos criando uma padronização de um conjunto de descrições de estilos para competições de cervejeiros caseiros, mas depois descobrimos que elas foram amplamente adotadas à nível mundial para descrever cervejas de forma geral. Diversos países com mercados de cerveja artesanal emergentes passaram a usar o guia como um manual do que produzir. Consumidores e grupos comerciais passaram a utilizar os estilos para descrever seus produtos. Infelizmente, muitos fizeram suposições surpreendentes muito além de nossa intenção original, consequentemente utilizando o guia de estilos como uma espécie de “Rosetta Stone” universal para cerveja.

Ao passo que entendemos que o guia de estilos pode ter sido utilizado de forma errônea para fins além de nossa intenção original, também notamos ele ser utilizado de forma errada em competições e para outras propostas dentro do BJCP como preparação para exames e correção de exames. Algumas pessoas interpretaram de forma errada o guia de estilos e, sem querer, passaram adiante o mal-uso dele. Nossa esperança é que a informação contida nesta sessão ajude a evitar más interpretações e o uso equivocado no futuro. Caso você encontre alguém utilizando o guia de estilos de forma incorreta, por favor indique esta sessão.

As diretrizes a seguir expressam nossas intenções originais e foram feitas para limitar o mal-uso do guia, não para evitar que o guia seja adotado para novos usos:
\begin{enumerate}
\item \textbf{O Guia de estilos do BJCP é um guia e não uma especificação.} Entenda e não se esqueça destas palavras. O guia de estilos tem o propósito de descrever as características gerais e os exemplares mais comuns, serve para auxiliar no julgamento; não foi criado para ser uma especificação que deve ser aplicada de forma rigorosa e punir exemplares um pouco fora do esperado. São sugestões, não limites expressos. Permita alguma flexibilidade no julgamento para que amostras bem executadas possam ser recompensadas. As diretrizes são escritas em detalhes para facilitar o processo de avaliação estruturada de cervejas conforme exercitado no julgamento de competições caseiras, não use cada detalhe individual na descrição de estilos como um motivo para desqualificar uma amostra.
\item \textbf{O guia de estilos foi escrito pensando nas competições de cervejas caseiras.} Descrições individuais de estilos são escritas primordialmente para auxiliar no julgamento. Nós temos, em alguns casos, buscado definir parâmetros claros entre os estilos para criar categorias de julgamento que não se sobrepõem. Nós entendemos que alguns estilos podem se sobrepor no mercado, alguns exemplares comerciais podem ir além de parâmetros do estilo. Nós organizamos as categorias de estilo com o propósito de organizar competições de cervejas caseiras, não para descrever e comunicar os estilos do mundo para outro público.
\item \textbf{Sabemos que muitas pessoas usam nosso guia de estilos.} Entendemos que outras organizações e grupos usam nosso guia de estilos para propostas que vão muito além de nossa proposta inicial. No sentindo que estes grupos enxergam valor em nosso trabalho, estamos felizes em ter nosso guia de estilos utilizado. Permitimos o uso de nossa nomenclatura e sistema de numeração de forma livre por outras pessoas. No entanto, não faça presunções precipitadas a respeito da natureza da cerveja e dos estilos baseado na aplicação do guia do estilo além de sua finalidade original. Nós também sabemos que alguns cervejeiros artesanais estão utilizando o guia de estilos para descobrir estilos históricos, ou para produzir estilos não nativos a seus respectivos países – estamos muito entusiasmados em poder ajudar a cerveja artesanal a avançar desta forma. Porém se lembre que esta não é nossa missão original, é apenas um efeito colateral bom.
\item \textbf{Estilos mudam com o tempo.} Estilos de cerveja mudam com o passar dos anos, alguns estilos estão abertos para interpretação e debate. Só porque o nome do estilo não mudou com os anos, isto não significa que as cervejas em si também não mudaram. Cervejeiros comerciais sujeitos à regulamentação governamental e forças do mercado definitivamente mudam seus produtos ao longo do tempo. Por exemplo, porque temos agora uma cerveja conhecida como porter não significa que ela sempre foi feita desta forma ao longo de sua história. Nossos estilos de cerveja geralmente têm o propósito de descrever cervejas modernas que estão disponíveis atualmente, a menos que descrito de forma contrária (por exemplo, a categoria de cervejas históricas).
\item \textbf{Nem toda cerveja comercial se enquadra em nossos estilos.} Não assuma que toda cerveja se enquadra perfeitamente em alguma de nossas categorias. Algumas cervejarias se deleitam em criar exemplares que não se enquadram em nosso (ou em qualquer) guia de estilos. Algumas cervejarias criam cervejas chamadas pelo nome de um estilo que deliberadamente não se enquadram em nosso guia de estilos. É perfeitamente aceitável que uma cerveja comercial não se enquadre em um de nossos estilos, nós não tentamos categorizar todas cervejas comerciais – não é nossa intenção, tampouco nossa missão.
\item \textbf{Nós não definimos todos estilos de cervejas possíveis.} É claro que sabemos de estilos de cervejas que não estão definidos em nosso guia. Talvez por serem obscuros ou pouco populares, ou porque cervejeiros caseiros não estão fazendo estes estilos, não existem exemplares suficientes ou material de pesquisa que os definem de forma adequada aos nossos padrões, ou estão em uma parte do mundo que não visitamos de forma extensiva. Talvez por serem estilos históricos que não estejam mais sendo feitos, ou porque acreditamos que sejam estilos que são uma moda passageira. Independente de nossas razões, não acredite que nosso guia de estilos represente uma categorização de toda cerveja já feita – ele não é. O guia, no entanto, descreve as cervejas mais comumente feitas atualmente por cervejeiros caseiros e muitas cervejarias artesanais.
\item \textbf{Exemplares comerciais mudam com o tempo.} Assim como estilos de cerveja mudam, exemplares individuais também mudam. Uma cerveja que já foi outrora um grande exemplar do estilo pode deixar de permanecer como um. Algumas vezes a cerveja muda (possivelmente com uma mudança de proprietário da marca) ou as vezes com uma mudança de tendência do estilo que o exemplar não acompanha. Por exemplo, a Anchor Liberty ajudou a definir o estilo American IPA quando foi criada, mas hoje se assemelha mais com uma típica American Pale Ale.
\item \textbf{Ingredientes mudam com o tempo.} Lúpulos são um ótimo exemplo atual, novas variedades estão chegando ao mercado com características únicas. Cervejeiros buscando se diferenciar podem rapidamente adotar (ou abandonar) ingredientes. É difícil dizer que o perfil de um estilo de cerveja é fixo quando os ingredientes típicos mudam com frequência. Permita estas mudanças ao julgar cervejas. Por exemplo, nem todos lúpulos americanos serão cítricos ou remeterão à pinho. Não seja rígido ao julgar se baseando no que era comumente utilizado no período em que o estilo foi escrito; entenda que ingredientes são tipicamente utilizados e adapte seu julgamento baseado nos perfis que estão evoluindo.
\item \textbf{A maioria dos estilos são razoavelmente abrangentes.} Alguns acreditam que nossos estilos proíbem a criatividade do cervejeiro por definir barreiras rígidas. Esta não é nossa intenção – nós acreditamos que a criatividade leva à inovação, e esta interpretação por parte dos cervejeiros deve ser permitida. No entanto, nem toda inovação é uma ótima ideia ou resulta em uma cerveja que é reconhecível no mesmo agrupamento de outras que levam o mesmo nome. Então, estilos devem ser interpretados como tendo espaço para flexibilidade, mas tudo dentro da razão.
\item \textbf{O guia de estilos não é “os 10 mandamentos”.} As palavras neste documento não são provenientes da inspiração divina –foram escritas por pessoas tentando realizar um esforço de boa-fé para descrever as cervejas conforme são percebidas. Não trate o guia como um tipo de escritura sagrada. Não se perca tanto interpretando palavras individuais de forma que você perca a intenção geral. A parte mais importante de qualquer estilo é o equilíbrio e a impressão geral; isto é, se a cerveja te lembra do estilo e é um produto facilmente bebível. Se perder nas descrições individuais fará com que você perca a essência do estilo. O mero fato de que as descrições de estilo podem mudar de uma edição do guia para a próxima deve ilustrar de forma muito clara que as palavras em si não são sagradas.
\item \textbf{Nossas diretrizes são extensíveis.} Entendemos que nossas diretrizes de estilos irão mudar no futuro e que podem haver diversos anos entre os ciclos de revisão. A principal missão do BJCP é conduzir exames e alterações constantes tornam estudar quase impossível. Então, optamos por adotar um meio de campo: nós listamos estilos provisórios em nosso website que podem ser usados da mesma forma que os estilos presentes neste guia. Isto permite com que possamos adicionar mudanças entre as edições. Nós também temos uma lista de sugestões de enquadramento de inscrições em nosso website para melhor entendimento de qual a melhor forma de inscrever estilos não definidos em nossas diretrizes ou como estilos provisórios. Estes recursos, junto da extensão de alguns estilos como por exemplo Specialty IPA e Historical Beer, permitem estilos definidos por cervejeiros serem utilizados em competições.  Combinadas, estes três recursos permitem que o guia de estilos evolua entre atualizações maiores.
\item \textbf{Não somos a polícia da cerveja.} Categorizamos e descrevemos estilos de cerveja que percebemos que existem e que são usados. De forma alguma estamos falando para os cervejeiros comerciais o que eles podem produzir, ou falando que eles estão errados caso seus produtos não se adequem às nossas diretrizes. Nós também não criamos estilos na esperança de que se tornem populares. O estado geral do mercado de cerveja de qualquer país não é nossa preocupação.
\item \textbf{Diferentes formatos existem.} Nossas diretrizes de estilos aparecem em diversos locais de terceiros, diversas plataformas móveis (mobile) e também são traduzidos para outras línguas. Infelizmente, nem todas estas versões contêm o texto completo de nosso guia de estilos, ou possuem traduções completamente fieis. Seja cuidadoso ao utilizar formatos fornecidos por alguém que não seja o BJCP de forma direta, quando em dúvida, sempre consulte a fonte original.
\item \textbf{O BJCP não organiza competições.} Às vezes competições utilizam softwares que limitam comentários, ou tornam difícil de seguir as instruções de inscrição presentes no guia de estilos. Comunique problemas aos organizadores da competição e aos fornecedores do software. Nosso intuito é que toda informação permitida e requerida pelo guia de estilos seja fornecida pelos cervejeiros, seja aceita pelas competições e fornecida aos juízes
\end{enumerate}
\subsection*{Formato da descrição de estilos}
\addcontentsline{toc}{subsection}{Formato da descrição de estilos}
Usamos um formato padrão para descrever os estilos de cerveja. As sessões dentro deste modelo/template têm significado específico que deve ser compreendido para não serem mal utilizados:
\begin{itemize}
\item \textbf{Impressão Geral:} Esta sessão descreve a essência do estilo - aqueles pontos que o distinguem dos demais estilos e o tornam único. Também pode ser pensado como uma descrição expandida à nível de consumidores útil para descrever e diferenciar o estilo para alguém que não é um beer geek ou juiz. Essa sessão reconhece os diversos usos fora de julgamentos e possibilita a outros descrevem uma cerveja de forma simples sem precisar de todos detalhes necessários aos juízes.
\item \textbf{Aparência, Aroma, Sabor, Sensação na Boca:} Estas quatro sessões são os blocos sensoriais básicos que constroem e definem o estilo e são os padrões pelos quais a cerveja é julgada durante uma competição. Estas sessões focam nas percepções sensoriais provenientes dos ingredientes, não nos ingredientes ou nos processos em si. Por exemplo, dizer que uma Munich Helles possui um gosto como o de malte Pilsen Continental é uma ótima forma reduzida de descrever o que é percebido; exceto, claro, caso você não tenha ideia de qual o gosto do malte Pilsen continental. Nossas diretrizes de guia de estilos são escritas de forma que um juiz treinado que não tenha provado exemplares de um determinado estilo possam efetuar um bom trabalho julgando e usando um modelo de avaliação estruturado junto do guia de estilos como referência.
\item \textbf{Comentários:} Esta sessão contém trivialidades interessantes ou notas adicionais sobre o estilo que não afetam sua avaliação sensorial. Nem todo estilo terá comentários extensos; alguns são bem simples.
\item \textbf{História:} O BJCP não é uma organização com foco em pesquisa histórica. Nos valemos nas informações disponíveis, revisando frequentemente nossos sumários conforme novos fatos são publicados. Nossas histórias consistem de sumários abreviados trazendo os pontos mais importantes do desenvolvimento do estilo. Por favor não entenda que estas notas são a história completa e inteira dos estilos.
\item \textbf{Ingredientes:} Identificamos os ingredientes ou processos típicos e comuns que trazem o caráter distinto do estilo. Por favor, não trate estas notas como receitas ou requerimentos. Cerveja pode ser feita de diversas maneiras diferentes.
\item \textbf{Comparação de Estilos:} Como alguns podem entender melhor um estilo não conhecido ao ser descrito em comparação com outros estilos conhecidos, fornecemos notas nos pontos chave que distinguem um estilo de outro estilo similar ou relacionado. Nem toda comparação de estilo possível é listada no guia.
\item \textbf{Instruções para Inscrição:} Esta sessão identifica as informações necessárias para que juízes possam avaliar a inscrição em uma competição. Essa informação deve sempre ser fornecida por quem inscrever uma amostra, aceita pelo software utilizado na competição e providenciada aos juízes. Participantes devem conseguir suprir comentários adicionais sobre suas inscrições, sujeito a revisão pelos organizadores da competição.
\item \textbf{Estatísticas Vitais:}
Original Gravity (OG) = Densidade Original ou Extrato Primário
Final Gravity (FG) = Densidade Final ou Extrato Final
Alcohol-by-Volume (ABV) = Volume por Volume (vol.)
International Bitterness Units (IBUs) = Unidade Internacional de Amargor (IBU)
Para aqueles de fora dos Estados Unidos que usam a escala de cor do \textit{European Brewing Convention} (EBC), saiba que o valor EBC é praticamente o dobro do valor equivalente no SRM. Para aqueles familiarizados com o sistema Lovibond, Lovibond é praticamente equivalente ao SRM para cores de cervejas excluindo as mais escuras. Para os puritanos presentes, estamos tratando do que é distinguível para um juiz usando os olhos, não químicos utilizando equipamentos analíticos dentro de um laboratório.
Algumas categorias de estilos incluem múltiplos estilos que apresentam certa continuidade como por exemplo a \textit{English Bitter} ou \textit{Scottish Ale}. Quando fornecemos uma linha divisória entre estes estilos, tipicamente usamos um único número para representar a faixa superior de um estilo e a faixa inferior do próximo. Isso não implica que a cerveja que esteja no limite de um parâmetro (por exemplo, ABV ou OG) deve ser inscrita em ambos estilos. Nenhuma sobreposição é intencional. Nesses casos, trate o limite superior como “terminando logo antes” e o limite inferior como “começando aqui” dos números apresentados.
Tenha em mente que essas estatísticas vitais ainda são apenas diretrizes, não dados absolutos. Exemplos Comerciais fora destes parâmetros com certeza existem, mas estas estatísticas são feitas para descrever onde a maior parte dos exemplares está. Elas ajudam a determinar a ordem de julgamento, não se um exemplo deve ser desclassificado.
\item \textbf{Exemplos Comerciais:} Incluímos uma seleção dos exemplares comerciais atuais que acreditamos serem representantes do estilo no período em que publicamos o guia. Podemos publicar exemplares adicionais no site do BJCP no futuro. Nós não podemos garantir que as cervejarias continuarão produzindo estes exemplares, que os nomes permaneçam o mesmo, que a receita não vá mudar ou que estejam sempre disponíveis na sua loja local. Alguns são sazonais, rotativos, apenas encontrados no \textbf{brewpub} ou mesmo difíceis de encontrar fora de festivais, competições e mercados locais.
Não faça presunções de significados adicionais sobre a ordem em que listamos os exemplares. Não assuma que todo exemplo comercial listado obteria a nota máxima quando avaliado seguindo as descrições do estilo. Não é apenas porque um exemplar comercial é referenciado para o estilo que todo exemplar será sempre de classe mundial. Algumas cervejas podem sofrer maus tratos e algumas mudam com o passar do tempo.
\item \textbf{Atributos de Estilo:} Para facilitar a organização dos estilos em agrupamentos alternativos, utilizamos \textbf{tags} para evidenciar atributos ou informações sobre o estilo. As \textbf{tags} não estão em nenhuma ordem em particular e não devem ser utilizadas para aferir um significado mais profundo.
\end{itemize}
\subsection*{Linguagem de descrição dos estilos}
\addcontentsline{toc}{subsection}{Linguagem de descrição dos estilos}

O guia de estilos é um conjunto de documentos longos e algumas descrições de estilos são complexas. Para evitar que a linguagem usada seja excessivamente entediante, sinônimos (palavras ou frases significando exatamente a mesma coisa ou tendo um sentido muito próximo) são frequentemente utilizados. Não busque achar mais significado nos sinônimos do que o esperado. No passado, algumas pessoas questionaram a diferença entre baixo e leve, médio e moderado, escuro e profundo e diversos outros exemplos. A resposta é, não existe diferença entre estas palavras neste contexto. Usamos elas pensando que elas têm o mesmo significado (normalmente, intensidades relativas de percepções). Entenda essas palavras com o seu significado básico. Caso você se pegue analisando o guia de estilos como se estivesse à procura de uma mensagem secreta dentro de uma música tocada de trás pra frente, você está se esforçando demais.

Quando usamos diversas palavras para expressar coisas similares, estamos apenas tentando ser versados e usar um vocabulário razoavelmente instruído. Não buscamos ser a polícia da linguagem e dizer que um sinônimo está sempre certo e os demais sempre errados. Portanto não busque por inconsistências no uso de termos, nem tente adicionar distinções com nuances em diferentes palavras usadas para expressar o mesmo contexto. Não exija que as palavras usadas no guia de estilos sejam exatamente as palavras usadas nas súmulas ou exames. Preocupe-se mais com o conceito que está sendo passado e menos com a expressão específica do conceito.

Formatamos as listas utilizando a vírgula de Oxford, que é uma construção gramatical menos ambígua. Ao descrever listas de características, \textit{\textbf{“ou”} significa algum ou todos os itens podem estar presentes, \textbf{“todos”} significa que todos os itens devem estar presentes, \textbf{“qualquer um/ou”} significa que apenas um pode estar presente, \textbf{“nenhum”} significa que nenhum dos itens podem estar presentes.} O uso anterior de \textbf{“e/ou”} foi substituído por apenas \textbf{“ou”}, que tem o mesmo significado lógico.

Quando usamos nomes de estilos em Letras Maiúsculas, a intenção é que seja uma referência cruzada aos estilos contidos neste guia. Nomes de estilos sem letras maiúsculas representa uma referência mais generalista.

Esteja atento aos modificadores usados nas descrições dos estilos. Busque por orientação na magnitude e qualidade de cada característica. Note que diversas características são opcionais. Cervejas que não trazem estes elementos não necessários não devem ser rebaixadas. Se uma intensidade é usada juntamente com um indicador opcional, significa que qualquer intensidade, desde nenhuma até a listada, é aceitável, mas esta característica não é obrigatória.

Frases como \textit{“pode ter”}, \textit{“pode conter”}, \textit{“pode demonstrar”}, \textit{“é aceitável”}, \textit{“é apropriado”}, \textit{“é típico”}, \textit{“opcionalmente”}, etc. são indicadores de elementos opcionais. Elementos necessários são normalmente escritos com frases declaratórias ou usam palavras como \textit{“deve”} ou \textit{“precisa”}. Elementos que não devem estar presentes normalmente usam frases como \textit{“inapropriado”}, \textit{“sem”} ou \textit{“não apresenta”}. Novamente, entenda essas palavras no seu sentido bruto.

Não foque exageradamente em palavras soltas ou frases dentro das descrições dos estilos à ponto de perder o propósito maior da descrição. Entenda a impressão geral do estilo, o equilíbrio geral e como o estilo difere de estilos relacionados ou similares. Não dê importância desproporcional à frases especificas se elas forem mudar a impressão geral, equilíbrio, o propósito do estilo ou caso isso faça com que ela seja desqualificada ou rebaixada à problemática para o estilo.
\end{multicols*}