\phantomsection
\subsection*{16A. Sweet Stout}
\addcontentsline{toc}{subsection}{16A. Sweet Stout}
\textbf{Impressão Geral}: Uma stout muito escura, doce, encorpada e levemente torrada que pode sugerir café com creme ou expresso adoçado. \\
\textbf{Aparência}: Cor de marrom muito escuro a preto. Límpida, se não for opaca. Colarinho cremoso de castanho claro a marrom. \\
\textbf{Aroma}: Aroma suave de cereais torrados, às vezes com notas de café ou chocolate. Muitas vezes existe uma impressão de dulçor semelhante a creme. O frutado pode ser de baixo a moderadamente alto. Diacetil baixo opcional. Aroma de lúpulo baixo floral ou terroso opcional. \\
\textbf{Sabor}: Sabores escuros, torrados, café ou chocolate dominam o paladar. Ésteres frutados de baixo a moderado. Amargor moderado. O dulçor médio a alto oferece um contraponto ao caráter torrado e ao amargor, perdurando até o final. O equilíbrio entre cereais escuros ou maltes e dulçor pode variar, de bastante doce a moderadamente seco e um pouco torrado. Diacetil baixo opcional. Baixo sabor de lúpulo floral ou terroso opcional. \\
\textbf{Sensação na Boca}: Corpo médio-cheio a cheio e cremoso. Carbonatação baixa a moderada. O alto dulçor residual de açúcares não fermentados aumenta a sensação na boca cheia. \\
\textbf{Comentários}: A densidade é baixa na Grã-Bretanha (às vezes mais baixas do que as estatísticas abaixo), mais altas nos produtos exportados e nos EUA. Existem variações, no nível de dulçor residual, na intensidade do caráter torrado, e o equilíbrio entre estas duas variáveis é passível de interpretação. \\
\textbf{História}: Um estilo inglês de stout desenvolvido no início de 1900. Historicamente conhecidas como stouts “Milk” ou “Cream”, legalmente essa designação não é mais permitida na Inglaterra, mas pode ser aceitável em outros lugares. O nome “Milk” é derivado do uso da lactose do açúcar do leite como adoçante. Originalmente comercializado como um tônico para pessoas com deficiências e lactantes. \\
\textbf{Ingredientes}: Base de malte pale com maltes ou cereais escuros. Pode usar adjuntos de cereais ou açúcar. A lactose é frequentemente adicionada para fornecer doçura residual adicional. \\
\textbf{Comparação de Estilos}: Muito mais doce e menos amargo do que outras stouts, exceto a mais forte Tropical Stout. O caráter torrado é suave, não queimado como outras stouts. Pode ser semelhante em equilíbrio à Oatmeal Stout, embora com mais dulçor. \\
\begin{tabular}{@{}p{35mm}p{35mm}@{}}
  \textbf{Estatísticas}: & OG: 1,040 - 1,060 \\
  IBU: 20 - 40  & FG: 1,012 - 1,024 \\
  SRM: 30 - 40  & ABV: 4\% - 6\%
\end{tabular}\\
\textbf{Exemplos Comerciais}: Bristol Beer Factory Milk Stout, Firestone Nitro Merlin Milk Stout, Left Hand Milk Stout, Lancaster Milk Stout, Mackeson's XXX Stout, Marston’s Oyster Stout. \\
\textbf{Última Revisão}: Sweet Stout (2015) \\
\textbf{Atributos de Estilo}: british-isles, dark-color, malty, roasty, standard-strength, stout-family, sweet, top-fermented, traditional-style
