\phantomsection
\subsection*{4A. Munich Helles}
\addcontentsline{toc}{subsection}{4A. Munich Helles}

\textbf{Impressão Geral}: Uma lager alemã de cor dourada com sabor maltado suave e com um final macio e seco. A sutil presença de lúpulo condimentado, floral e/ou herbal e amargor contido ajuda a manter o equilibro maltado, mas não doce, o que ajuda a fazer dessa cerveja uma bebida refrescante para o dia a dia

\textbf{Aparência}: Cor de amarelo claro a dourado claro. Límpida. Colarinho branco, cremoso e persistente.

\textbf{Aroma}: Aroma de malte como cereais adocicado. Aroma de lúpulo condimentado, floral e/ou de herbal de baixo a moderadamente baixo. Agradável perfil de fermentação limpa com o malte dominando no equilíbrio. Os exemplares mais frescos vão ter mais aroma de adocicado de malte.

\textbf{Sabor}: Início moderadamente maltado com uma sugestão de dulçor, sabor de malte de cereais adocicado moderado com uma impressão macia e arredondada na boca, apoiada pelo amargor de baixo a médio-baixo. Final macio e seco, não é crisp e cortante. Sabor de lúpulo condimentado, floral e/ou de herbal de baixo a moderadamente baixo. O malte domina o lúpulo na boca, no final e no retrogosto, porém o lúpulo deve ser perceptível. Sem dulçor residual, somente a impressão maltada com amargor contido. Perfil de fermentação limpo.

\textbf{Sensação na Boca}: Corpo médio. Carbonatação média. Perfíl suave de maturação a frio bem feita (lagering).

\textbf{Comentários}: Exemplares muito frescos podem ter um característica de malte e lúpulo mais proeminente que desaparece com o tempo, como é frequentemente notado nas cervejas exportadas. A Helles em Munique tende a ser uma versão mais leve do que as de fora da cidade. Pode ser chamada de Helles Lagerbier.

\textbf{História}: Criada em Munique em 1894 para competir com cervejas clara do tipo Pilsner, geralmente sendo creditada pela primeira vez a Spaten. Mais popular no sul da Alemanha.

\textbf{Ingredientes}: Malte Pilsner da Europa continental. Lúpulos tradicionais alemães. Levedura lager alemã limpa.

\textbf{Comparação de Estilo}: Com equilíbrio e amargor similar a Munich Dunkel, mas com uma natureza menos adocicada de malte e de cor clara ao invés de ser escura e rica. Maior corpo e presença de malte do que uma German Pils, mas menos crisp e com menos caráter de lúpulo. Perfil de malte similar a de uma German Helles Exportbier, porém com menos lúpulo no equilíbrio e levemente menos alcoólica. Menos corpo e álcool do que uma Festbier

\begin{tabular}{@{}p{35mm}p{35mm}@{}}
  \textbf{Estatísticas}: & OG: 1,044 - 1,048 \\
  IBU: 16 - 22  & FG: 1,006 - 1,012  \\
  SRM: 3 - 5   & ABV: 4,7\% - 5,4\%
\end{tabular}

\textbf{Exemplos Comerciais}: Augustiner Lagerbier Hell, Hacker-Pschorr Münchner Gold, Löwenbraü Original, Paulaner Münchner Lager, Schönramer Hell, Spaten Münchner Hell, Weihenstephaner Original Helles.

\textbf{Revisões anteriores}: Munich Helles (2015)

\textbf{Atributos de Estilo}: bottom-fermented, central-europe, lagered, malty, pale-color, pale-lager-family, standard-strength, traditional-style
