\phantomsection
\subsection*{22B. American Strong Ale}
\addcontentsline{toc}{subsection}{22B. American Strong Ale}
\textbf{Impressão Geral}: Uma cerveja americana maltada, amarga e com teor alcoólico alto, que se encaixa no espaço entre American Barleywine, Double IPA e Red IPA. Os sabores maltados e lupulados podem ser bastante fortes, mas geralmente estão em equilíbrio.

\textbf{Aparência}: De âmbar médio a cobre profundo ou marrom claro. Colarinho de quase branco a castanho claro, com tamanho de moderado baixo a médio; pode ter baixa retenção. Boa limpidez. Possibilidade de lágrimas.

\textbf{Aroma}: Aroma de lúpulo de médio a alto, tipicamente apresentando características modernas de lúpulo americano ou do novo mundo, como cítrico, floral, pinho, resinoso, condimento, frutas tropicais, frutas de caroço, frutas vermelhas ou melão. Malte de moderado a forte, trazendo suporte ao perfil de lúpulo, caramelo médio a escuro é comum, possivelmente tostado ou como pão, além disso, são permitidas em segundo plano notas levemente torradas ou chocolate. Perfil de fermentação de neutro a moderadamente frutado. O álcool pode ser notado, mas não deve ser similar a solvente.

\textbf{Sabor}: Malte médio a alto, com qualidades de caramelo, toffee ou frutas escuras. A complexidade do malte pode ter o suporte adicional de sabores tostados, como pão ou ricos. Chocolate leve ou torrado é permitido, mas não deve ser queimado ou acentuado. Amargor de médio alto a alto. Sabor de lúpulo de moderado a alto, mesmos descritores do aroma. Ésteres de baixos a moderados. Pode ter um sabor de álcool perceptível, mas não deve ser acentuado. Dulçor de malte de médio a alto no paladar, terminando de um pouco seco a um pouco adocicado. Não deve ser xaroposa, doce ou enjoativa. Retrogosto de amargo a amargo adocicado, com lúpulo, malte e álcool perceptíveis.

\textbf{Sensação na Boca}: Corpo de médio a cheio. Um aquecimento alcoólico pode estar presente, mas não deve ser excessivamente quente. Adstringência leve de lúpulo é permitida. Carbonatação de média baixa a média.

\textbf{Comentários}: Um estilo bastante amplo que descreve cervejas rotuladas de várias maneiras, incluindo as modernas Double Red Ales e outras cervejas com teor alcoólico alto, maltadas, porém lupuladas, que não se encaixam no estilo Barleywine. Diversificada o suficiente para incluir o que pode ser esperado para uma American Amber Ale com teor alcoólico mais alto, com espaço para versões com mais teor alcoólico de outros estilos de American Ale.

\textbf{História}: Enquanto as versões artesanais modernas foram desenvolvidas como versões de força “imperial” das American Amber ou Red Ales, o estilo tem muito em comum com as históricas cervejas americanas envelhecidas (Stock Ales). Cervejas com teor alcoólico alto e maltadas eram altamente lupuladas para servir como suprimento antes da Lei Seca. Não há um legado contínuo de produzir cervejas de guarda dessa maneira, mas a semelhança é considerável (embora sem o caráter de idade).

\textbf{Ingredientes}: Base de malte pale. Maltes crystal, de médio a escuro, são comuns. Lúpulo americano ou do novo mundo. Levedura neutra ou levemente frutada.

\textbf{Comparação de Estilos}: Geralmente com teor alcoólico não tão alto e nem tão rica quanto um American Barleywine. Mais maltada do que uma American ou Double IPA. Mais intensidade de lúpulo americano do que uma British Strong Ale. Mais maltado e encorpada que uma Red IPA.

\begin{tabular}{@{}p{35mm}p{35mm}@{}}
  \textbf{Estatísticas}: & OG: 1,062 - 1,090  \\
  IBU: 50 - 100  & FG: 1,014 - 1,024   \\
  SRM: 7 - 18 & ABV: 6,3\% - 10\%
\end{tabular}

\textbf{Exemplos Comerciais}: Fat Head's Bone Head, Great Lakes Nosferatu, Oskar Blues G'Knight, Port Brewing Shark Attack Double Red, Stone Arrogant Bastard.

\textbf{Última Revisão}: American Strong Ale (2015)

\textbf{Atributos de Estilo}: amber-color, bitter, craft-style, high-strength, hoppy, north-america, strong-ale-family, top-fermented
