\clearpage
\phantomsection
\divisorLine
\section*{Introdução às Cervejas de Especialidade}
\addcontentsline{toc}{section}{Introdução às Cervejas de Especialidade }
\textit{\textbf{Cerveja de especialidade} é um termo amplo que se refere às categorias de estilo 28 à 34 e são contrastantes com as cervejas de Estilos Clássicos, das categorias 1 até 27. Os estilos clássicos possuem descrições completas e singulares, mas os estilos de especialidade compreendem uma transformação de um Estilo Base usando ou um processo ou por adição de um ou mais \textbf{Ingredientes especiais} (levedura/bactéria, defumação, madeira, fruta, especiaria, cereais ou açúcar). Descrições de estilos de cervejas de especialidade normalmente descrevem como os ingredientes ou processos especiais modificam o estilo base.}\\
\textit{Quando o \textbf{Estilo Base} para as cervejas de especialidade for necessário, um Estilo Clássico ou uma família de estilos amplas (por exemplo, IPA, Blond Ale, Stout) pode ser usado. De maneira geral, cervejas de especialidade não podem ser usadas como um Estilo Base para outras cervejas de especialidade a não ser que as instruções de inscrição para o estilo permitam expressamente. Alguns estilos de cerveja de especialidade não necessitam de um Estilo Base declarado – leia cuidadosamente a sessão de Instruções de Inscrição de cada estilo.}\\
\textit{Consulte a página dos Estilos Provisórios no website do BJCP para adições na lista de estilos. Esses Estilos Provisórios podem ser citados como o Estilo Base ao inscrever uma cerveja de especialidade. A página de Sugestões de Inscrições de Estilos do website do BJCP esclarece onde inscrever alguns estilos atualmente sem definições.}\\
\textit{Normalmente ao inscrever uma Cerveja de Especialidade que utiliza algum ingrediente que envolve comida, utilize a definição culinária deste ingrediente e não a botânica. Veja o prefácio de cada categoria de estilo para uma lista detalhada de ingredientes comuns. }\\
\textit{Presume-se que esta introdução incorporada à todas as descrições de estilos das cervejas de especialidade. Ela descreve como inscrever uma Cerveja de Especialidade de modo geral. Instruções específicas par a inscrição de cada estilo estão contidas nas descrições de estilos individuais.}\\
\begin{multicols*}{2}
\subsection*{Inscrevendo Cervejas de Especialidade}
\addcontentsline{toc}{subsection}{Inscrevendo Cervejas de Especialidade}
Muitos cervejeiros têm dúvidas sobre onde inscrever suas Cervejas de Especialidade e a melhor forma de descrevê-las. Siga estas sugestões para um resultado melhor:
\subsubsection*{Instruções de Inscrições}
Inscrever uma cerveja de especialidade em uma competição requer mais informações do que apenas selecionar o estilo. Examine a sessão de Instruções de Inscrições dentro do estilo selecionado para as informações específicas obrigatórias. Juizes necessitam destas informações e não conseguem julgar apropriadamente a sua cerveja sem elas. Sua nota será prejudicada se elas forem omitidas.

Quando estiver decidindo quais informações opcionais fornecer, se imagine na posição dos juízes. Forneça informações pertinentes que irão ajudar eles a entender sua cerveja e seu propósito. Evite informações inúteis e irrelevantes que não ajude os juízes a entender sua cerveja. Não utilize descrições exageradas de marketing/vendas. Não utilize nenhuma informação que possa fazer com que os juízes determinem sua identidade. Alguns softwares de competição limitam o tamanho dos comentários, então escolha suas palavras com cuidado.

\subsubsection*{Estilo Base}
A maioria das Cervejas de Especialidade necessitam que um Estilo Base seja identificado ou ao menos uma descrição da cerveja – olhe as Instruções de Inscrição dos estilos para o que é necessário. Se é obrigatório declarar um Estilo Base, use algum dos estilos nomeados nas Categorias 1 até 27, incluindo cervejas de estilos ou categorias com alternativas enumeradas (como \textit{Historical} ou \textit{Specialty IPA}). Estilos Provisórios do website do BJCP e Estilos Locais do Apêndice B também podem ser usados como Estilo Base.

Se as Instruções de Inscrições permitem que uma família de estilos genérica pode ser usada, isto quer dizer um estilo amplo no senso geral – como IPA, Porter ou Stout. Não é necessário que você diga qual o tipo de Porter, por exemplo, mas você deve dar uma descrição geral da cerveja. Algumas cervejas que são planejadas para demonstrar algum ingredient especial possuem bases bastante neutras.

Não utilize Cerveja de Especialidade como Estilo Base de outra Cerveja de Especialidade a menos que as Instruções de Inscrições para aquele estilo permitam explicitamente. Diversas categorias de Estilos de Especialidade possuem um estilo ‘de Especialidade’ que permite certos Ingredientes de Especialidade. Do contrário, o estilo de cerveja 34B ‘Mixed-Style’ pode ser usado.

\subsubsection*{Ingredientes de Especialidade/Especiais}
Quanto mais especifico ou extravagante você for na descrição de seu ingrediente especial, mais os juízes irão procurar por este caráter. Prove sua cerveja e então destaque os ingredientes que são identificáveis. Se apenas um ingrediente especializado foi utilizado, ele deve contribuir com uma característica identificável para a cerveja. Caso você mencione múltiplos ingredientes, eles não precisam ser todos individualmente identificáveis, mas devem contribuir para a experiência sensorial como um todo.

Caso você mencione um ingrediente fora do comum, você talvez queira descrever seu caráter, ou ao menos verificar se uma busca na internet pelo seu nome trará uma referência útil para os juízes. Fornecer um termo de busca é uma boa alternativa.

Um nome simples ou genérico do ingrediente é normalmente a melhor escolha, a menos que seu ingrediente tenha um perfil incomum. Caso você use uma combinação de ingredientes, como por exemplo condimentos, normalmente você pode se referir ao blend pelo seu nome comum (por exemplo, pumpkin pie spice/condimentos de torta de abóbora, pó de curry) ao invés de cada condimento de forma individual.

(Nota do Tradutor: para um exemplo mais local, especiarias de doce de abóbora - composto normalmente por canela, carvo e gengibre).

Caso você utilize algum ingrediente que seja potencialmente alergênico, sempre declare, mesmo que não seja possível sua percepção.

Exemplo: “alergênico: amendoim” – juizes não devem penalizar uma cerveja quando um alergênico declarado não é percebido.

\subsubsection*{Melhor Enquadramento}
Inscrever uma cerveja com um Ingrediente de Especialidade apenas e um Estilo Base Clássico é algo óbvio. Escolher o melhor estilo para a cerveja com uma combinação de Ingredientes de Especialidade requer um pouco de pensamento. Quando estiver escolhendo o estilo no qual sua Cerveja de Especialidade será inscrito, procure pelo melhor enquadramento dentre todas possíveis alternativas onde a combinação dos ingredientes é permitida. Escolhe um estilo que represente o ingrediente dominante ou caso os ingredientes estejam em equilíbrio, selecione o primeiro Estilo de Especialidade na qual ela se enquadre.

Inscrever uma cerveja em um Estilo de Especialidade é uma mensagem aos juízes que sua cerveja possui certos elementos identificáveis. Caso você tenha utilizado um ingrediente, mas ele não pode ser percebido, então não inscreva em um estilo que requer aquele ingrediente. Se os juízes não conseguem identificar algo, eles vão considerar que está ausente e reduzirão a nota de acordo.
\subsection*{Julgando Cervejas de Especialidade}
\addcontentsline{toc}{subsection}{Julgando Cervejas de Especialidade}
\textit{Juizes devem ler e entender as instruções da sessão “Inscrevendo Cervejas de Especialidade” fornecidas aos participantes.}\\
Equilíbrio geral e drinkability são os fatores críticos para o sucesso de uma Cervejas de Especialidade. A inscrição deve ser uma fusão coerente da cerveja com seus ingredientes especiais, onde nenhum ofusca/domina o outro.\\
Ingredientes Especiais devem complementar e realçar a cerveja base e o produto final deve ser agradável de beber. A cerveja deve apresentar componentes reconhecíveis de acordo com os requerimentos de inscrição para o estilo, tendo em mente que algumas cervejas podem se enquadrar em diversos estilos.\\
Juizes devem estar atentos que existe um elemento criativo ao produzir estes estilos e não devem ter preconceito de combinações que pareçam estranhas. Mantenha uma mente aberta porque algumas harmonizações de sabores estranhos podem ser surpreendentemente deliciosas. Incomum não necessáriamente quer dizer melhor, no entanto. O sabor sempre deve ser o fator de decisão final, não a criatividade percebida, a dificuldade de produção ou raridade dos ingredientes.

\subsubsection*{Avaliação Geral}
Juizes experientes vão normalmente provar as Cervejas de Especialidade buscando o prazer geral associado a ela antes de avaliar seus detalhes. Essa rápida avaliação é feita para detectar se a combinação funciona ou não. Se a cerveja possui sabores conflitantes, ela não será agradável independentemente de seu mérito técnico.\\
O velho provérbio que diz “not missing the forest for the trees – não perder a floresta pelas árvores” é aplicável. (Nota do Tradutor: no sentido de acabar perdendo o conjunto da obra analisando os pequenos detalhes). Não julgue estes estilos de forma tão rígida como os Estilos Clássicos, correndo o risco de perder uma sinergia bem sucedida de ingredientes.

\subsubsection*{Estilo Base}
Juízes não devem ser pedantes demais ao procurar todas características do Estilo Base da cerveja declarado. Afinal, a cerveja base normalmente não contém ingredientes especiais, então a característica sensorial não será a mesma da cerveja original. Existem interações de sabores que podem produzir efeitos sensoriais adicionais.\\
Juizes também devem entender que o processo de fermentação pode transformar alguns ingredientes (particularmente aqueles com açúcares fermentescíveis), e que a característica do ingrediente especial na cerveja pode não ser percebido da mesma forma do que o ingrediente de especialidade sozinho. Portanto, juízes devem procurar pela agradabilidade geral e o equilíbrio da combinação final, desde que a cerveja sugira tanto o Estilo Base quanto o Ingrediente de Especialidade ou processo especial.
\subsubsection*{Múltiplos Ingredientes}
Juízes não precisam sentir cada Ingrediente de Especialidade (como por exemplo especiarias) de forma individual quando diversos forem declarados. Normalmente é a combinação resultante que contribui para o caráter maior, então permita que estes ingredientes sejam usados em intensidades variadas para produzir uma experiência de degustação mais agradável.\\
Nem toda cerveja irá se enquadrar em um estilo perfeitamente. Algumas cervejas com diversos ingredientes podem ser inscritas em diversos estilos. Seja gentil ao julgar estas cervejas. Recompense aquelas cervejas que estão bem-feitas e são agradáveis de beber ao invés de repreender o participante sobre onde ele deveria ter inscrito ela.\\
Se um inscrito declarar um alergênico potencial na cerveja, não deduza pontos caso você não consiga notar sua presença.

\subsubsection*{Efeito do Ingrediente Especial no Equilíbrio}
A característica do Ingrediente de Especialidade/ Especial deve ser agradável e prover suporte, não artificial ou inapropriadamente forte, levando em conta que alguns ingredientes tem ums característica inerente muito prominente.  Lúpulos de aroma, subprodutos de fermentação e componentes de malte da cerveja base podem não ser identificáveis quando ingredientes adicionais estão presentes e também podem ser ofuscados de forma intencional para deixar que a característica do ingrediente adicionado adiciona uma complexidade extra para a cerveja, mas não seja tão prominente que desequilibre o resultado apresentado.
\end{multicols*}