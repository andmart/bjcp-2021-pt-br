\phantomsection
\subsection*{30A. Spice, Herb, or Vegetable Beer}
\addcontentsline{toc}{subsection}{30A. Spice, Herb, or Vegetable Beer}
\textit{Muitas vezes chamada apenas de Spice Beer, independentemente de serem usadas especiarias, ervas ou vegetais.}

\textit{Impressão Geral}: Uma atraente fusão entre especiarias, ervas ou vegetais (SHVs) e cerveja, mas ainda assim reconhecível como cerveja. O caráter de SHV deve estar em equilíbrio com a cerveja e ser evidente, mas não proeminente a ponto de sugerir um produto artificial.

\textit{Aparência}: Varia de acordo com o estilo base e ingredientes especiais utilizados. Cervejas de cor mais clara, incluindo o colarinho, podem apresentar cores vindas da adição. Limpidez variável, ainda que a turbidez seja geralmente não desejável. Alguns ingredientes podem impactar na retenção da espuma.

\textit{Aroma}: Varia de acordo com o estilo base. O caráter de SHV deve ser notável no aroma; no entanto, alguns SHVs (p. ex., gengibre, canela e alecrim) têm aromas mais fortes e mais distintos do que outros (p. ex., a maioria dos vegetais) – o que permite uma gama de caráter e intensidade de SHV, de sutil a agressivo. Aroma de lúpulo pode ser mais baixo do que o esperado para o estilo base, para mostrar melhor o caráter de SHV. As especiarias, ervas e vegetais devem adicionar complexidade, mas não devem estar proeminentes a ponto de desequilibrar o produto final.

\textit{Sabor}: Varia de acordo com o estilo base. Como no aroma, um distinto sabor de SHV deve ser notável e pode variar em intensidade, de sutil a agressivo. Alguns SHV são inerentemente amargos e podem resultar numa cerveja mais amarga do que o estilo base declarado. Amargor, sabores de lúpulo e malte, teor alcoólico e subprodutos de fermentação, como ésteres, devem ser apropriados para o estilo base, porém harmoniosos e equilibrados com os distintos sabores de SHVs presentes.

\textit{Sensação na Boca}: Varia de acordo com o estilo base. SHVs podem aumentar ou diminuir o corpo. Alguns SHVs podem adicionar um pouco de adstringência, ainda que um caráter de especiaria "crua" seja indesejável.

\textit{Comentários}: A descrição da cerveja é crítica para avaliação. Os juízes devem pensar mais no conceito declarado do que tentar detectar individualmente cada ingrediente utilizado. Equilíbrio, facilidade de beber e execução do tema (do estilo proposto) são os fatores mais importantes. Os SHVs devem complementar o estilo base, não sobrecarregá-lo. Os atributos do estilo base serão diferentes após a adição dos SHVs, ou seja, não espere que a cerveja tenha o sabor idêntico ao do estilo base inalterado.

\textit{Instruções para inscrição}: O participante deve especificar o tipo de especiarias, ervas ou vegetais utilizados, mas ingredientes individuais não necessitam serem especificados, caso façam parte de uma combinação conhecida de especiarias (p. ex., torta de maçã com especiarias, curry em pó, pimenta em pó). O participante deve fornecer uma descrição da cerveja, identificando o estilo base e os ingredientes, estatísticas e/ou caráter desejado para a cerveja. Uma descrição geral da natureza especial da cerveja pode abranger todos os itens necessários.

\textit{Estatísticas}: OG, FG, IBU, SRM e ABV vão variar de acordo com a cerveja base.

\textit{Exemplos Comerciais}: Alesmith Speedway Stout, Elysian Avatar Jasmine IPA, Founders Breakfast Stout, Rogue Yellow Snow Pilsner, Traquair Jacobite Ale, Young's Double Chocolate Stout.

\textit{Última Revisão}: Spice, Herb, or Vegetable Beer (2015)

\textit{Atributos de Estilo}: specialty-beer, spice
