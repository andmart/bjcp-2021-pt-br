\section*{26. Monastic Ale}
\addcontentsline{toc}{section}{26. Monastic Ale}
\textit{As instituições religiosas possuem uma longa história de produção cerveja na Bélgica, embora algumas vezes interrompida por conflitos e ocupações, como a que ocorreu durante as Guerras Napoleônicas e a I Guerra Mundial. Pouquíssimas dessas instituições produzem cervejas atualmente, embora muitas licenciaram seus nomes para cervejarias comerciais. Apesar da produção limitada, os estilos tradicionais derivados dessas cervejarias foram bem influentes e se espalharam para além da Bélgica.}

\textit{Vários termos têm sido usados para descrever essas cervejas, mas muitas são denominações protegidas e refletem a origem da cerveja e não um estilo. Esses mosteiros podem produzir qualquer estilo que escolherem, entretanto os estilos indicados nessa categoria são os estilos mais comuns associados a essa tradição cervejeira.}

\textit{Diferenciamos as cervejas desta categoria daquelas que foram inspiradas pelas cervejarias religiosas. Apesar das alegações de exclusividade, essas cervejas compartilham de vários atributos comuns que ajudam a caracterizar os estilos. São todos estilos de alta fermentação, com uma alta atenuação ("mais digerível" na Bélgica), atingem alta carbonatação através de acondicionamento em garrafa ("refermentada em garrafa" na Bélgica) e tem caráter de levedura belga distinto, complexo, com condimentado e esterificado agressivo. Muitas tem teor alcóolico alto.}