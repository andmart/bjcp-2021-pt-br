\phantomsection
\subsection*{5D. German Pils}
\addcontentsline{toc}{subsection}{5D. German Pils}
\textbf{Impressão Geral}: Uma lager alemã, clara, seca e amarga com aroma proeminente de lúpulo. Apresenta final bem definido, limpo e refrescante, exibindo uma cor dourada brilhante com excelente retenção de espuma.

\textbf{Aroma}: Lúpulo floral, picante/condimentado e/ou herbal de moderado a moderadamente alto. Presença de malte lembrando cereais, adocicado e/ou massa de pão com intensidade de baixo a médio, frequentemente com leve notas de mel e de biscoito tostado. Perfil de fermentação limpo. O lúpulo deve estar à frente, mas não dominar totalmente o malte no equilíbrio.

\textbf{Aparência}: De amarelo palha a amarelo profundo, de brilhante a muito límpida, com espuma branca, cremosa e de longa duração.

\textbf{Sabor}: O sabor inicial de malte é rapidamente superado pelo sabor e amargor do lúpulo, levando a um final seco e crisp/bem definido. Sabor de malte e lúpulo similar ao aroma (mesmos descritores e intensidades). Amargor de médio a alto, persistente no retrogosto junto com um toque de malte e lúpulo. Perfil de fermentação limpo. Água com alto teor de minerais pode acentuar e alongar o final seco. O lúpulo e o malte podem diminuir com o tempo, mas a cerveja deve ter sempre o equilíbrio tendendo ao amargor.

\textbf{Sensação na Boca}: Corpo médio-baixo. Carbonatação de média a alta. Não deve ser pesada. Sem aspereza, mas pode apresentar uma nitidez mineral em alguns exemplares.

\textbf{Comentários}: Exemplares modernos da Pils tendem a se tornar mais claros na cor, mais secos e nítidos no final e mais amargos conforme se move do sul para o norte na Alemanha, frequentemente refletindo o aumento de sulfatos na água. As Pils encontradas na Baviera tendem a ser um pouco mais suaves no amargor, com mais sabor de malte e caráter de lupulagem tardia, mas ainda com lúpulo o suficiente e final bem definido para se diferenciar da Munich Helles. O uso do termo 'Pils' é mais comum na Alemanha do que o termo 'Pilsner' para diferenciar do estilo tcheco e (algumas pessoas dizem) para mostrar respeito.

\textbf{História}: Adaptada da Pilsner Tcheca para se adequar às condições de produção cervejeira da Alemanha, particularmente à água com maior teor de minerais e variedades de lúpulo nacionais. Fabricada pela primeira vez na Alemanha no início da década de 1870. Tornou-se mais popular após a Segunda Guerra Mundial, quando as escolas cervejeiras alemãs enfatizaram as técnicas modernas. Junto com sua “prima” Czech Pilsner, é a ancestral dos estilos de cerveja mais amplamente produzidos atualmente.

\textbf{Ingredientes}: Malte Pilsner continental. Lúpulos tradicionais alemães. Levedura lager alemã limpo.

\textbf{Comparação de Estilos}: Mais leve em corpo e cor, mais seca, com final mais bem definido, mais atenuada, amargor mais persistente e carbonatação mais alta do que uma Czech Premium Pale Lager. Mais caráter de lúpulo, sabor de malte e amargor do que a International Pale Lager. Mais caráter de lúpulo e amargor, com um final mais seco e mais bem definido do que uma Munich Helles; a Helles tem mais intensidade de malte, mas com o mesmo caráter da German Pils.

\begin{tabular}{@{}p{35mm}p{35mm}@{}}
  \textbf{Estatísticas}: & OG: 1,044 - 1,050 \\
  IBU: 22 - 40  & FG: 1,008 - 1,013  \\
  SRM: 2 - 4  & ABV: 4,4\% - 5,2\%
\end{tabular}

\textbf{Exemplos Comerciais}: ABK Pils Anno 1907, Jever Pilsener, König Pilsener, Paulaner Pils, Bierstadt Slow-Pour Pils, Rothaus Pils, Schönramer Pils, Trumer Pils.

\textbf{Última revisão}: German Pils (2015)

\textbf{Atributos de Estilo}: bitter, bottom-fermented, central-europe, hoppy, lagered, pale-color, pilsner-family, standard-strength, traditional-style
