\phantomsection
\subsection*{22A. Double IPA}
\addcontentsline{toc}{subsection}{22A. Double IPA}

\textbf{Impressão Geral}: Uma ale clara com teor alcoólico mais alto, amarga e intensamente lupulada, sem o dulçor residual, corpo e a rica maltosidade complexa de um American Barleywine. Fortemente lupulada, mas límpida, seca e sem aspereza. Apesar de mostrar sua força, a facilidade em beber é um ponto importante.

\textbf{Aparência}: De dourada a laranja-acobreada clara, mas a maioria das versões modernas é bastante clara. Boa limpidez, embora um pouco de turbidez seja aceitável. Colarinho de tamanho moderado, persistente, de quase branco e branco.

\textbf{Aroma}: Aroma de lúpulo de proeminente a intenso, tipicamente apresentando características modernas de lúpulos americanos ou do novo mundo, como cítrico, floral, pinho, resinoso, condimentado, frutas tropicais, frutas de caroço, frutas vermelhas ou melão. O malte pode vir como suporte, em segundo plano, limpo, de neutro a como cereais. Perfil de fermentação neutro a levemente frutado. O álcool pode ser notado, mas não deve ser similar a solvente.

\textbf{Sabor}: Sabor de lúpulo forte e complexo (mesmos descritores do aroma). Amargor de moderadamente alto a muito alto, mas não deve ser áspero. Suporte do malte de baixo a médio, limpo, suave e discreto; pode ter sabores leves de caramelo ou tostado. Final de seco a médio-seco, não pode ser doce ou pesado, com um retrogosto persistente de lúpulo e amargor. Frutado baixo a moderado opcional. Um sabor de álcool leve, limpo e suave é permitido.

\textbf{Sensação na Boca}: Corpo de médio leve a médio, com uma textura macia. Carbonatação de média a média alta. Sem adstringência áspera vinda do lúpulo. Aquecimento alcoólico suave e contido é aceitável.

\textbf{Comentários}: Raramente chamada de Imperial IPA. Muitas versões modernas têm várias adições de dry-hop.

\textbf{História}: Uma inovação de cerveja artesanal americana desenvolvida pela primeira vez em meados da década de 1990 como uma versão mais intensa da American IPA. Tornou-se mais dominante e popular ao longo dos anos 2000 e inspirou a criatividade adicional das IPAs. Russian River Pliny the Elder, fabricada pela primeira vez em 2000, ajudou a popularizar o estilo.

\textbf{Ingredientes}: Malte base neutro. É comum o uso de adjuntos de açúcar. Maltes crystal são raros. Lúpulo americano ou do novo mundo. Levedura neutra ou levemente frutada. Sem carvalho.

\textbf{Comparação de Estilos}: Maior teor alcoólico, amargor e lúpulo do que as IPAs inglesas e americanas. Menos rica em malte, menos corpo, mais seca e com um maior equilíbrio para o lúpulo do que uma American Barleywine.

\begin{tabular}{@{}p{35mm}p{35mm}@{}}
  \textbf{Estatísticas} & OG: 1,065 - 1,085  \\
  IBU: 60 - 100  & FG: 1,008 - 1,018   \\
  SRM: 6 - 14 & ABV: 7,5\% - 10\%
\end{tabular}

\textbf{Exemplos Comerciais}: Columbus Brewing Bohdi, Fat Heads Hop Juju, Port Brewing Hop 15, Russian River Pliny the Elder, Stone Ruination Double IPA 2,0, Wicked Weed Freak of Nature.

\textbf{Última Revisão}: Double IPA (2015)

\textbf{Atributos de estilo}: bitter, craft-style, hoppy, ipa-family, north-america, pale-color, top-fermented, very-high-strength
