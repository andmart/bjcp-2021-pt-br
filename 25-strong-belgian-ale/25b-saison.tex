\phantomsection
\subsection*{25B. Saison}
\addcontentsline{toc}{subsection}{25B. Saison}
\textbf{Impressão Geral}: Uma família de cervejas belgas refrescantes, altamente atenuadas, lupuladas e bastante amargas com alta carbonatação e um final bem seco. Caracterizada por um perfil de fermentação frutado, condimentado e, às vezes, fenólico. É usado grãos de cereais e, às vezes, especiarias para dar complexidade. Existem variações de cor e de força alcoólica.

\textbf{Aparência}: Cor de dourado claro a âmbar profundo, algumas vezes laranja claro. Colarinho de longa duração, denso, com cor de branco a marfim. "Rendado belga". Não filtrada, então a limpidez é variável (de baixa a boa) podendo ser turva. Efervescente. Versões escuras podem ser de cobre a marrom escuro. Versões com teor alcoólico mais alto podem ter coloração mais escura.

\textbf{Aroma}: Uma mistura agradavelmente aromática de condimentado e frutado oriundos dos lúpulos e da levedura. Os ésteres frutados são de moderados a altos e geralmente tem um perfil de fruta cítrica, maçã, pera, marmelo ou frutas de caroço. Especiarias de baixo a moderadamente alto são encontradas como pimenta preta, mas não como cravo. Os lúpulos são de baixos a moderados e possuem características de perfil continental (condimentado, floral, terroso ou frutado). O malte é ofuscado, mas se percebido deve ter perfil levemente como cereais. Especiarias e ervas são opcionais, mas não devem ser dominantes. Acidez é opcional (ver Comentários). Versões com teor alcoólico mais alto têm mais intensidade aromática e pode adicionar um leve caráter alcoólico e um maltado moderado. Versões de mesa são menos intensas e não têm caráter alcoólico. Versões escuras adicionam caráter de malte associados com grãos escuros.

\textbf{Sabor}: Um equilíbrio levedura frutada e condimentada, o amargor do lúpulo e o maltado oriundo dos grãos com moderado a alto amargor e um final muito seco. Os aspectos frutados e condimentados variam de médio baixo a médio alto e o sabor do lúpulo varia de baixo a médio, ambos com características semelhantes ao do aroma (são aplicados os mesmos descritores). Na boca, o malte é suave com perfil de cereais e varia de baixo a médio. A atenuação é muito alta, nunca com um final doce ou pesado. Retrogosto condimentado, amargo. Especiarias e ervas são opcionais, mas quando utilizados devem estar em harmonia com a levedura. Acidez é opcional (veja Comentários). Versões escuras terão mais caráter de malte, incluindo os sabores oriundos dos maltes escuros. Versões com teor alcoólico mais alto terão mais intensidade de malte e leve nota alcoólica.

\textbf{Sensação na Boca}: Corpo de leve a médio leve. Carbonatação são muito altas. Efervescente. Leve aquecimento alcoólico é opcional. Acidez é rara e opcional (veja comentários). Versões com teor alcoólico mais alto podem ter até corpo médio algum aquecimento alcoólico. As versões de mesa não possuem aquecimento.

\textbf{Comentários}: Este estilo geralmente descreve a versão clara e de força standart (padrão), seguida pelas variações de cor e força alcoólica. Versões escuras tendem a ter mais características de malte e menos amargor aparente de lúpulo, o que proporciona uma apresentação mais equilibrada. Versões com teor alcoólico mais alto muitas vezes tem mais sabor de malte, riqueza, aquecimento e corpo, simplesmente por causa da maior densidade. Não há correlação entre força alcoólica e cor. A acidez é totalmente opcional e, se estiver presente, seu nível deve ser de baixo a moderado, podendo substituir alguma coisa do amargor no equilíbrio. Uma Saison não deve ser ácida e amarga ao mesmo tempo. A alta atenuação pode fazer parecer que a cerveja seja mais amarga do que o IBU sugerem. Versões claras costumam ser um pouco mais lupuladas e amargas do que as versões escuras. A seleção de leveduras conduz para o equilíbrio entre notas de frutado e condimentado, podendo mudar significativamente de característica, o que permite uma faixa de interpretações. Muitas vezes é chamada de Farmhouse ales nos Estados Unidos, porém esse termo não é tão comum na Europa, onde elas seriam apenas parte de um grupo maior de cervejas artesanais. Brettanomyces não é típica para este estilo. Saison com Brett devem ser inscritos no estilo 28A Brett Beer. Uma Grisette é um tipo de Saison bem conhecido pelos mineiros (que trabalham em minas de carvão). Inscrever Grisette como 25B Saison, com força Session, Informações: Grisette com trigo com caráter de grão.

\textbf{História}: Uma cerveja de abastecimento oriunda da Valônia, região belga onde se fala o francês. Originariamente era um produto de baixo teor alcoólico para não debilitar os trabalhadores do campo e das fazendas, mas também existiam os produtos com força de taverna. A mais conhecida Saison moderna, Saison Dupont, foi produzida pela primeira vez na década de 1920. A 'super' Saison (Blonde) da Dupont foi produzida pela primeira vez em 1954 e a versão Brune em meados da década de 1980. A cervejaria Fantôme começou a produzir suas Saison "sazonais" em 1988. Apesar do estilo manter uma imagem rústica, elas agora são produzidas principalmente em grandes cervejarias.

\textbf{Ingredientes}: Malte pale como base. Cereais, como trigo, aveia, espelta ou centeio. Pode conter adjuntos açucarados. Lúpulos Continental. Levedura belga para Saison com características de frutado-condimentadas. Especiarias e ervas são incomuns, mas são permitidas desde que não sejam dominantes.

\textbf{Comparação de Estilos}: As versões claras e com força alcoólica standart são como o estilo Belgian Blond, porém mais amargas, mais lupuladas e mais bem atenuadas e um caráter de levedura mais forte. Já as claras com força alcoólica 'super' são semelhantes ao estilo Belgian Tripel, mas, muitas vezes, com qualidade rústica e como cereais, já em outras com condimentado oriundo da levedura.

\textbf{Instruções para Inscrição}: O participante deve especificar a força alcoólica (table, standart ou super) e a cor (clara ou escura). O participante pode identificar o caráter de cerais usados.

\textbf{Estatísticas}:

IBU: 20 - 35
SRM: 5 – 14 (clara)
\leftskip9mm15 – 22 (escura)
\leftskip0mmOG: 1,048 – 1,065 (densidade padrão)
FG: 1,002 – 1,008 (densidade padrão)
ABV: 3,5 – 5,0\% (teor alcoólico leve)
\leftskip9mm5,0 – 7,0\% (teor alcoólico padrão)
\leftskip9mm7,0 – 9,5\% (teor alcoólico super)\\
\leftskip0mm\textbf{Exempleo Comerciais}: Ellezelloise Saison 2000, Lefebvre Saison 1900, Saison Dupont, Saison de Pipaix, Saison Voisin, Boulevard Tank 7 Farmhouse Ale. \\
\textbf{Última Revisão}: Saison (2015)

\textbf{Atributos de Estilo}: bitter, pale-color, standard-strength, top-fermented, traditional-style, western-europe
