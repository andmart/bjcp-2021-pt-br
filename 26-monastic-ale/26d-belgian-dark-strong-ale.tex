\phantomsection
\subsection*{26D. Belgian Dark Strong Ale}
\addcontentsline{toc}{subsection}{26D. Belgian Dark Strong Ale}
\textbf{Impressão Geral}: Uma ale belga escura, complexa, bem forte, com uma mistura rica de maltes, sabores de frutas escuras e notas condimentadas. Complexa, rica, macia e perigosa.

\textbf{Aparência}: Cor de âmbar profundo a marrom acobreado profundo ('escuro' no nome do estilo significa mais escuro do que dourado, mas não 'preta'). Colarinho castanho claro, alto, denso, com consistência de uma mousse e creme persistente. Normalmente límpida.

\textbf{Aroma}: Uma mistura complexa e bastante intensa de rico maltado e frutado profundo, acentuado por fenólico condimentado e álcool. O caráter de malte é de moderadamente alto a alto e possui uma base de pão tostado profundo com notas de caramelo escuro, mas sem a impressão de uso de malte preto ou torrado. Ésteres são de fortes a moderadamente baixos e remetem a passas, ameixas, cerejas secas, figos, tâmaras ou ameixas. Fenólicos condimentados com intensidade de baixa a moderados, em que a pimenta preta ou a baunilha podem estar presentes em segundo plano, exceto cravo. Álcool é de baixo a moderado, mas macio, que remete a condimentado, perfumado ou aroma de rosas, mas nunca quente ou que remeta a solvente. Lúpulos geralmente não são perceptíveis, mas, quando presentes, pode adicionar um leve caráter condimentado, floral ou herbal.

\textbf{Sabor}: Maltado rico e complexo, mas sem ser pesado no final. O sabor possui caráter similar ao do aroma (os mesmos comentários sobre malte, éster, fenol, álcool e lúpulo são aplicados aqui). Na boca, o maltado rico é moderado, o que pode passar uma impressão adocicada se o amargor for baixo. Normalmente é de moderadamente seco a seco no final, embora possa ser até moderadamente doce. Amargor de médio-baixo a moderado; álcool fornece um pouco do equilíbrio para o malte. Geralmente, o equilíbrio é para um rico maltado, mas pode conter bastante amargor. A complexa variedade de sabores que devem se misturar de forma macia e harmoniosa e muitas vezes se beneficiam com a idade. O final não deve ser pesado ou similar a xarope.

\textbf{Sensação na Boca}: Alta carbonatação, mas sem ser agudo. Suave, mas perceptível aquecimento alcoólico. O corpo fica na faixa de médio-baixo a médio-alto e cremoso. A maioria é de corpo médio.

\textbf{Comentários}: Também conhecido como Belgian Quad, principalmente fora da Bélgica (Quadruple é o nome de uma cerveja específica). Possui uma gama de interpretação mais ampla do que muitos outros estilos belgas. As versões tradicionais tendem a ser mais secas do que as versões comerciais modernas, que podem ser mais doces e encorpadas. Muitos exemplos são simplesmente conhecidos por sua força alcoólica ou por designação de cor. Alguns são rotulados como Grand Cru, mas isso tem mais a ver com a qualidade da cerveja do que com o estilo.

\textbf{História}: Westvleteren começou a fabricar sua versão pouco antes da II Guerra Mundial, ao passo que a Chimay e a Rochefort fizeram suas versões logo após. Outras cervejarias monásticas criaram seus produtos no final do século 20, mas algumas cervejarias centenárias começaram a produzir produtos similares em 1960.

\textbf{Ingredientes Caraterísticos}: Levedura belga que produz subprodutos condimentados e frutados. Uma complexa variedade de grãos, embora algumas versões tradicionais sejam bem simples, com xarope de açúcar caramelizado ou açúcares não refinados e a levedura fornecendo grande parte da complexidade. Lúpulos continentais. Especiarias não são típicas, se presente, devem ser sutis.

\textbf{Comparação de Estilos}: Como uma grande Belgian Dubbel, com um corpo mais cheio e maior riqueza maltada. Não é mais amarga ou lupulada como uma Belgian Tripel, mas com força alcoólica semelhante.

\begin{tabular}{@{}p{35mm}p{35mm}@{}}
  \textbf{Estatísticas}: & OG: 1,075 - 1,110 \\
  IBU: 20 - 35  & FG: 1,010 - 1,024  \\
  SRM: 12 - 22  & ABV: 8\% - 12\%
\end{tabular}

\textbf{Exemplos Comerciais}: Achel Extra Bruin, Boulevard The Sixth Glass, Chimay Blue, Rochefort 10, St. Bernardus Abt 12, Westvleteren 12.

\textbf{Última Revisão}: Belgian Dark Strong Ale (2015)

\textbf{Atributos de Estilo}: amber-color, malty, top-fermented, traditional-style, very-high-strength, western-europe
