\phantomsection
\subsection*{23B. Flanders Red Ale}
\addcontentsline{toc}{subsection}{23B. Flanders Red Ale}
\textbf{Impressão Geral}: Uma ale do estilo belga ácida e frutada, marrom-avermelhada envelhecida em carvalho com sabores de malte tostados servindo como suporte e complexidade de fruta. O final seco e tânico reforça a sugestão de um vinho tinto \textit{vintage} (de uma safra de excepcional qualidade).

\textbf{Aparência}: De cor vermelho profundo, bordô, a marrom-avermelhado. Boa limpidez. Colarinho, na cor, de branco a castanho muito claro. Média a boa retenção de espuma.

\textbf{Aroma}: Perfil ácido-frutado complexo com o malte fornecendo suporte. Frutuosidade é alta e remete a cerejas pretas, laranjas, ameixas, groselhas ou \textit{fruit leather} (polpas de frutas que são espalhadas e secas em formato de tiras). Baunilha, chocolate ou fenol condimentado de baixo a médio-baixo podem estar presentes para complexidade. O aroma ácido pode variar de moderado a alto. Um caráter avinagrado dominante é inapropriado, embora de níveis baixos a moderados de ácido acético sejam aceitáveis desde que equilibrados com o malte. Sem aroma de lúpulo.

\textbf{Sabor}: Sabores maltados de moderados a moderadamente altos em geral apresentam uma qualidade rica-tostada. Sabores intensos de frutas, sendo os descritores os mesmos do aroma. Acidez complexa de moderada a alta, acentuada pelos esteres; não deve ser uma simples acidez lática. Um caráter avinagrado dominante é inapropriado, embora de níveis baixos a moderados de ácido acético sejam aceitáveis desde que equilibrados com o malte. Geralmente com o aumento do caráter ácido o malte reduz gradualmente para um sabor mais de segundo plano (e vice-versa). Baunilha, chocolate ou fenol condimentado de baixo a médio-baixo. Sem aroma de lúpulo. Amargor contido com o equilíbrio para o lado do malte. Ácidos e taninos podem realçar a percepção de amargor e prover equilíbrio e estrutura. Algumas versões são adoçadas ou combinadas para serem doce; permita uma ampla variedade nos níveis de dulçor, que podem suavizar a acidez picante e a percepção acética.

\textbf{Sensação ne Boca}: Corpo médio, frequentemente realçado por taninos. Carbonatação baixa a média-baixa. Baixa a média adstringência, geralmente com uma acidez que causa um sutil formigamento na língua. No paladar é enganosamente leve e com final bem definido (\textit{crisp}), embora um final meio adocicado não seja incomum.

\textbf{Comentários}: A observação "como vinho" não deve ser levada muito literalmente; ela pode sugerir um vinho francês da Borgonha de alta acidez, mas claramente não é idêntica. Produzida por envelhecimento extenso (até dois anos) em imensos barris de madeira (\textit{foeders}), combinando cervejas jovens e bem maturadas e adoçantes em quantidades variáveis na finalização do produto. Uma alta gama de produtos são possíveis dependendo da real combinação e se há utilização de adoçantes. Sabores acéticos podem ser notados, mas nem toda acidez nesta cerveja é derivada de ácido acético; vinagre supera em mais de seis vezes a acidez total deste estilo. Versões com adição de fruta devem ser inscritas como uma 29A Fruit Beer.

\textbf{História}: Uma cerveja nativa da Flandres Oriental, representada pelos produtos da cervejaria Rodenbach, estabelecida em 1821. O envelhecimento em imensos barris de madeira e a combinação de cervejas jovens e envelhecidas foram emprestados da tradição inglesa. Os cervejeiros belgas consideram a Flanders Red e a Oud Bruin da mesma família de estilos, mas a primeira diferenciação foi feita quando Michael Jackson inicialmente definiu os estilos de cerveja, dado que os perfis de sabor são claramente diferentes. Muitos exemplares modernos são influenciados pela popularidade da Rodenbach Grand Cru.

\textbf{Ingredientes}: Maltes Vienna ou Munique, uma variedade de maltes caramelo e milho. Lúpulos da Europa continental de baixo alfa ácido. Sacch, Lacto e Bretta. Envelhecida em carvalho. Por vezes combinada e adoçada (naturalmente ou artificialmente).

\textbf{Comparação de Estilo}: Menos rica em sabores maltados que uma Oud Bruin, geralmente mais voltada para um perfil azedo-frutado e acético.

\begin{tabular}{@{}p{35mm}p{35mm}@{}}
  \textbf{Estatísticas}: & OG: 1,048 - 1,057 \\
  IBU: 10 - 25  & FG: 1,002 - 1,012  \\
  SRM: 10 - 17  & ABV: 4,6\% - 6,5\%
\end{tabular}

\textbf{Exemplos Comerciais}: Cuvée des Jacobins Rouge, Duchesse de Bourgogne, New Belgium La Folie, Rodenbach Classic, Rodenbach Grand Cru, Vichtenaar Flemish Ale.

\textbf{Última Revisão}: Flanders Red Ale (2015)

\textbf{Atributos de Estilo}: amber-color, balanced, sour, sour-ale-family, standard-strength, top-fermented, traditional-style, western-europe, wood
