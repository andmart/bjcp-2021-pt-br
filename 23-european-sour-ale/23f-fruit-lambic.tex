\phantomsection
\subsection*{23F. Fruit Lambic}
\addcontentsline{toc}{subsection}{23F. Fruit Lambic}
\textbf{Impressão Geral}: Uma cerveja de trigo belga, complexa, refrescante e agradavelmente ácida que combina um caráter complementar de frutas fermentadas com acidez \textit{funky} de uma Gueuze.

\textbf{Aroma}: A fruta especificada deve ser o aroma dominante, misturando-se bem com os aromas similares da Gueuze (a mesma descrição se aplica, mas com a adição de um caráter de fruta fermentada).

\textbf{Aparência}: Como uma Gueuze, mas modificada pela cor da fruta utilizada, perdendo intensidade com o tempo. A limpidez costuma ser boa, embora algumas frutas não sedimentem o suficiente para torná-la límpida. Se altamente carbonatada da maneira tradicional terá uma espuma grossa, densa, geralmente de longa duração, semelhante a mousse, às vezes com uma tonalidade que reflete a fruta adicionada.

\textbf{Sabor}: Combina o perfil de sabor de uma Gueuze (aplica-se a mesma descrição) com contribuições perceptíveis do sabor da fruta adicionada. As versões tradicionais são secas e ácidas, com sabor de frutas fermentadas. As versões modernas podem ter uma doçura variável, que pode compensar a acidez. Os sabores de frutas também desaparecem com o tempo e perdem sua vitalidade, portanto, podem ser de baixa a alta intensidade.

\textbf{Sensação na Boca}: Corpo baixo a médio-baixo; não deve ser aguada. Tem uma acidez de baixa a alta intensidade, trazendo a capacidade de franzir o rosto sem ser acentuadamente adstringente. Algumas versões têm um leve caráter de aquecimento. A carbonatação pode variar de efervescente a quase sem gás.

\textbf{Comentários}: Produzida como Gueuze, com a fruta comumente adicionada no meio do envelhecimento, para que as leveduras e as bactérias possam fermentar todos os açúcares da fruta; ou adicionando frutas a uma Lambic, que é menos comum. Às vezes, a variedade de frutas pode ser difícil de identificar, pois as frutas fermentadas e envelhecidas costumam ser percebidas de maneira diferente das frutas frescas, mais reconhecíveis. A fruta pode trazer acidez e taninos, além de sabor e aroma; entender o caráter da fruta fermentada ajuda a julgar o estilo.

\textbf{História}: Mesma história básica da Gueuze, incluindo a tendência recente de adoçar, mas com frutas além do açúcar. A fruta era tradicionalmente adicionada pelo responsável por fazer a mistura ou pelo taberneiro para aumentar a variedade de cervejas disponíveis nos cafés locais.

\textbf{Ingredientes}: Mesma base da Gueuze. Fruta adicionada aos barris durante a fermentação e a mistura. Frutas tradicionais incluem cerejas ácidas e framboesas; frutas modernas incluem pêssegos, damascos, uvas e outros. Pode usar adoçantes naturais ou artificiais.

\textbf{Comparação de Estilos}: Uma Gueuze com frutas, não apenas uma Fruit Beer ácida; o caráter selvagem deve ser evidente.

\textbf{Instruções para Inscrição}: O tipo de fruta utilizada deve ser especificado. O cervejeiro deve declarar um nível de carbonatação (baixo, médio, alto) e um nível de dulçor (nenhum/baixo, médio, alto).

\begin{tabular}{@{}p{35mm}p{35mm}@{}}
  \textbf{Estatísticas}: & OG: 1,040 - 1,060 \\
  IBU: 0 - 10  & FG: 1,000 - 1,010  \\
  SRM: 3 - 7  & ABV: 5\% - 7\%
\end{tabular}

\textbf{Exemplos Comerciais}: 3 Fonteinen Schaerbeekse Kriek, Cantillon Fou’ Foune, Cantillon Lou Pepe Framboise, Cantillon Vigneronne, Hanssens Oude Kriek, Oude Kriek Boon.

\textbf{Última Revisão}: Fruit Lambic (2015)

\textbf{Atributos de Estilo}: fruit, pale-color, sour, standard-strength, traditional-style, western-europe, wheat-beer-family, wild-fermented
