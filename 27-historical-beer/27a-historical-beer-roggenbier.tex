\phantomsection
\subsection*{27A. Historical Beer: Roggenbier}
\addcontentsline{toc}{subsection}{27A. Historical Beer: Roggenbier}

\textbf{Impressão Geral}: Uma Dunkles Weissbier feita com centeio ao invés de trigo, mas com um corpo mais alto e menos lúpulos de finalização. O centeio dá um sabor de pão e apimentado, um corpo mais cremoso e um final mais seco e como cereais, que combina com o característico banana e cravo da levedura Weizen.

\textbf{Aparência}: Cor de cobre alaranjado claro a um cobre amarronzado ou avermelhado escuro. Colarinho alto e cremoso de cor quase branca até bronzeado, bastante denso e persistente e, muitas vezes, espesso e firme.

\textbf{Aroma}: Aroma condimentado de centeio de baixo a moderado (como pimenta preta) mesclado com aromáticos de levedura Weizen (condimentado como cravo e ésteres frutados, banana ou cítrico) de baixo a moderado. Lúpulo condimentado, floral ou de ervas de intensidade baixa é aceitável.

\textbf{Sabor}: Sabor de cereais e centeio como condimentado ou apimentado de moderadamente baixo a moderadamente alto, muitas vezes tendo um sabor terroso lembrando pão de centeio ou centeio integral. Amargor de médio a médio baixo permite um dulçor inicial de malte (às vezes com um pouco de caramelo) que pode ser saboreado antes do caráter de levedura e apimentado de centeio assumam o controle. Caráter de banana e cravo de levedura Weizen de baixo a moderado, embora o equilíbrio possa variar. Final meio seco, como cereais com um retrogosto levemente amargo (do centeio). Lúpulo condimentado, de ervas ou floral de baixo a moderado é aceitável e pode persistir no retrogosto.

\textbf{Sensação na Boca}: Corpo de médio a médio alto. Carbontação alta. Cremosidade moderada.

\textbf{Comentários}: Centeio é um grão sem casca e dificulta a mostura, muitas vezes resultando em um mosto com textura pegajosa propensa a ficar pegajosa. O centeio tem sido caracterizado como o cereal com o sabor mais assertivo. Não é apropriado adicionar semente de cominho como alguns cervejeiros norte-americanos fazem; o caráter condimentado do centeio é tradicionalmente vindo apenas dos grãos.

\textbf{História}: Uma cerveja de centeio alemã especial fabricada em Regensburg, na Baviera, em 1988, na cervejaria Schierlinger. Depois da eventual compra pela Paulaner a cerveja hoje é posicionada no mercado como uma marca regional e por isso é difícil de ser encontrada.

\textbf{Ingredientes}: Até 60\% de centeio malteado na lista de grãos. Maltes Pale e de trigo. Maltes tipo cristal e maltes escuro sem casca podem ser usados. Levedura Weizen. Lúpulos alemães ou tchecos. Mostura por decocção usando um processo patenteado.

\textbf{Comparação de Estilo}: Uma variante mais distinta de uma Dunkles Weissbier usando centeio malteado no lugar de trigo malteado. Cervejas de centeio norte americanas não terão o caráter de levedura weizen e terão mais lúpulo.

\begin{tabular}{@{}p{35mm}p{35mm}@{}}
  \textbf{Estatísticas} & OG: 1,010 - 1,014  \\
  IBU: 10 - 20  & FG: 1,010 - 1,014  \\
  SRM: 14 - 19 & ABV: 4,5\% - 6\%
\end{tabular}

\textbf{Exemplos Comerciais}: Thurn und Taxis Roggen.

\textbf{Última Revisão}: Historical Beer: Roggenbier (2015)

\textbf{Atributos de Estilo}: amber-color, central-europe, historical-style, standard-strength, top-fermented, wheat-beer-family
