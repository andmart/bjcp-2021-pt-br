\phantomsection
\subsection*{10A. Weissbier}
\addcontentsline{toc}{subsection}{10A. Weissbier}

\textbf{Impressão Geral}: Uma cerveja alemã de trigo clara, refrescante e levemente lupulada com alta carbonatação, final seco, sensação macia na boca e um característico perfil de banana e cravo da fermentação com levedura weizen.

\textbf{Aparência}: Cor de palha claro a dourado. Colarinho branco duradouro, muito denso, como mousse. Pode ser turva e ter um brilho do trigo e levedura, embora isso possa se depositar no fundo das garrafas.

\textbf{Aroma}: Ésteres e fenóis de moderado a forte, tipicamente banana e cravo, geralmente bem equilibrados e geralmente mais fortes do que o malte. Aroma de leve a moderado lembrando pão, massa de pão e/ou cereais como trigo. Leve toque de baunilha é opcional. Lúpulo floral, condimentado e/ou herbal de intensidade leve é opcional. Bubblegum/Tuttifrutti (morango com banana), acidez e defumado são falhas.

\textbf{Sabor}: Sabor de banana e cravo de baixo a moderadamente forte, geralmente bem equilibrado. Sabor de baixo a moderadamente baixo lembrando pão, massa de pão e/ou de cereais lembrando trigo, suportado por leve dulçor de cereias do malte Pils. Amargor de muito baixo a moderadamente baixo. Na boca é bem equilibrada e saborosa com um final relativamente seco. Leve baunilha é opcional. Sabor muito baixo de lúpulo floral, condimentado e/ou herbal é opcional. Qualquer impressão de dulçor se deve mais ao baixo amargor do que a qualquer dulçor residual; um final doce ou pesado prejudica a drinkability. Bubblegum/Tuttifrutti, acidez e defumado são falhas. Embora o perfil de banana e cravo seja importante, ele não deve ser tão forte a ponto de ser extremo e desequilibrado.

\textbf{Sensação na Boca}: Corpo de médio-baixo a médio, nunca é pesado. Sensação cheia macia e cremosa progredindo até um final leve e borbulhante, auxiliado por uma carbonatação de alta a muito alta. Efervescente.

\textbf{Comentários}: Também conhecida como \textit{hefeweizen} ou \textit{weizenbier}, principalmente fora da Baviera. Essas cervejas são melhores desfrutadas jovens e frescas, já que não costumam envelhecer bem. Na Alemanha, versão com menor teor alcoólico light (leicht) e não alcoólicas são populares. Versões \textit{Kristall} são filtradas e tem limpidez brilhante.

\textbf{História}: Enquanto a Baviera tem tradição em cervejas de trigo datando de antes do século 16, brassar cerveja de trigo costumava ser um monopólio reservado para a realeza bávara. Weissbiers modernas datam de 1872 quando a Schneider começou a produzir sua versão âmbar. Entretanto, Weissbier claras só se tornaram populares a partir de 1960 (embora o nome historicamente pudesse ser usado na Alemanha para descrever uma cerveja feita com malte seco usando ar, uma tradição diferente). É bem popular hoje em dia, particularmente no sul da Alemanha.

\textbf{Ingredientes}: Malte de trigo, pelo menos metade do perfil de maltes. Malte Pilsner. Mostura por decocção é tradicional. Levedura weizen, temperatura baixa de fermentação.

\textbf{Comparação de Estilo}: Comparada com a American Wheat, tem perfil de banana e cravo da levedura e menos amargor. Comparada com a Dunkles Weissbier, tem cor mais clara e menos riqueza de malte e sabor.

\textbf{Instruções para Inscrição}: O participante pode especificar se a levedura deve ser agitada antes de servir.

\begin{tabular}{@{}p{35mm}p{35mm}@{}}
  \textbf{Estatísticas} & OG: 1,044 - 1,053 \\
  IBU: 8 - 15 & FG: 1,008 - 1,014 \\
  SRM: 2 - 6 & ABV: 4,3\% - 5,6\%
\end{tabular}

\textbf{Exemplos Comerciais}: Ayinger Bräu Weisse, Distelhäuser Hell Weizen, Hacker-Pschorr Hefeweißbier, Hofbräuhaus Münchner Weisse, Schneider Weisse Original Weissbier, Weihenstephaner Hefeweissbier.

\textbf{Última Revisão}: Weissbier (2015)

\textbf{Atributos de Estilo}: central-europe, malty, pale-color, standard-strength, top-fermented, traditional-style, wheat-beer-family.
