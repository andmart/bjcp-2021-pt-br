\phantomsection
\subsection*{24A. Witbier}
\addcontentsline{toc}{subsection}{24A. Witbier}
\textbf{Impressão Geral}: Uma cerveja de trigo belga clara e turva, com especiarias que acentuam o caráter de levedura. Uma ale delicada, levemente picante e de força moderada. Witbier é uma bebida refrescante de verão com alta carbonatação, final seco e leve lupulagem.

\textbf{Aparência}: De cor palha muito clara a amarelo profundo. A cerveja ficará muito turva devido turbidez causada pelo amido ou levedura, o que lhe confere um brilho leitoso amarelo-esbranquiçado. Colarinho denso, branco e como mousse. A retenção do colarinho deve ser muito boa.

\textbf{Aroma}: Malte moderado como pão, geralmente com leves notas de mel ou baunilha. Aroma de trigo levemente cereal e picante. Moderado perfume de coentro, cítrico; muitas vezes com uma nota complexa de ervas, especiarias ou apimentada no fundo. Moderado aroma frutado de casca cítrico-laranja. Um baixo aroma de lúpulo condimentado e herbal é opcional, mas geralmente ausente. As especiarias devem se misturar com aromas frutados, florais e doces e não devem ser excessivamente fortes.

\textbf{Sabor}: Sabor agradável de malte de pão e grãos, muitas vezes com um caráter de mel ou baunilha. Moderado sabor frutado cítrico de casca de laranja. Sabores condimentados de ervas, que podem incluir coentro e outras especiarias, são comuns e devem ser sutis e equilibrados, não dominantes. Um sabor de lúpulo condimentado e terroso pode ser de baixo a nenhum e nunca ofusca as especiarias. O amargor do lúpulo é de baixo a médio-baixo e suporta os sabores refrescantes de frutas e especiarias. Refrescantemente com um final seco e bem definido, sem retrogosto amargo ou áspero.

\textbf{Sensação na Boca}: Corpo médio-leve a médio, muitas vezes apresentando maciez e leve cremosidade. Caráter efervescente vindo da alta carbonatação. Refrescância gerada pela carbonatação, secura e falta de amargor no final. Sem aspereza ou adstringência. Não deve ser excessivamente seca ou aguada, nem deve ser espessa e pesada.

\textbf{Comentários}: Versões históricas podem ter alguma acidez láctica, mas isso está ausente nas versões modernas. As especiarias podem ter alguma variação, mas não devem ser exageradas. Coentro de certa origem ou idade pode dar um caráter inadequado de presunto ou aipo. A cerveja tende a ser perecível, então exemplos mais jovens, frescos e bem manuseados são os mais desejáveis. Uma impressão de doçura geralmente se deve ao baixo amargor, não ao açúcar residual. A maioria dos exemplos parece ter aproximadamente 5\% ABV.

\textbf{História}: Uma cerveja de um grupo de cervejas brancas belgas medievais da área de Leuven, que se extinguiu em 1957 e mais tarde foi revivida por Pierre Celis em 1966, que se tornou a Hoegaarden. Depois que a Hoegaarden foi adquirida pela Interbrew, o estilo cresceu rapidamente e inspirou muitos produtos similares que são rastreáveis à recriação do estilo por Celis, não aqueles de séculos passados.

\textbf{Ingredientes}: Trigo não maltado (30-60\%), o restante de malte de cevada com baixa cor. Algumas versões usam até 5-10\% de aveia crua ou outros grãos de cereais não maltados. Tradicionalmente usa sementes de coentro e casca seca de laranja Curaçao. Há rumores de que outras especiarias secretas são usadas em algumas versões, assim como as cascas de laranja doce. Levedura ale belga levemente frutada e picante.

\textbf{Comparação de Estilos}: Baixo nível de amargor com equilíbrio semelhante ao de uma Weissbier, mas com especiarias e caráter cítrico provenientes de adições mais do que da levedura.

\begin{tabular}{@{}p{35mm}p{35mm}@{}}
  \textbf{Estatísticas}: & OG: 1,044 - 1,052 \\
  IBU: 8 - 20  & FG: 1,008 - 1,012  \\
  SRM: 2 - 4  & ABV: 4,5\% - 5,5\%
\end{tabular}

\textbf{Exemplos Comerciais}: Allagash White, Blanche de Bruxelles, Celis White, Hoegaarden Wit, Ommegang Witte, St. Bernardus Witbier.

\textbf{Última Revisão}: Witbier (2015)

\textbf{Atributos de Estilo}: pale-color, spice, standard-strength, top-fermented, traditional-style, western-europe, wheat-beer-family
