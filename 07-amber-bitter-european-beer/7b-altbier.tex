\phantomsection
\subsection*{7B. Altbier}
\addcontentsline{toc}{subsection}{7B. Altbier}
\textit{Uma cerveja de Düsseldorf, de alta fermentação e acondicionada a frio que tem um paladar mais limpo e suave do que é típico na maioria das ales. “Alt” refere-se ao estilo “antigo” de fabricação de cerveja (usando levedura de alta fermentação) que era comum antes da lager de baixa fermentação se tornar popular.}

\textbf{Impressão Geral}: Uma cerveja amarga, de cor moderada, bem atenuada e com uma riqueza de malte equilibrando o forte amargor. Caráter leve e condimentado de lúpulo complementa o malte. Uma cerveja seca, com corpo firme e suave na boca.

\textbf{Aparência}: A cor varia de âmbar a cobre profundo, parando um pouco antes do marrom; bronze alaranjado é mais comum. Limpidez brilhante. Colarinho espesso, quase branco, cremoso e duradouro.

\textbf{Aroma}: Maltado e rico com característica de cereais lembrando pão assado e/ou de nozes e tostado como casca de pão. Não deve ter notas torradas mais escuras ou de chocolate. A intensidade do malte é de moderada a moderadamente alta. Lúpulo de moderado a baixo complementa, mas não domina o malte e muitas vezes têm um caráter condimentado, apimentado e/ou floral. Caráter de fermentação é muito limpo. Ésteres de baixo a médio-baixo são opcionais.

\textbf{Sabor}: Perfil de malte similar ao aroma, com amargor de lúpulo assertivo de médio a alto, equilibrando os sabores maltados ricos. O final da cerveja é de médio-seco a seco com retrogosto como cereais, amargo e maltado rico. O final é duradouro, as vezes com uma impressão amendoada ou agridoce. O nível de amargor aparente as vezes é mascarado pelo caráter de malte se a cerveja não for muito seca, mas o amargor tende a crescer junto com a riqueza de malte para manter o equilíbrio. Sem torrado. Sem aspereza. Perfil limpo de fermentação. Ésteres frutados leves, especialmente frutas escuras, podem estar presente. Lúpulo condimentado, apimentado e/ou floral de baixa a média intensidade. Leve caráter mineral é opcional.

\textbf{Sensação na Boca}: Corpo médio. Suave. Carbonatação de média a média-alta. Adstringência de baixa a nenhuma.

\textbf{Comentários}: Exemplares clássicos tradicionais localizados na \textit{Altstadt} (cidade velha) são servidos de barril. A maior parte dos exemplares tem um amargor equilibrado (25-35 IBU) e não o caráter agressivo de lúpulo da conhecida \textit{Zum Uerige}. Cervejas mais fortes conhecidas como \textit{sticke} e \textit{doppelsticke} devem ser inscritas no estilo 27 Historical Beer.

\textbf{História}: Desenvolvida em Düsseldorf no final do século XIX para usar técnicas de produção de lagers e competir com lagers. Estilos alemães mais antigos eram produzidos na área, mas não têm relação com a Altbier moderna.

\textbf{Ingredientes}: O perfil dos maltes varia, mas geralmente consiste de maltes base alemães (geralmente Pils, às vezes Munich) com pequenas quantidades de maltes crystal, chocolate e/ou black. Pode ter trigo, inclusive trigo torrado. Lúpulo Spalt é tradicional, mas outros lúpulos tradicionais alemães ou tchecos podem ser usados. Levedura ale limpa com alta atenuação. Fermentada a temperaturas de ale em seu limite inferior e depois acondicionada a frio.

\textbf{Comparação de Estilo}: Mais amarga e maltada do que a International Amber Lagers. Um tanto similar a California Common tanto em técnica de produção quanto em sabor e cor final, no entanto não em ingredientes. Menor teor alcoólico, menor riqueza maltada e mais amargor que uma Dunkles Bock. Mais seca, mais rica e mais amarga do que uma Vienna Lager.

\begin{tabular}{@{}p{35mm}p{35mm}@{}}
  \textbf{Estatísticas}: & OG: 1,044 - 1,052 \\
  IBU: 25 - 50 & FG: 1,008 - 1,014 \\
  SRM: 9 - 17 & ABV: 4,3\% - 5,5\%
\end{tabular}

\textbf{Exemplos Comerciais}: Bolten Alt, Diebels Alt, Füchschen Alt, Original Schlüssel Alt, Schlösser Alt, Schumacher Alt, Uerige Altbier.

\textbf{Última Revisão}: Altbier (2015)

\textbf{Atributos de Estilo}: amber-ale-family, amber-color, bitter, central-europe, lagered, standard-strength, top-fermented, traditional-style
