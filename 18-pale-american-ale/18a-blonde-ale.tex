\phantomsection
\subsection*{18A. Blonde Ale}
\addcontentsline{toc}{subsection}{18A. Blonde Ale}

\textbf{Impressão Geral}: Cerveja artesanal americana fácil de beber, acessível e orientada para o malte, muitas vezes com notas interessantes de frutas, lúpulo ou caráter maltado. Bem equilibrada e límpida, é uma cerveja refrescante sem sabores agressivos.

\textbf{Aparência}: Cor de amarelo claro a dourado profundo. Transparente a brilhante. Colarinho branco, com formação de baixa a média, com retenção de razoável a boa.

\textbf{Aroma}: Aroma maltado de leve a moderado, geralmente neutro ou como cereais, possivelmente com uma leve nota de pão ou de caramelo. Frutado de baixo a moderado é opcional, mas aceitável. Pode ter um aroma de lúpulo de baixo a médio e pode refletir quase qualquer variedade de lúpulo, embora notas cítricas, florais, frutadas e condimentadas sejam comuns. Perfil limpo de fermentação.

\textbf{Sabor}: Maltado inicial suave, mas também pode ter sabor de malte com caráter leve (por exemplo, pão, torrada, biscoito, trigo). Sabores de caramelo geralmente ausentes; se presentes, são tipicamente notas de caramelo ou mel de baixa cor. Ésteres frutados de baixo a médio são opcionais, mas são bem-vindos. Sabor de lúpulo leve a moderado (qualquer variedade), mas não deve ser excessivamente agressivo. Amargor de médio baixo a médio, mas o equilíbrio é normalmente para o malte ou mesmo entre malte e lúpulo. Final de meio seco a levemente maltado; uma impressão de dulçor é, muitas vezes, devido ao menor amargor do que o dulçor residual. Perfil limpo de fermentação.

\textbf{Sensação na Boca}: Corpo de médio leve a médio. Carbonatação de média a alta. Suave, mas sem ser pesado.

\textbf{Comentários}: As versões oxidadas podem desenvolver notas de caramelo ou mel, que não devem ser confundidas com sabores similares derivados do malte. Às vezes conhecido como Golden Ale ou simplesmente Gold.

\textbf{História}: Um estilo, de cerveja artesanal, americano produzido como uma alternativa de produção mais rápida às lagers americanas padrão. Acredita-se que foi produzido pela primeira vez, em 1987, na Catamount. Muitas vezes posicionada cerveja da casa como nível de entrada.

\textbf{Ingredientes}: Geralmente puro malte, mas pode incluir malte de trigo ou adjuntos de açúcar. Qualquer variedade de lúpulo pode ser usada. Levedura pode ser americana limpa, inglesa levemente frutada ou do tipo Kölsch. Também pode ser feito com levedura lager ou acondicionada a frio.

\textbf{Comparação de Estilos}: Normalmente tem mais sabor do que a American Lager e a Cream Ale. Menos amargor que uma American Pale Ale. Talvez semelhante a alguns exemplos mais maltados de Kölsch.

\begin{tabular}{@{}p{35mm}p{35mm}@{}}
  \textbf{Estatísticas} & OG: 1,038 - 1,054 \\
  IBU: 15 - 28  & FG: 1,008 - 1,013  \\
  SRM: 3 - 6  & ABV: 3,8\% - 5,5\%
\end{tabular}

\textbf{Exemplos Comerciais}: Firestone Walker 805, Kona Big Wave Golden Ale, Real Ale Firemans \#4 Blonde Ale, Russian River Aud Blonde, Victory Summer Love, Widmer Citra Summer Blonde Brew.

\textbf{Última Revisão}: Blonde Ale (2015)

\textbf{Atributos de Estilo}: any-fermentation, balanced, craft-style, north-america, pale-ale-family, pale-color, standard-strength
