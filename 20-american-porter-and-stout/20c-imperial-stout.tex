\phantomsection
\subsection*{20C. Imperial Stout}
\addcontentsline{toc}{subsection}{20C. Imperial Stout}

\textbf{Impressão Geral}: Uma \textit{stout} de sabor intenso, com teor alcoólico muito alto, muito escura, com uma ampla gama de interpretações. Malte torrado ou queimado com uma profundidade de sabores de frutas escuras ou secas e um final amargo, adocicado e que traz aquecimento alcoólico. Apesar dos sabores intensos, os componentes precisam se fundir para criar uma cerveja complexa e harmoniosa, não uma grande bagunça – que às vezes só é atingido com envelhecimento.

\textbf{Aparência}: A cor varia de marrom avermelhado muito escuro a preto intenso. Opaca. Colarinho castanho profundo a marrom escuro. Geralmente tem um colarinho bem formado, embora a retenção da espuma possa ser de baixa a moderada. O alto teor alcoólico e viscosidade podem ser visíveis como lágrimas.

\textbf{Aroma}: Aroma com caráter rico, profundo, complexo e muitas vezes bastante intenso, com uma agradável mistura de torra, frutas, lúpulo e álcool. Torra de leve a moderadamente forte pode ter uma qualidade de café, chocolate meio amargo ou amargo, cacau, alcaçuz, alcatrão ou grãos levemente queimados, às vezes com um leve dulçor de caramelo ou malte tostado. Ésteres de baixos a moderadamente fortes, muitas vezes percebidos como frutas escuras ou secas, como ameixas, ameixas secas, figos, groselhas pretas ou uvas-passas. Lúpulos de muito baixos a bem agressivos, muitas vezes de caráter inglês ou americano. Sabor de álcool é opcional, mas não deve ser agressivo, quente ou como de solvente. O equilíbrio entre esses quatro componentes principais pode variar muito; nem todos precisam ser perceptíveis, mas os presentes devem ter uma interação fluida. A idade pode adicionar outra dimensão, incluindo uma impressão vínica ou de vinho do Porto, mas não ácida. A idade pode diminuir a intensidade dos aromas.

\textbf{Sabor}: Como o aroma, uma mistura complexa de torra, frutas, lúpulo e álcool (mesmos descritores do aroma aqui são aplicados). Os sabores podem ser bastante intensos, muitas vezes maiores do que no aroma, mas o mesmo aviso sobre o equilíbrio variando muito ainda se aplica. Amargor de médio a agressivamente alto. O malte equilibra e sustenta os outros sabores, podendo ter qualidades de pão, torrada ou caramelo. No final e no palato, podem ser de bastante seca a moderadamente doce, uma impressão que geralmente muda com a idade. Não deve ser xaroposa ou enjoativa. Retrogosto torrado, amargo e com aquecimento. Aplicam-se os mesmos efeitos da idade descritos no aroma.

\textbf{Sensação na Boca}: Corpo de alto a muito muito e mastigável, com uma textura aveludada e deliciosa. O corpo e a textura podem diminuir com a idade. Aquecimento alcoólico suave e delicado deve estar presente e perceptível, mas em segundo plano. Carbonatação de baixa a moderada.

\textbf{Comentários}: Às vezes conhecido como Russian Imperial Stout ou RIS. Existem interpretações variadas com versões americanas com maior amargor e mais caráter torrado e lúpulos com adição tardia, enquanto as variedades inglesas geralmente refletem um caráter de malte especial mais complexo, com um perfil de éster mais aparente. Nem todas as Imperial Stouts têm um caráter claramente ‘inglês’ ou ‘americano’; tudo o que se encontra entre eles também é permitido, e é por isso que é improdutivo definir subtipos estritos. Os juízes devem estar cientes da ampla variedade do estilo e não tentar julgar todos os exemplos como clones de uma cerveja comercial específica.

\textbf{História}: Um estilo com uma herança longa, embora não necessariamente contínua. Traça as raízes das porters inglesas com teor alcoólico mais alto, fabricadas para exportação em 1700, que dizem ter sido populares com a Corte Imperial Russa. Depois que as guerras napoleônicas interromperam o comércio, essas cervejas foram cada vez mais vendidas na Inglaterra. Porém, o estilo acabou se extinguindo até ser popularmente adotado na Inglaterra, na era moderna da cerveja artesanal, como um renascimento da cerveja que era exportada e nos Estados Unidos como uma adaptação, ampliando o estilo com características americanas.

\textbf{Ingredientes}: Malte pale, com significativa quantidade de maltes ou grãos torrados. Adjuntos em flocos são comuns. Leveduras ale e lúpulos, americanos ou ingleses, são típicos. Envelhece muito bem. Cada vez mais usada como cerveja base para muitos estilos especiais.

\textbf{Comparação de Estilos}: Mais escuro e mais torrado do que Barleywines, mas com álcool semelhante. Mais complexo, com uma gama mais ampla de sabores possíveis, do que as *stouts* de menor densidade.

\begin{tabular}{@{}p{35mm}p{35mm}@{}}
  \textbf{Estatísticas} & OG: 1,075 - 1,115 \\
  IBU: 50 - 90  & 1,018 - 1,030 \\
  SRM: 30 - 40  & ABV: 8\% - 12\%
\end{tabular}

\textbf{Exemplos Comerciais}: American -, Bell's Expedition Stout, Great Divide Yeti Imperial Stout, North Coast Old Rasputin Imperial Stout, Oskar Blues Ten Fidy, Sierra Nevada Narwhal Imperial Stout, English -, Thornbridge Saint Petersburg, Courage Imperial Russian Stout, Le Coq Imperial Extra Double Stout, Samuel Smith Imperial Stout, 2SP Brewing Co The Russian.

\textbf{Revisão Anterior}: Imperial Stout (2015)

\textbf{Atributos de Estilo}: bitter, british-isles, craft-style, dark-color, malty, north-america, roasty, stout-family, top-fermented, traditional-style, very-high-strength
