\phantomsection
\subsection*{32A. Classic Style Smoked Beer}
\addcontentsline{toc}{subsection}{32A. Classic Style Smoked Beer}
\textit{Destinado a versões defumadas de Estilos Clássicos, exceto se a cerveja do Estilo Clássico tiver o defumado como parte inerente de sua definição (então, claro, a cerveja deve ser inscrita em seu estilo base, como Rauchbier).}\\
\textbf{Impressão Geral}: Uma fusão bem equilibrada de malte e lúpulos, do estilo base, com um agradável caráter defumado. \\
\textbf{Aparência}: Varia. A aparência deve refletir o estilo base, embora a cor geralmente seja um pouco mais escura que o esperado para o estilo base sem o defumado. \\
\textbf{Aroma}: Um agradável equilíbrio entre o aroma esperado na cerveja base e o malte defumado. O caráter defumado varia de baixo a assertivo e pode ter o caráter derivado de vários tipos de madeira (p. ex., amieiro, carvalho, faia). O equilíbrio entre o defumado e a cerveja pode variar - eles não precisam estar igualmente intensos. Entretanto, a combinação resultante deve ser atrativa. Aroma acentuado, fenólico, áspero, como borracha ou queimado derivado de defumação é inapropriado. \\
\textbf{Sabor}: Similar ao aroma, com o caráter de malte defumado de baixo a assertivo e em equilíbrio com a cerveja base. Cada tipo de madeira produz diferentes perfis de sabores. O equilíbrio entre o defumado e a cerveja pode variar, mas a combinação resultante deve ser agradável. O defumado pode adicionar alguma secura ao final. Sabor acentuado, amargo, queimado, carbonizado, como borracha, sulfuroso, medicinal ou fenólico derivado de defumação é inapropriado. \\
\textbf{Sensação na Boca}: Varia de acordo com o estilo base da cerveja. Significante adstringência, fenólico ou aspereza derivada de defumação é uma falha. \\
\textbf{Comentários}: Use este estilo para cervejas diferentes do estilo Rauchbier de Bamberg (ou seja, Märzen defumada com madeira de faia), que tem seu próprio estilo. Os juízes devem avaliar essas cervejas principalmente com relação ao equilíbrio geral e em quão bem o caráter defumado aprimora a cerveja base. \\
\textbf{História}: O processo de usar maltes defumados foi adaptado a muitos estilos pelos cervejeiros artesanais. Cervejeiros alemães usaram maltes defumados, tradicionalmente, em cervejas do tipo Bock, Doppelbock, Weissbier, Munich Dunkel, Schwarzbier, Munich Helles, Pils e outros estilos especiais. \\
\textbf{Ingredientes}: Os diferentes elementos usados para defumar o malte resultam em características únicas de sabor e aroma. Maltes defumados por faia e outras madeiras de lei podem ser usados (por exemplo, carvalho, bordo, mesquita, amieiro, pecã, macieira, cerejeira, e outras madeiras de árvores frutíferas). Estas madeiras podem lembrar certos alimentos defumados (p. ex., nogueira com costelas, bordo com bacon ou linguiça, amieiro com salmão). Madeiras perenes nunca devem ser usadas, já que adicionam sabor medicinal, resinoso ao malte. Caráter notável de malte defumado por turfa é universalmente indesejável, devido ao seus fenóis punjentes, com um terroso que lembra poeira. Os outros ingredientes restantes variam de acordo com o estilo base. Se os maltes defumados forem combinados com outros ingredientes não habituais (p. ex., frutas, vegetais, especiarias, mel) em quantidades notáveis, a cerveja resultante deve ser inscrita no estilo 32B Specialty Smoked Beer \\
\textbf{Instruções para Inscrição}: O participante deve especificar um Estilo Base. O participante deve especificar o tipo de madeira ou defumação, se um caráter defumado incomum for notável. \\
\textbf{Estatísticas}: Varia de acordo com o estilo base da cerveja. \\
\textbf{Exemplos Comerciais}: Alaskan Smoked Porter, Schlenkerla Oak Smoke Doppelbock, Schlenkerla Rauchbier Weizen, Schlenkerla Rauchbier Ur-Bock, O’Fallon Smoked Porter, Spezial Rauchbier Lagerbier. \\
\textbf{Última Revisão}: Classic Style Smoked Beer (2015) \\
\textbf{Atributos de Estilo}: smoke, specialty-beer
