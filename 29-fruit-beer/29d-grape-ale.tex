\phantomsection
\subsection*{29D. Grape Ale}
\addcontentsline{toc}{subsection}{29D. Grape Ale}
\textit{Originalmente um estilo italiano que, posteriormente, inspirou cervejeiros em regiões produtoras de uvas em todo o mundo para produzir versões apresentando varietais locais. Veja X3 Italian Grape Ale para a versão local italiana.}\\
\textbf{Impressão Geral}: Combina o perfil de um vinho espumante e uma cerveja base relativamente neutra, permitindo que as qualidades aromáticas da uva se misturem agradavelmente com os aromas de lúpulo e de levedura. Pode variar de refrescante à complexa.

\textbf{Aparência}: A cor pode variar de dourado claro a rubi, mas aquelas que usam uvas vermelhas tendem ao bordô. Estas cores mais escuras podem vir do uso de produtos de uva cozida ou concentrada, nunca de cereais especiais escuros. Espuma de branca a avermelhada, com retenção geralmente média-baixa. A limpidez é geralmente boa. Nunca turva.

\textbf{Aroma}: As características aromáticas da variedade da uva são perceptíveis, mas não devem dominar. O carácter da uva deve se fundir bem com o caráter do malte base subjacente. Embora o aroma do lúpulo seja geralmente contido, pode variar de médio-baixo a totalmente ausente. A fermentação é geralmente bastante limpa, mas pode apresentar notas de especiarias delicadas e ésteres frutados. Banana, tutti-frutti e similares são considerados como falhas.

\textbf{Sabor}: Assim como no aroma, o caráter da uva pode variar a intensidade de sutil a média-alta e ser mais proeminente. Sabores de frutas (tropical, vermelhas, de caroço etc) adequados a variedade da uva utilizada. Uvas vermelhas mais escuras podem contribuir com sabores mais rústicos (p. ex., terroso, tabaco, couro). O caráter de malte é de suporte, não deve ser forte e geralmente deve ser de variedade clara, levemente tostada. Níveis muito baixos de malte cristal claro são permitidos, mas o caráter torrado ou forte de chocolate é sempre inadequado. O amargor é geralmente baixo e os sabores de lúpulo podem ser baixos a inexistentes. Notas ácidas suaves, devido à variedade e quantidade de uva utilizada, são comuns e podem ajudar a melhorar a digestibilidade, mas não devem se aproximar do limite do “azedo”. Carvalho complementar é opcional, mas um caráter \textit{funky} de Brettanomyces não deve estar presente. Fermentação limpa.

\textbf{Sensação na Boca}: A carbonatação de média alta a alta melhora a percepção do aroma. O corpo é geralmente de baixo a médio e alguma acidez pode contribuir para aumentar a percepção de secura. O final é extremamente seco e bem definido. Exemplos fortes no teor alcoólico podem mostrar algum aquecimento alcoólico.

\textbf{Comentários}: O teor alcoólico pode ir de 4,5\% a 12,5\%, mas é mais comum estar na faixa de ABV listada abaixo. A percepção da cor varia amplamente, com base na tonalidade da fruta adicionada.

\textbf{História}: Inicialmente produzida por Birrificio Montegioco e Birrificio Barley em 2006-2007. Tornou-se mais popular depois de ser publicada nas Diretrizes de 2015 como Italian Grape Ale (IGA) e inspirou muitas variações locais em outros países.

\textbf{Ingredientes}: Malte Pilsen ou Pale, uso contido de malte cristal ou de trigo. O mosto de uva (variedades tintas ou brancas, tipicamente mosto fresco) geralmente representa de 15 a 20\% do total do perfil de maltes, mas pode exceder 40\%. O mosto é fermentado com a cerveja e não uma mistura de vinho e cerveja. Leveduras de perfil frutado e picante são as mais comumente usadas, mas podem ser utilizadas variedades neutras. O lúpulo deve ser selecionado para complementar o perfil geral. Esta cerveja não é sujeita a \textit{dry-hopping}. O uso de carvalho é permitido, mas não é obrigatório e não deve ser predominante ou em níveis mais fortes do que os encontrados em vinho.

\textbf{Comparação de Estilos}: Base semelhante a vários estilos belgas, como Belgian Blonde, Saison e Belgian Single, mas com uvas. Exemplos de mais fortes são semelhantes a Belgian Tripel ou Belgian Golden Strong Ale, mas com uvas. Não contém o caráter \textit{funky}, como a Fruit Lambic.

\textbf{Instruções para Inscrição}: O participante deve especificar o tipo de uva utilizada. O participante pode fornecer informações adicionais sobre o estilo base ou ingredientes característicos utilizados.

\begin{tabular}{@{}p{35mm}p{35mm}@{}}
  \textbf{Estatísticas}: & OG: 1,059 - 1,075 \\
  IBU: 10 - 30  & FG: 1,004 - 1,013  \\
  SRM: 4 - 8  & ABV: 6\% - 8,5\%
\end{tabular}

\textbf{Exemplos Comerciais}: Montegioco Open Mind, Birrificio del Forte Il Tralcio, Luppolajo Mons Rubus, Firestone Walker Feral Vinifera, pFriem Family Brewers Druif, 4 Árvores Abbondanza.

\textbf{Atributos de Estilo}: fruit, specialty-beer
