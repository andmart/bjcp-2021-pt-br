\phantomsection
\subsection*{15B. Irish Stout}
\addcontentsline{toc}{subsection}{15B. Irish Stout}

\textbf{Impressão Geral}: Uma cerveja preta com um sabor torrado pronunciado, muitas vezes semelhante ao café. O equilíbrio pode variar do razoavelmente uniforme a muito amargo, com as versões mais equilibradas com um pouco de dulçor maltado e as versões amargas sendo bastante secas. As versões de barril normalmente são cremosas pelo serviço com nitro, mas as versões engarrafadas não terão esse caráter. O sabor torrado.

\textbf{Aparência}: Cor muito preta a marrom profundo com reflexos granada. De acordo com a Guinness, “a cerveja Guinness pode parecer preta, mas na verdade é um tom muito escuro de rubi”. Opaca. Um colarinho espesso, cremoso, durador, de cor castanho claro a marrom é característico quando servida com nitro, mas não espere um colarinho firme e cremoso em uma cerveja engarrafada.

\textbf{Aroma}: Aroma moderado de café normalmente domina; pode ter notas secundárias leves de chocolate amargo, cacau ou cereais torrados. Ésteres médio-baixo opcionais. Aroma baixo de lúpulo terroso ou floral é opcional.

\textbf{Sabor}: Sabor moderado de cereais torrados ou malte com um amargor médio a alto. O final pode ser seco e como café, a moderadamente equilibrado com um toque de caramelo ou dulçor maltado. Normalmente tem sabores semelhantes ao café, mas também pode ter um caráter de chocolate amargo ou meio amargo no paladar, durando até o final. Os fatores de equilíbrio podem incluir alguma cremosidade, frutado médio-baixo ou médio sabor de lúpulo terroso. O nível de amargor é um pouco variável, assim como o caráter torrado e a secura do final; permitindo interpretação pelos cervejeiros.

\textbf{Sensação na Boca}: Corpo médio-leve a médio-cheio, com um caráter um tanto cremoso – especialmente quando servido com nitro. Carbonatação baixa a moderada. Pelo alto amargor do lúpulo e proporção significativa de cereais escuros presentes, esta cerveja é notavelmente suave. Pode ter uma leve adstringência dos cereais torrados, embora a aspereza seja indesejável.

\textbf{Comentários}: Tradicionalmente um produto de barril. Exemplos modernos estão quase sempre associados a um serviço com nitro. Não espere que as cervejas engarrafadas tenham a textura cheia ou cremosa ou o colarinho de longa duração associados ao serviço com mistura de gases. Existem diferenças regionais na Irlanda, semelhantes à variabilidade nas Bitters inglesas. As stouts de Dublin usam cevada torrada, são mais amargas e mais secas. As stouts da Cork são mais doces, menos amargas e têm sabores de chocolate e maltes especiais.

\textbf{História}: O estilo evoluiu a partir da London Porters, mas refletindo um corpo mais cheio, mais cremoso, mais robusto* e forte. A Guinness começou a fabricar apenas porter em 1799, e um “tipo mais robusto de porter” por volta de 1810. A Irish Stout divergiu da London Single Stout (ou simplesmente Porter) no final de 1800, com ênfase em maltes mais escuros e cevada torrada. A Guinness começou a usar cevada em flocos após a Segunda Guerra Mundial, e a Guinness Draft foi lançada como marca em 1959. O dispositivo similar ao chope para latas e garrafas (“widget”) foi desenvolvido no final dos anos 1980 e 1990.

\textbf{Ingredientes}: Maltes ou cereais torrados escuros, o suficiente para tornar a cerveja preta. Malte pale. Pode usar cereais não maltados para o corpo.

\textbf{Comparação de Estilos}: Força inferior à de uma Irish Extra Stout. Mais escuro na cor (preto) do que uma English Porter (marrom).

\begin{tabular}{@{}p{35mm}p{35mm}@{}}
  \textbf{Estatísticas}: & OG: 1,036 - 1,044 \\
  IBU: 25 - 45  & FG: 1,007 - 1,011 \\
  SRM: 25 - 40  & ABV: 4\% - 4,5\%
\end{tabular}

\textbf{Exemplos Comerciais}: Beamish Irish Stout, Guinness Draught, Harpoon Boston Irish Stout, Murphy's Irish Stout, O'Hara's Irish Stout, Porterhouse Wrasslers 4X.

\textbf{Última Revisão}: Irish Stout (2015)

\textbf{Atributos de Estilo}: bitter, british-isles, dark-color, roasty, standard-strength, stout-family, top-fermented, traditional-style